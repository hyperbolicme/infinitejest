\documentclass[12pt,
               color={usenames,   
                      dvipsnames},% Since beamer loads color package 
                                  %  automatically \usepackage{color}   
                                  %  cannot be used to pass options.
%             draft,             % Draft WIP version without header &
%                                  %  footer. Faster compilation.
%               handout,           % Handout version. not sure what it does.
%              notes              % To print notes.
                    ]{beamer}   

%\includeonlyframes{FR_MOTIVE} %FR_PROBLEMS} %FR_EX_IMAGE,FR_EX_TERM}              
                                 %  Use when drafting for fast compile
                                 % turnaround time. 

%%%%%%%%%%%%%%%%%%%%%%%%%%%%%%%%%%%%%%%%%%%%%%%%%%%%%%%%%%%%%%%%%%%%%%%%%%%%%
% Theme customization

\usetheme[compress]{Ilmenau}    % Small circle based TOC on top bar
                                %  single line.
%\usetheme[width=1in]{Hannover}  % TOC left side bar.
%\usetheme{Szeged}               % Similar to Ilmenau except borders on
%                                %  top bar and bottom.
%\usecolortheme  {beetle}        % Grey and blue based.
\useinnertheme  {circles}       % Circle dingbats with no shading.
\usefonttheme[onlylarge]
                {structurebold} % Structure fonts are
                                % bold. Set to only the title / big
                                % fonts.
%\useoutertheme[compress,
%               footline=empty]
%               {miniframes}     % Don't use with explicit themes 

% Beamer elements customization: setbeamercolor/-font/-template
% \setbeamercolor{title}                {fg=light3,bg=light1}
% \setbeamercolor{subtitle}             {fg=dark1,bg=light1}
% \setbeamercolor{date}                 {fg=dark1}
% \setbeamercolor{institute}            {fg=dark1}
% \setbeamercolor{frametitle}           {fg=dark1,bg=light1}
% \setbeamercolor{framesubtitle}        {fg=light5}
% \setbeamercolor{structure}            {fg=dark1}
% \setbeamercolor{palette primary}      {fg=light1,bg=light2} % Palette
%                                               % primary, secondary,
%                                               % tertiary, quaternary
%                                               % for headline and
%                                               % footline 
% \setbeamercolor{palette secondary}    {fg=dark2,bg=light3}
% \setbeamercolor{palette tertiary}     {fg=light1,bg=light3} 
% \setbeamercolor{palette quaternary}   {fg=light1,bg=light2} 
% \setbeamercolor{normal text}          {fg=dark2,bg=light1} 
% %\setbeamercolor{block body alerted}   {fg=white} %isn't working
% \setbeamercolor{alerted text}         {fg=white} % contrast}  %light3} 
% \setbeamercolor{block body}           {bg=light5}
% \setbeamercolor{block title}          {fg=light1} %,bg=light3} % Uncomment 
%                                               % for different block
                                              % heading background

% \setbeamertemplate{blocks}[rounded] % Blocks are rounded on the
%                                     % corners
\setbeamertemplate{blocks}[default] % Blocks are not rounded on the
                                    % corners

\setbeamerfont{framesubtitle}{series={\fontseries{c}}} % Condensed
                                %shape=\sc             % Small caps

%%%%%%%%%%%%%%%%%%%%%%%%%%%%%%%%%%%%%%%%%%%%%%%%%%%%%%%%%%%%%%%%%%%%%%%%%%%%%
% Packages used
\usepackage{../lib/mylatexlib}
\usepackage[normalem]{ulem}
\usepackage{setspace}
\usepackage{latexsym}
\usepackage{amssymb}
\usepackage{amsfonts}
\usepackage{amsmath}
%\usepackage{epsfig}
%\usepackage{graphicx}
%\usepackage{epstopdf}
\usepackage{algorithm}
\usepackage{algorithmic}
\usepackage{enumerate}
\usepackage[normalem]{ulem}  % Underline. normalem keeps the default
                             % \em to be italics. 
\usepackage{textcomp}

\usepackage{stmaryrd}

%\usepackage{pifont}
%\usepackage{natbib}
%\usepackage[finalnew]{../lib/trackchanges}


%%%%%%%%%%%%%%%%%%%%%%%%%%%%%%%%%%%%%%%%%%%%%%%%%%%%%%%%%%%%%%%%%%%%%%%%%%%%%
% Loading spcial fonts
\DeclareMathAlphabet{\mathpzc}{OT1}{pzc}{m}{it}
\DeclareMathAlphabet{\mathcalligra}{T1}{calligra}{m}{n}


% %%%%%%%%%%%%%%%%%%%%%%%%%%%%%%%%%%%%%%%%%%%%%%%%%%%%%%%%%%%%%%%%%%%%%%%%%%%%%
% % Color definitions

% %% Inspired by Kuler - Vintage Beach Wear -
% %% http://kuler.adobe.com/#themeID/1507195 
% \definecolor{dark1}      {HTML}{8A866A}   % Goldish gray     
%                                           %     [Structure, Block heading]
% \definecolor{light1}     {HTML}{FFF0C2}   % Pale amber          
%                                           %     [Slide BG] 
% \definecolor{light2}     {HTML}{9C968B}     % Gambogeish gray [???]
% \definecolor{light3}     {HTML}{A36D5C}   % Grayish vermilion   
%                                           %     [Title, Header & footer BG] 
% \definecolor{dark2}      {HTML}{473C35}   % Dark grayish tangelo
%                                           %     [Slide normal text FG]
% \definecolor{light5}     {HTML}{A8A87D}   % Greyish olive
%                                           %     [Block BG]
% \definecolor{contrast}   {HTML}{197DE5}   % Brilliant azure     
%                                           %     [Stark Contrast]



%%%%%%%%%%%%%%%%%%%%%%%%%%%%%%%%%%%%%%%%%%%%%%%%%%%%%%%%%%%%%%%%%%%%%%%%%%%%%
% String defs
\def\cA{{\cal A}}  \def\cB{{\cal B}}  \def\cC{{\cal C}}  \def\cD{{\cal D}}
\def\cE{{\cal E}}  \def\cF{{\cal F}}  \def\cG{{\cal G}}  \def\cH{{\cal H}}
\def\cI{{\cal I}}  \def\cJ{{\cal J}}  \def\cK{{\cal K}}  \def\cL{{\cal L}}
\def\cM{{\cal M}}  \def\cN{{\cal N}}  \def\cO{{\cal O}}  \def\cP{{\cal P}}  
\def\cQ{{\cal Q}}  \def\cR{{\cal R}}  \def\cS{{\cal S}}  \def\cT{{\cal T}}  
\def\cU{{\cal U}}  \def\cV{{\cal V}}  \def\cW{{\cal W}}  \def\cX{{\cal X}}
\def\cY{{\cal Y}}  \def\cZ{{\cal Z}}  \def\hA{{\hat A}}  \def\hB{{\hat B}}
\def\hC{{\hat C}}  \def\hD{{\hat D}}  \def\hE{{\hat E}}  \def\hF{{\hat F}}
\def\hG{{\hat G}}  \def\hH{{\hat H}}  \def\hI{{\hat I}}  \def\hJ{{\hat J}}
\def\hK{{\hat K}}  \def\hL{{\hat L}}  \def\hP{{\hat P}}  \def\hQ{{\hat Q}}
\def\hR{{\hat R}}  \def\hS{{\hat S}}  \def\hT{{\hat T}}  \def\hX{{\hat X}}
\def\hY{{\hat Y}}  \def\hZ{{\hat Z}}
\def\A{{\mathcal A}}  \def\bI {\mathbb I}
\def\C{{\mathcal C}}  \def\bO {\mathbb O}
\def\F{{\mathcal F}}  \def\cl {\mathpzc{l}}
\def\H{{\mathcal H}}  \def\cg {\mathpzc{g}}
                      \def\ccT{\mathpzc{T}}

\def\overlap   {\between}
\def\eps       {\epsilon}
\def\icppl     {\maltese} 
\def\invb      {\textreferencemark}
\def\assign    {\leftarrow}

\def\lndisplay      {1}
\def\commentboxsize {7cm}
\def\prelimspace    {2mm}
\def\alertseccolor {contrast}

% For BibTeX formatting \newblock
\def\newblock{\hskip .11em plus .33em minus .07em}
% for Natbib
%\bibpunct{(}{)}{;}{a}{,}{,}


% %%%%%%%%%%%%%%%%%%%%%%%%%%%%%%%%%%%%%%%%%%%%%%%%%%%%%%%%%%%%%%%%%%%%%%%%%%%%%
% % New/renew commands
% % Format of comments in algorithmic package
% \renewcommand{\algorithmiccomment}[1]
% { 
%   \vspace {0.5mm}
%   \hfill
%   {\small
%   \begin{tabular}{|r}
%     \parbox[right]{\commentboxsize}{ \space \tt{ #1 }}\\  
%     % {\tt /* #1 */} \hspace{2mm}
%   \end{tabular}
%   }
% }


% % Theorems etc. 
% \newtheorem{observation}{Observation}
%%% 1 DEC
% \newcommand{\seq}[1]{\left\langle #1 \right\rangle}
% \newcommand{\set}[1]{\left\{ #1\right\}}
% \newcommand{\supp}[1]{supp\left( #1 \right)}
%%% END 1 DEC

%%%%%%%%%%%%%%%%%%%%%%%%%%%%%%%%%%%%%%%%%%%%%%%%%%%%%%%%%%%%%%%%%%%%%%%%%%%%%
% Title slide details
\title[thesis evaluation - a generalization of consecutive-ones property]
         {Thesis Evaluation}
\subtitle{A discussion of review comments}

\author[anju s $|$ \tt{anjuzabil@gmail.com}]{Anju Srinivasan}

\institute[CS09S012]
         {thesis titled {\em A Generalization of Consecutive Ones
             Property} \\
           as part of {\bf M.\;S.} by Research \\ 
          advised by {\bf Dr.\;N.\;S.\;Narayanaswamy}\\ 
          CSED, IITM, Chennai - 36}

\date{25 Mar 2013}


%%%%%%%%%%%%%%%%%%%%%%%%%%%%%%%%%%%%%%%%%%%%%%%%%%%%%%%%%%%%%%%%%%%%%%%%%%%%%
% Document begins

\begin{document}

\frame{
 \titlepage
}
\section*{Outline}
\frame{
  \tableofcontents % [pausesections]
}

\section{Reviewers}
\frame{
  \frametitle{Reviewers}
  \begin{enumerate}
  \item {\bf Dr. C. R. Subramaniam}\\ Institute of Mathematical
    Sciences, Chennai
  \item {\bf Dr. Subhas Chandra Nandy}\\ Indian Statistical Institute,
    Kolkata
  \end{enumerate}
}


\section{Comments from Reviewers}

\subsection{Comments from Reviewer 1}

\frame{\center \large Comments from Reviewer 1}

\frame{
 \frametitle{{Missing citation}}
  \begin{block}{}
    The result on characterization of TPL in the thesis is a variant
    of Fournier \cite{f80} on characterization of hypergraph
    isomorphism who showed that two hypergraphs $H_1 = (V_1, X)$ and
    $H_2 = (V_2, Y)$ such that $|X| = |Y| = m$ are isomorphic iff there
    is a bijection $\phi$ on $I = \{1, ... , m\}$ s.t. for every $L
    \subset I, |\bigcap_{i \in L}X_i| = |\bigcap_{j=\phi(i),i \in
      L}Y_j|$
  \end{block}
  \begin{block}{Response}
    \begin{itemize}
    \item Yes because feasibility of TPL requires equicardinality of
      intersections of at most 3 edges.

    \item This citation is added to Conclusions chapter Page 72.
  \end{itemize}
\end{block}
}


\frame{
  \frametitle{Corrections in definitions}
  \begin{block}{Comments 0, 1, 2, 3, 4}
    Missing conditions/terms to basic preliminary definitions in
    Chapter 1 and addition of Graph isomorphism definition for
    completeness.
  \end{block}
  \begin{block}{Response}
    \begin{itemize}
    \item Suggested changes made to definitions of Poset, Undirected
      graphs, Tree, Interval graph, Path graph.
    \item Graph Isomorphism definition has been added.
    \item Pages 7, 8.
    \end{itemize}
\end{block}
}

\frame{
  \frametitle{Clarity of problem description (TPL)}
  \begin{block}{Comment 5}
    Page 12, 3rd para of Sec 1.5 the idea described is not clear and precise.
  \end{block}
  \begin{block}{Response}
    \begin{itemize}
    \item
    An informal description of {\sc Compute Feasible Tree Path
      Labeling} and its motivation in the Introduction chapter.
  \item Link
    to formal definition in Chapter 3 added.
  \item Page 12.
  \end{itemize}
\end{block}
}


\frame{
  \frametitle{Clarity of definition (Operator '/')}
  \begin{block}{Comment 6}
    Definition 2.2.4(7) of binary operator ``/'' is not precise. What
    is meant by ``$b$ is a new element added..''?
  \end{block}
  \begin{block}{Response}
    \begin{itemize}
    \item Used by \cite{mm96} to decompose permutation of $A$ to that
      of $A \setminus B$ and permutation of $B$.
    \item $A / B = A \setminus B \cup \{b\}$ if $B \subseteq A$.
    \item $b$ is changed to $x_B$ for clarity.
    \item Page 26, 27

    \end{itemize}
 \end{block}
}

\frame{
  \frametitle{Clarity of definition (Incompatibility graph)}
  \begin{block}{Comment 7}
    Definition 2.3.1: the definition of $A_\cF$ is not clear. Where is the dependence on $\cF$?
  \end{block}
  \begin{block}{Response}
    \begin{itemize}
    \item $A_\cF = \{(a,b) \mid a, b \in U, a \ne b\}$ is the vertex set of the incompatibility graph of
      $\cF$ in \cite{mcc04}
    \item but has no dependence on $\cF$, only on its universe $U$.
    \item $A_\cF$ changed to $A_U$
    \item Page 36
    \end{itemize}
 \end{block}
}

\frame{
 \frametitle{Misunderstood definition ($k$-subdivided star)}
  \begin{block}{}
    A $k$-subdivided \sout{tree} [star] is a tree with a single central vertex with
    a number of paths each of $k$ edges emanating from it.
  \end{block}
  \begin{block}{Response}
    \begin{itemize}
    \item This is incorrect/a typo by reviewer.
    \item Def 3.2.6 clearly specifies paths have $k+2$ vertices
      i.e. $k+1$ edges.
   \end{itemize}
  \end{block}
}

\subsection{Comments from Reviewer 2}
\frame{\center \large Comments from Reviewer 2}

\frame{
 \frametitle{Missing definition}
  \begin{block}{}
    The definition of $k$-subdivided star is not given in the thesis.
  \end{block}
  \begin{block}{Response}
    \begin{itemize}
    \item Was in fact defined in Def 3.2.6.
    \item Link to formal definition given at appropriate places.
    \item Page 13, 38
  \end{itemize}
\end{block}
}


\frame{
 \frametitle{Clarity of proof}
  \begin{block}{}
    The proof of Lemma 3.3.4 is unclear. 
  \end{block}
  \begin{block}{Response}
    \begin{itemize}
    \item Proof was correct but terse.
    \item Has been elaborated.
    \item Page 46
  \end{itemize}
\end{block}
}

\def \filteri {{\tt filter common leaf} }
\def \filterii {{\tt filter fix leaf} }

\frame{
 \frametitle{Clarity in statement of problem}
  \begin{block}{}
    In Sec 3.3 two filters are mentioned and their pseudocodes
    given. If possible state problem in words, explicitly state roles
    of filters with proper figures.
  \end{block}
  \begin{block}{Response}
    \begin{itemize}
    \item Summary of problems stated before pseudocode and narrative of \filteri and
      \filterii.
    \item Page 44
    \item Refer Sec 3.4 for illustrated example with figures.
  \end{itemize}
\end{block}
}


\def \FOCPS {\textsc{Find Overlap Component Partition Subtrees} }
\def \FMFT {\textsc{Find MUB Feasible TPL} }

\frame{
 \frametitle{Addition of formal problem definition}
  \begin{block}{}
    The two subproblems of TREE PATH LABELING PROBLEM on arbitrary
    trees are not formally defined. Given description is with several
    notations and the motivation is unclear.
  \end{block}
  \begin{block}{Response}
    \begin{itemize}
    \item Notations necessary - long abstract theory set up cannot be avoided..
    \item Formal definitions added \FOCPS, \FMFT
    \item Motivation summarised before long theory set up
    \item Page 61, 66
  \end{itemize}
\end{block}
}

\section{Results in thesis}


\section{Q\&A}

\frame{
  \frametitle{Thank You}
  \begin{centering}
   Q \& A
  \end{centering}
}

\section*{References}
\frame%[allowframebreaks]{References} 
{
\tiny  %\footnotesize
\bibliographystyle{alpha}
\bibliography{../lib/cop-variants__thesis}
}

% \frame{
%   % TEMP: to enforce toc entries when frames haven't been decided
%   % yet. 
% } 

\end{document}
