\xclearpage

\definecolor{dim}{rgb}{0.7,0.7,0.7}
\def \dimspace {0in} %{0.125in}
\def \dimspacemore {0.5in} %{0.125in}
\def \authspace {-0.25in}
\def \brightspace {0.375in}


{%
  \thispagestyle{empty}
  \fontencoding{T1}%
  \fontfamily{bch} %{put} %{bch} % {pbk} % \fontfamily{cmdh}
  % \fontfamily{pag}
  \selectfont%
 
  \vglue -1.5in % this changes top margin 
%   \vskip -5ex%
%   \hspace{5in}\parbox{1in}{\raggedright Hello world}

  \hspace{-0.75in}%
  \parbox[t]{6.5in}{%
    \Large \noindent \raggedright%
    \color{dim}{%
      % One teacher called these moments 'mathematical
      % experiences'. What I didn't know then was that
      [...] a mathematical experience was aesthetic in nature, an
      epiphany in Joyce's original sense. These moments appeared in
      proof-completions, or maybe algorithms. Or like a gorgeously
      simple solution you suddenly see after filling half a notebook
      with gnarly attempted solutions. It was really an experience of
      what I think Yeats called ``the click of a well made box''.%
      % The word I always think of is as 'click'.
      {%
        \vspace{\authspace}
        \small \par \hfill%
        D.\,F.\,W.%
      }%
    }%
  }%

  \vspace{\dimspace}
    
 
  \hspace{1in}%
  \parbox{4in}{ }{%
    \LARGE 
    \raggedleft \noindent%
    \color{dim}{%
      Math Graffiti: Kilroy wasn't Haar.  Free the group.  Nuke the kernel.  Power to
      the $n$.  $N = 1 \Rightarrow P = NP$ %
      {%
%        \vspace{\authspace}%
        \small \par \hfill%
        \emph{Concrete Mathematics (margin notes)}%
      }%
    }%
  }%

  \vspace{\dimspace}
    

  \hspace{-1in}%
%  \ding{125}
  \parbox[t]{5in}{%
    \large \raggedright \noindent %
    \color{dim}{%
      The process of preparing programs for the digital computer is
      especially attractive, not only because it can be economically
      and scientifically rewarding, but also because it can be an
      aesthetic experience much like composing poetry or music.%
      {%
%        \vspace{\authspace}%
        \small \par \hfill \emph{The Art of Computer Programming},
        D.\,E.\,K.%
      }%
    }%
  }%

  
  \vspace{\brightspace}

  {%
      \centering%
      \hspace{-1.25in}
      \parbox{8in}{%
      \fontsize{36}{1.5}\selectfont%
      \centering \noindent%
      \emph{L'art c'est la solution au chaos.}%
      \par \noindent%
      \normalsize (Art is the solution to chaos.)%
    }%
  }%
  
  \vspace{\brightspace}
    
 \hspace{-0.25in}%
  \parbox[t]{6.5in}{%
    \LARGE  \noindent \raggedleft%
    \color{dim}{%
      \textbf{myth\textperiodcentered os} {|{\huge $'$mi $\theta$
          $\overline{o}$s}|} \\
      a set of beliefs or assumptions about something
      \emph{ :\\ 
      the rhetoric and mythos of science create the comforting image
      of linear progression toward truth.}%
      {%
%        \vspace{\authspace}%
        \small \par \hfill%
        \emph{New Oxford American Dictionary, 2nd
          Ed.}%
     }%
    }%
  }



  \vspace{\dimspace}

  \hspace{-1in}%
  \parbox{6in}{%
    \singlespacing
    \large \raggedright \noindent%
    \color{dim}{%
      Anything that happens, happens.  Anything that, in happening,
      causes something else to happen, causes something else to
      happen.  Anything that, in happening, causes itself to happen
      again, happens again.  It doesn't necessarily do it in
      chronological order, though.%
      {%
%       \vspace{\authspace}%
        \small \par \hfill%
        \emph{The Salmon of Doubt}, D.\,N.\,A.%
      }%
    }%
  }%

  \vspace{\dimspace}


  \hspace{1.5in}%
  \parbox{4in}{%
    %\huge 
    \Large 
    \noindent \raggedleft%
    \color{dim}{%
      Everything in its right place.%
      {%
        \small \par \hfill%
       \emph{Kid A, Radiohead}, T.\,Y.%
      }%
    }%
  }%

%   \hspace{1.5in}%
%   \parbox{4in}{%
%     %\huge 
%     \Large 
%     \noindent \raggedleft%
%     \color{dim}{%
%       This is water, this is water.%
%       {%
% %        \vspace{\authspace}%
%         \small \par \hfill%
%         D.\,F.\,W.%
%       }%
%     }%
%   }%
  % xxxx xx xxx xxxx xx xxx xxxx xx xxx xxxx xx xxx xxxx xx xxx xxxx
  % xx xxx xxxx xx xxx xxxx xx xxx xxxx xx xxx xxxx xx xxx xxxx xx xxx
  % xxxx xx xxx xxxx xx xxx xxxx xx xxx xxxx xx xxx xxxx xx xxx xxxx
  % xx xxx xxxx xx xxx xxxx xx xxx xxxx xx xxx xxxx xx xxx xxxx xx xxx
  % xxxx xx xxx xxxx xx xxx xxxx xx xxx xxxx xx xxx xxxx xx xxx xxxx
  % xx xxx xxxx xx xxx xxxx xx xxx xxxx xx xxx xxxx xx xxx xxxx xx xxx
  % xxxx xx xxx xxxx xx xxx xxxx xx xxx xxxx xx xxx xxxx xx xxx xxxx
  % xx xxx xxxx xx xxx xxxx xx xxx xxxx xx xxx xxxx xx xxx xxxx xx xxx

  \vskip -02.5in % this changes bottom margin (hopefully)
    
}
