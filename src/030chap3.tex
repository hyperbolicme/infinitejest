\chapter{Tree Path Labeling of Path Hypergraphs - New Results}
\label{ch:myresearch}

This chapter documents all the new results obtained by us
in the area of tree path labeling of path hypergraphs which is the
parent problem addressed in this thesis.

Section~\ref{sec:myresearchintro} recalls the idea of tree path
labeling. The necessary preliminaries with definitions etc. are
presented in Section~\ref{ch:prelims}. Section~\ref{sec:feasible}
documents the characterization of a feasible path labeling for any
path labeling from any target trees. Section~\ref{sec:spltargettree}
describes two special cases where the target tree is of a particular
family of trees. The first one is shown to be equivalent to COP
testing in Section~\ref{sec:icpplicpia}. Section~\ref{sec:ksubdivstar}
discusses the second special case and presents a polynomial time
algorithm to find the tree path labeling of a given set system from a
given \kstar. Section~\ref{sec:norestraint} discusses the plain
vanilla version where the target tree has no restrictions and the
algorithm to find a feasible TPL, if any, in this case.


\section{Summary of Proposed Problems}
\label{sec:myresearchintro}
In Section~\ref{sec:motive} we see that consecutive-ones property and
its equivalent problem of interval labeling of a hypergraph is a
special case of the general problem of tree path labeling of path
hypergraphs.  The problem of consecutive-ones property testing can be
easily seen as a simple constraint satisfaction problem involving a
hypergraph or a system of sets from a universe. Every column (row) of
the binary matrix can be converted into a set of non-negative integers
which are the indices of the rows (columns) with {\un}s in that column
(row). When observed in this context, if the matrix has the COP on
columns (rows), a reordering of its rows (columns) will result in sets
that have only consecutive integers. In other words, the sets after
applying the COP row (column) permutation are intervals.  In this
form, one can see that this is indeed the problem of finding interval
assignments to the given set system \cite{nsnrs09} with a single
permutation of the universe (set of row or column indices for COP of
columns or rows, respectively) which permutes each set to an
interval. The result in \cite{nsnrs09} characterizes interval
assignments to the sets which can be obtained from a single
permutation of the universe. The cardinality of the interval
assigned to it must be same as the cardinality of the set, and the
intersection cardinality of any two sets must be same as the
intersection cardinality of the corresponding intervals. This is a
necessary and sufficient condition. \tnote{(ICPIA) put reference to
  survey chapter}

Naturally, intervals are paths from a tree with maximum degree
two. Thus the interval assignment problem can be generalized into path
assignment problem from any tree. We refer to this as the {\em tree
  path labeling problem of path hypergraphs}. This is analogous to the
interval labeling problem in literature \cite{kklv10} to interval
hypergraphs. To elaborate, the problem is defined as follows -- given
a hypergraph $\cF$ from universe (hypergraph vertex set) $U$ and a target tree
$T$, does there exist a bijection $\phi$ from $U$ to the vertices of $T$ such
that for each hyperedge when applied to its elements, $\phi$ gives a path on
$T$.  More formally, the problem definition is as defined by \CFTPL.

{\small
  \begin{minipage}[h]{5in}
    \vspace{2mm}
    {\large \CFTPL}\\
    \begin{tabular}[t]{l|l}
      \hline\\
      {\tt Input} & 
      \begin{minipage}[t]{\probdefwidth}
        A hypergraph $\cF$ with vertex set $U$ and a tree $T$.\\
      \end{minipage}\\

      {\tt Question} &
      \begin{minipage}[t]{\probdefwidth}
        Does there exist a set of paths $\cP$ from $T$ and a bijection
        $\cl$~$:$~$\cF \rightarrow \cP$, such that {\FTPL} returns
        {\bf true} on $(\cF, T, \cl)$.
      \end{minipage}\\
    \end{tabular}
  \end{minipage}\\
}


To characterize a feasible TPL, we consider the case where a tree path
labeling is also given as input and we are required to test if the
given labeling is feasible. This is defined by the \FTPL problem.

%\begin{tabular}[t]{c}
{\small
\begin{minipage}[h]{5in}
 % \begin{singlespace}
 \vspace{2mm}
 {\large \FTPL}\\
 \begin{tabular}[t]{l|l}
 \hline\\
    {\tt Input} & 
    \begin{minipage}[t]{\probdefwidth}
      A hypergraph $\cF$ with vertex set $U$, a tree $T$, a set of
      paths $\cP$ from $T$ and a
      bijection $\cl$~$:$~$\cF \rightarrow \cP$.\\
    \end{minipage}\\
    {\tt Question} &
    \begin{minipage}[t]{\probdefwidth}
      Does there exist a bijection $\phi$~$:$~$U \rightarrow V(T)$
      such that $\phi$ when applied on any hyperedge in $\cF$ will
      give
      the path mapped to it by the given tree path labeling $\cl$.\\
      { i.e., $\cl(S) = \{\phi(x) \mid x \in S\}$, for every hyperedge
        $S \in \cF$.}
    \end{minipage}\\
  \end{tabular}
  % \end{singlespace}
\end{minipage}\\
}
% \end{tabular}

Section~\ref{sec:feasible} discusses \FTPL and presents an algorithmic
characterization for a feasible TPL.

With respect to computing a feasible TPL, as suggested by \CFTPL
problem, we were unable to discover an efficient algorithm for
it. Hence we consider two special cases of the same -- namely, \CFTPLINT
and \CFTPLKTREE on special target trees, namely, intervals and \kstars,
respectively.  Section~\ref{sec:spltargettree} discusses these
problems.

{\small
  \begin{minipage}[h]{5in}
    % \begin{singlespace}
    \vspace{2mm}
    {\large \CFTPLINT}\\
    \begin{tabular}[t]{l|l}
      \hline\\
      {\tt Input} & 
      \begin{minipage}[t]{\probdefwidth}
        A hypergraph $\cF$ with vertex set $U$ and a tree $T$ with
        maximum degree 2.\\
      \end{minipage}\\
      {\tt Question} &
      \begin{minipage}[t]{\probdefwidth}
        Does there exist a set of paths $\cP$ from $T$ and a bijection
        $\cl$~$:$~$\cF \rightarrow \cP$, such that {\FTPL} returns
        {\bf true} on $(\cF, T, \cl)$.
      \end{minipage}\\
    \end{tabular}
    % \end{singlespace}
  \end{minipage}\\
}

{\small
  \begin{minipage}[h]{5in}
    % \begin{singlespace}
    \vspace{2mm}
    {\large \CFTPLKTREE}\\
    \begin{tabular}[t]{l|l}
      \hline\\
      {\tt Input} & 
      \begin{minipage}[t]{\probdefwidth}
        A hypergraph $\cF$ with vertex set $U$ such that every
        hyperedge
        $S \in \cF$ is of cardinality at most $k+2$ and a {\kstar} $T$.\\
      \end{minipage}\\
      {\tt Question} &
      \begin{minipage}[t]{\probdefwidth}
        Does there exist a set of paths $\cP$ from $T$ and a bijection
        $\cl$~$:$~$\cF \rightarrow \cP$, such that {\FTPL} returns
        {\bf true} on $(\cF, T, \cl)$.
      \end{minipage}\\
    \end{tabular}
    % \end{singlespace}
  \end{minipage}\\
}

\CFTPLINT is nothing but the consecutive-ones property testing
problem.

An algorithm for \CFTPL on general trees which is less efficient than
polynomial time is presented in
Section~\ref{sec:norestraint}. \tnote[minor]{put some more description
  here. refer WG11.}

\remove[minor]{We focus on the question of generalizing the notion of an
  ICPIA cite-nsnrs09 to characterize feasible path assignments.  We
  show that for a given set system $\cF$, a tree $T$, and an
  assignment of paths from $T$ to the sets, there is a feasible
  bijection between $U$ and $V(T)$ if and only if all intersection
  cardinalities among any three sets (not necessarily distinct) is
  same as the intersection cardinality of the paths assigned to them
  and the input runs a filtering algorithm (described in this paper)
  successfully.  This characterization is proved constructively and it
  gives a natural data structure that stores all the relevant feasible
  bijections between $U$ and $V(T)$.  Further, the filtering algorithm
  is also an efficient algorithm to test if a tree path labeling to
  the set system is feasible.  This generalizes the result in
  cite-nsnrs09.}


\section{Preliminaries to new results}
\label{ch:prelims}

This section states definitions and basic facts necessary in the scope
of this document.

\subsection{Hypergraph Preliminaries}

\begin{definition}[Set systems and hypergraphs] 
  \noindent 
  The set $\F \subseteq (2^{U} \setminus \emptyset)$ is a {\em set
    system} of a universe $U$ with $|U| = n$.  The {\em support} of a
  set system $\F$ denoted by $\supt{\cF}$\tnote[minor]{change all supp
    references to supt command} is the union of all the sets in $\F$;
  $supp(\F) = \bigcup_{S \in \F}S$. For the purposes of this paper, a
  set system is required to ``cover'' the universe; $ supp(\cF) = U$.
 
  \noindent
  A set system $\cF$ can also be visualized as a {\em hypergraph}
  whose vertex set is $supp(\cF)$ and hyperedges are the sets in
  $\cF$. This is a known representation for
  interval systems in literature \cite{bls99,kklv10}.  We extend this
  definition here to path systems. Due to the equivalence of set
  system and hypergraph in the scope of this paper, we drop the
  subscript $_H$ in the notation and refer to both the structures by
  $\cF$.

  \noindent
  The {\em intersection graph}\, $\bI(\cF)$ of a hypergraph
  $\cF$ is a graph such that its vertex set has a bijection to $\cF$
  and there exists an edge between two vertices \iff their
  corresponding hyperedges have a non-empty intersection \cite{mcg04}.

  \noindent 
  Two hypergraphs $\cF'$, $\cF''$ are said to be {\em isomorphic} to
  each other, denoted by $\cF' \cong \cF''$, \iff there exists a
  bijection $\phi: supp(\cF') \rightarrow supp(\cF'')$ such that for
  all sets $A \subseteq supp(\cF')$, $A$ is a hyperedge in $\cF'$ \iff
  $B$ is a hyperedge in $\cF''$ where $B = \{\phi(x) \mid x \in A\}$
  \cite{kklv10}, written as $B=\phi(A)$.  This is called {\em
    hypergraph isomorphism}.  
\end{definition}

\begin{definition}[Path Hypergraph from a Tree]
  The graph $T$ represents a {\em target tree} with same number of
  vertices as elements in $U$; $|V(T)|=|U|=n$.  A {\em path system}\,
  $\cP$ is a set system of paths from $T$; $\cP \subseteq \{P \mid P
  \subseteq V, \text{ } T[P] \text{ is a path} \}$.  This generalizes
  the fact, from the literature \cite{bls99,kklv10}, that intervals
  can be viewed as sub-paths of a path.

  \noindent
  If the intersection graphs of $\cF$ and $\cP$ (a path system) are
  isomorphic, $\bI(\cF) \cong \bI(\cP)$, then the associated bijection
  $\cl: \cF \rightarrow \cP$ due to this isomorphism is called a {\em
    path labeling} of the hypergraph $\cF$.\footnote{Note that there
    are two kinds of isomorphisms here.  One is the isomorphism of
    intersection graphs on $\cF$ and $\cP$, i.e. $\bI(\cF)$ and
    $\bI(\cP)$ respectively. Second is the isomorphism between the
    hypergraphs $\cF$ and $\cP$.} 

% The subset of the path labeling that
%   maps elements of the universe to paths that contain only a leaf the
%   target tree is  known as a {\em leaf labeling}.
%   \ie if the target tree has the set of $k$ leaves $L = \{v_i, i \in
%   [k]\}$, then 
% $\cl': \rightarrow U' \rightarrow L$ is a leaf labeling where $U'
% \subseteq U$, $|U'| = k$.

  

%   $\cL_0 = \{ (x_i,v_i) \mid x_i \in
%   U, v_i \in T, x_i \ne x_j, v_i \ne v_j, i \ne j, i,j \in [k] \}$
%   where $k$ is the number of leaves in $T$
  

  \noindent
  \add{To illustrate further, let $\cg: V(\cF) \rightarrow V(\cP)$ be
    the above mentioned isomorphism where $V(\cF)$ and $V(\cP)$ are
    the vertex sets that represent the hyperedges for each hypergraph
    respectively, $V(\cF) = \{ v_S \mid S \in \cF\}$ and $V(\cP) = \{
    v_P \mid P \in \cP\}$. Then the path labeling $\cl$ is defined as
    follows: $\cl(S_1) = P_1$ iff $\cg (v_{S_1}) = v_{P_1}$.  Just to
    emphasize, for a path labeling $\cl$ of $\cF$ with $\cP$ as the
    path system, $\cF^\cl$ is same as $\cP$.  The path system $\cP$
    may be alternatively denoted in terms of $\cF$ and $\cl$ as
    $\cF^\cl$. In most scenarios in this paper, what is given are the
    pair $(\cF, \cl)$ and the target tree $T$; hence this notation
    will be used more often.}

  \noindent
  If $\cF \cong \cP$ where $\cP$ is a path system, then $\cF$ is
  called a {\em path hypergraph} and $\cP$ is called {\em path
    representation} of $\cF$. If this isomorphism is $\phi: supp(\cF)
  \rightarrow V(T)$, then it is clear that there is an {\em induced
    path labeling} $\cl_\phi: \cF \rightarrow \cP$ to the set system;
  $\cl_\phi(S) = \set{y \mid y = \phi(x), x \in S}$ for all $S \in
  \cF$. Recall that $supp(\cP) = V(T)$.

  \noindent
  \add{A graph} $G$ is a {\em path graph} if it is isomorphic to the
  intersection graph $\bI(\cP)$ of a path system $\cP$.  This
  isomorphism gives a bijection $\cl': V(G) \rightarrow
  \cP$. Moreover, for the purposes of this paper, we require that in a
  path labeling, $supp(\cP) = V(T)$.  If graph $G$ is also isomorphic
  to $\bI(\cF)$ for some hypergraph $\cF$, then clearly there is a
  bijection $\cl: \cF \rightarrow \cP$ such that $\cl(S) = \cl'(v_S)$
  where $v_S$ is the vertex corresponding to set $S$ in $\bI(\cF)$ for
  any $S \in \cF$. This bijection $\cl$ is called the {\em path
    labeling} of the hypergraph $\cF$ and the path system $\cP$ may be
  alternatively denoted as $\cF^\cl$.

  \noindent
  A path labeling $(\cF, \cl)$ is defined to be {\em feasible} if $\cF
  \cong \cF^\cl$ and this hypergraph isomorphism $\phi: supp(\cF)
  \rightarrow supp(\cF^\cl)$ induces a path labeling $\cl_\phi: \cF
  \rightarrow \cF^\cl$ such that $\cl_\phi = \cl$.  \add{In this
    work}, we are given as input $\cF$ and a tree $T$, and the
  question is whether there is a path labeling $\cl$ to a set of paths
  in $T$.  We refer to such a solution path system by $\cF^\cl$.  A
  path labeling $(\cF, \cl)$ is defined to be {\em feasible} if
  % $\cF \cong \cF^\cl$ and this
  there is a hypergraph isomorphism $\phi: supp(\cF) \rightarrow
  supp(\cF^\cl)=V(T)$ induces a path labeling $\cl_\phi: \cF
  \rightarrow \cF^\cl$ such that $\cl_\phi = \cl$.
\end{definition}


\begin{definition}[Overlap Graphs and Marginal Hyperedges]
  An {\em overlap graph}\, $\bO(\cF)$ of a hypergraph $\cF$ is a graph
  such that its vertex set has a bijection to $\cF$ and there exists
  an edge between two of its vertices \iff their corresponding
  hyperedges overlap. Two hyperedges $S$ and $S'$ are said to {\em
    overlap}, denoted by $S \overlap S'$, if they have a non-empty
  intersection and neither is contained in the other; $S \overlap S'
  \text{ \iff } S \cap S' \ne \emptyset, S \nsubseteq S', S'
  \nsubseteq S$. Thus $\bO(\cF)$ is a spanning subgraph of $\bI(\cF)$
  and not necessarily connected. Each connected component of
  $\bO(\cF)$ is called an {\em overlap component}.

  \noindent
  A hyperedge $S \in \cF$ is called {\em marginal} if for all $S'
  \overlap S$, the overlaps $S \cap S'$ form a single inclusion chain
  \cite{kklv10}. Additionally, if $S$ is such that it is contained in
  no other marginal hyperedge in $\cF$, then it is called {\em
    super-marginal}.
\end{definition}

\begin{definition}[$k$-subdivided star] 
  A {\em star} graph is a
  complete bipartite graph $K_{1,p}$ which is clearly a tree and $p$
  is the number of leaves. The vertex with maximum degree is called
  the {\em center} of the star and the edges are called {\em rays} of
  the star. A {\em $k$-subdivided star} is a star with all its rays
  subdivided exactly $k$ times. The definition of a {\em ray of a
    $k$-subdivided star} is extended to the path from the center to a
  leaf. It is clear that all rays are of length $k+2$.
\end{definition}


\temptext{ --- WG11 begin --- }

\remove[dire]{
A path assignment }% $\A$ to $\F$ is defined as a set assignment
% where second universe is the vertex set $V$ of a given tree $T$ and
% every second subset in the ordered pairs is a path in this
% tree. Formally, the definition is as follows. 
% \begin{align*}
%   \A = \{ (S_i,P_i) \mid S_i \in \F, P_i \subseteq V \text{
%     s.t. }T[P_i] \text{ is a path, } i \in I \}
% \end{align*}
% In other words, $P_i$ is the path on the tree $T$ assigned to $S_i$ in
% $\A$. As mentioned before for set systems, the paths cover the whole
% tree, i.e. $\bigcup_{i \in I}P_i = V$ \\

% Generalizing the definition of { feasibility} in cite{nsnrs09} to
% a set assignment, a path assignment $\A$ is defined to be {
%   feasible} if there exists a bijection defined as follows.
% \begin{align}
%   \sigma: U \rightarrow V(T), \text{ such that }\sigma(S_i) = P_i
%   \text{ for all } i \in I, \sigma \text{ is a bijection}
%   \label{eq:stf}
% \end{align}


\begin{definition}[Partially ordered sets]
  
Let $X$ be a partially ordered set with $\preccurlyeq$ being the
partial order on $X$.  $mub(X)$ represents an element in $X$ which is
a maximal upper bound on $X$.  $X_m \in X$ is a maximal upper bound of
$X$ if $\nexists X_q \in X$ such that $X_m
\preccurlyeq X_q$. 
\end{definition}


\noindent
The set $I$ represents the index set $[m]$. If index $i$ is used
without further qualification, it is meant to be $i \in I$. Any
function, if not defined on a domain of sets, when applied on a set is
understood as the function applied to each of its elements. i.e. for
any function $f$ defined with domain $U$, the abuse of notation is as
follows; $f(S)$ is used instead of $\hat f(S)$ where
$\hat f(S) = \{y \mid y = f(x), x \in S\}$. \\

\noindent
When refering to a tree as $T$ it could be a reference to the tree
itself, or the vertices of the tree. This will be clear from the
context.\\

\noindent
Finally, an in-tree is a directed rooted tree in which all edges are
directed toward to the root.

\temptext{ --- WG11 end --- }

\section{Characterization of Feasible Tree Path Labeling}
\label{sec:feasible}

\tnote{\sc refer to Problem Defs here}
In this section we give an algorithmic characterization of the
feasibility of a tree path labeling.  Consider a path labeling $(\cF,
\cl)$ on the given tree $T$. We call $(\cF, \cl)$ an {\em Intersection
  Cardinality Preserving Path Labeling (ICPPL)} if it has the
following properties.

\begin{enumerate}[{(\icpplpr\ }i)]
\item \label{pr:i} $|S| = |\cl(S)|$ for all $S \in \cF$
  % \vspace{\topshrink}
\item \label{pr:ii}$|S_1 \cap S_2| = |\cl(S_1) \cap \cl(S_2)|$ for all
  distinct $S_1, S_2 \in \cF$
  % \vspace{\topshrink}
\item \label{pr:iii}$|S_1 \cap S_2 \cap S_3| = |\cl(S_1) \cap \cl(S_2)
  \cap \cl(S_3)|$ for all distinct $S_1, S_2, S_3 \in \cF$
\end{enumerate}


The following lemma is useful in subsequent arguments.
\begin{lemma}
  \label{lem:setminuscard}
  If $\cl$ is an ICPPL, and $S_1, S_2, S_3 \in \cF$, then $|S_1 \cap
  (S_2 \setminus S_3)| = |\cl(S_1) \cap (\cl(S_2) \setminus
  \cl(S_3))|$.
\end{lemma}
\begin{proof}\thesisspacing%[Proof of Lemma~\ref{lem:setminuscard}]
  Let $P_i = \cl(S_i)$, for all $1 \le i \le 3$.  $|S_1 \cap (S_2
  \setminus S_3)| = |(S_1 \cap S_2) \setminus S_3| = |S_1 \cap S_2| -
  |S_1 \cap S_2 \cap S_3|$. Due to properties (ii) and (iii) of ICPPL,
  $|S_1 \cap S_2| - |S_1 \cap S_2 \cap S_3| = |P_1 \cap P_2| - |P_1
  \cap P_2 \cap P_3| = |(P_1 \cap P_2) \setminus P_3| = |P_1 \cap (P_2
  \setminus P_3)|$. %\qed
\end{proof}


In the remaining part of this section we show that $(\cF, \cl)$ is
feasible if and only if it is an ICPPL and
Algorithm~\ref{al:icppl-find-isomorph} returns a non-empty path
hypergraph isomorphism
function. Algorithm~\ref{al:icppl-find-isomorph} recursively does two
levels of filtering of $(\cF, \cl)$ to make it simpler in terms of set
intersections while retaining the set of isomorphisms, if any, between
$\cF$ and $\cF^\cl$.
% One direction of this claim isclear: that if a path labeling is
% feasible, then all intersection cardinalities are preserved,
% i.e. the path labeling is an ICPPL. Algorithm~\ref{al:filterone}
% \annote{has no premature exit condition hence any input will go
%   through it}{Prove that the filtered sets has ICPPL \iff input PL
%   has ICPPL?}. Algorithm~\ref{al:filtertwo} has an exit condition at
% line~\ref{xempty}. It can be easily verified that $X$ cannot be
% empty if $\cl$ is a feasible path labeling. The reason is that a
% feasible path labeling has an associated bijection between
% $supp(\cF)$ and $V(T)$ \remove{i.e. $supp(\cF^{\cl})$} such that the
% sets map to paths, ``preserving'' the path labeling.  The rest of
% the section is devoted to constructively proving that it is
% sufficient for a path labeling to be an ICPPL and pass the two
% filtering algorithms.  To describe in brief, the constructive
% approaches refine an ICPPL iteratively, such that at the end of each
% iteration we have a ``filtered'' path labeling, and finally we have
% a path labeling that defines a family of bijections from $supp(\cF)$
% to $V(T)$\remove{ i.e. $supp(\cF^{\cl})$}.
First, we present Algorithm~\ref{al:filterone} which describes {\tt
  filter\_1}, and prove its correctness.  This algorithm refines the
path labeling by processing pairs of paths in $\cF^\cl$ that share a
leaf until no two paths in the new path labeling share any leaf.

\begin{algorithm}[h]
  \caption{Refine ICPPL {\tt filter\_1($\cF, \cl, T$)}}
  \label{al:filterone} % {perms}
  \begin{algorithmic}[\lndisplay]
    \STATE $\cF_0 \assign \cF$, $\cl_0(S) \assign \cl(S)$ for all $S \in \cF_0$\\
    \STATE $j \assign 1$\\
    \WHILE {there is $S_1, S_2 \in \cF_{j-1}$ such that
      $\cl_{j-1}(S_1)$ and $\cl_{j-1}(S_2)$ have a common leaf in
      $T$}\label{shareleaf} \STATE $\cF_j \assign (\cF_{j-1} \setminus
    \{S_1, S_2\}) \cup \{S_1 \cap S_2, S_1 \setminus S_2, S_2
    \setminus S_1 \}$ \label{setbreak} \COMMENT {Remove $S_1$, $S_2$
      and add the ``filtered'' sets} \STATE {\bf for} every $S \in
    \cF_{j-1}$ s.t. $S \ne S_1$ and $S \ne
    S_2$ {\bf do} $\cl_j(S) \assign \cl_{j-1}(S)$ {\bf end for}\\

    \STATE $\cl_j(S_1 \cap S_2) \assign \cl_{j-1}(S_1) \cap
    \cl_{j-1}(S_2)$ \COMMENT {Carry forward the path labeling for all
      existing sets other than $S_1$, $S_2$} \STATE $\cl_j(S_1
    \setminus S_2) \assign \cl_{j-1}(S_1) \setminus \cl_{j-1}(S_2)$
    \COMMENT {Define path labeling for new sets} \STATE $\cl_j(S_2
    \setminus S_1) \assign \cl_{j-1}(S_2) \setminus \cl_{j-1}(S_1)$

    \IF{$(\cF_j, \cl_j)$ does not satisfy (\icpplpr~\ref{pr:iii}) of
      ICPPL}
    \label{ln:3waycheck}
    \STATE {\bf exit} \label{ln:exit1} \\
    \ENDIF

    \STATE $j \assign j+1$\\
    \ENDWHILE
    \STATE $\cF' \assign \cF_j$, $\cl' \assign \cl_j$\\
    \RETURN $(\cF', \cl')$
  \end{algorithmic}
\end{algorithm}

\begin{lemma}
  \label{lem:feasible}
  In Algorithm~\ref{al:filterone}, if input $(\cF, \cl)$ is a feasible path
  assignment then at the end of $j$th iteration of the {\bf while}
  loop, $j \ge 0$, $(\cF_j, \cl_j)$ is a feasible path assignment.
\end{lemma}
\begin{proof}\thesisspacing%[Proof of Lemma~\ref{lem:feasible}]
  We will prove this by mathematical induction on the number of
  iterations. The base case $(\cF_0, \cl_0)$ is feasible since it is
  the input itself which is given to be feasible. Assume the lemma is
  true till $j-1$th iteration. \ie every hypergraph isomorphism $\phi:
  supp\left(\cF_{j-1}\right) \rightarrow V\left(T \right)$ that
  defines $(\cF, \cl)$'s feasibility, is such that the induced path
  labeling on $\cF_{j-1}$, say denoted by $\cl_{\phi[{\cF_{j-1}}]}$, is equal to
  $\cl_{j-1}$. We will prove that $\phi$ is also the bijection that
  makes $(\cF_j, \cl_j)$ feasible. Note that $supp(\cF_{j-1}) =
  supp(\cF_{j})$ since the new sets in $\cF_j$ are created from basic
  set operations to the sets in $\cF_{j-1}$ adding or removing no
  elements. For the same reason and $\phi$ being a bijection, it is
  clear that when applying the $\phi$-induced path labeling on
  $\cF_j$, $ \cl_{\phi[{\cF_{j}}]}(S_1 \setminus S_2) =
  \cl_{\phi[{\cF_{j-1}}]}(S_1) \setminus
  \cl_{\phi[{\cF_{j-1}}]}(S_2)$. Now observe that $ \cl_j(S_1
  \setminus S_2) = \cl_{j-1}(S_1) \setminus \cl_{j-1}(S_2) =
  \cl_{\phi[{\cF_{j-1}}]}(S_1) \setminus
  \cl_{\phi[{\cF_{j-1}}]}(S_2)$. Thus the induced path labeling
  $\cl_{\phi[{\cF_{j}}]} = \cl_{j}$.  %\qed
\end{proof}



\begin{lemma}
  \label{lem:invar1} In Algorithm~\ref{al:filterone}, at the end of $j$th
  iteration, $j \ge 0$, of the {\bf while} loop, the following
  invariants are maintained.
  \begin{enumerate}[I {\ }] %\vspace{\topshrink}
  \item $\cl_j(R)$ is a path in $T$, \ \ \ \ \ \ \ \ \ \ \ \ \ \ \ \ \
    \ \ \ \ \ \ \ \ \ \ for all $R \in \cF_j$%\vspace{\topshrink}
  \item $|R| = |\cl_j(R)|$, \ \ \ \ \ \ \ \ \ \ \ \ \ \ \ \ \ \ \ \ \
    \ \ \ \ \ \ \ \ \ \ \ \ \ \ \ for all $R \in
    \cF_j$%\vspace{\topshrink}
  \item $|R \cap R'| = |\cl_j(R) \cap \cl_j(R')|$, \ \ \ \ \ \ \ \ \ \
    \ \ \ \ \ \ \ \ \ \ for all $R, R' \in \cF_j$%\vspace{\topshrink}
  \item $|R \cap R' \cap R''|=|\cl_j(R) \cap \cl_j(R') \cap
    \cl_j(R'')|$, \ \ \ for all $R, R', R'' \in \cF_j$
  \end{enumerate}
\end{lemma}

\begin{proof}\thesisspacing
  Proof is by induction on the number of iterations, $j$. In this
  proof, the term ``new sets'' will refer to the sets added to $\cF_j$
  in $j$th iteration in line~\ref{setbreak} of Algorithm~\ref{al:filterone},
  $S_1 \cap S_2, S_1 \setminus S_2, S_2 \setminus S_1$ and its images
  in $\cl_j$ where $\cl_{j-1}(S_1)$
  and $\cl_{j-1}(S_2)$ intersect and share a leaf.\\
  The invariants are true in the base case $(\cF_0, \cl_0)$, since it
  is the input ICPPL.  Assume the lemma is true till the $j-1$th
  iteration. Let us consider the possible cases for each of the above
  invariants for the $j$th iteration.
  
  \begin{enumerate}[\xbullet]
  \item {\em Invariant} I/II
    \begin{enumerate}[{I/II}a $|$] % \textbullet
    \item {\em $R$ is not a new set.} It is in $\cF_{j-1}$. Thus
      trivially true by induction hypothesis.
    \item {\em $R$ is a new set.} If $R$ is in $\cF_{j}$ and not in
      $\cF_{j-1}$, then it must be one of the new sets added in
      $\cF_j$. In this case, it is clear that for each new set, the
      image under $\cl_j$ is a path since by definition the chosen
      sets $S_1$, $S_2$ are from $\cF_{j-1}$ and due to the while loop
      condition, $\cl_{j-1}(S_1)$, $\cl_{j-1}(S_2)$ have a
      common leaf. Thus invariant I is proven.\\
      Moreover, due to induction hypothesis of invariant III and the
      definition of $l_j$ in terms of $l_{j-1}$, invariant II is
      indeed true in the $j$th iteration for any of the new sets.  If
      $R = S_1 \cap S_2$, $|R| = |S_1 \cap S_2| = |\cl_{j-1}(S_1) \cap
      \cl_{j-1}(S_2)| = |\cl_j(S_1 \cap S_2)| = |\cl_j(R)|$.  If $R =
      S_1 \setminus S_2$, $|R| = |S_1 \setminus S_2| = |S_1| - |S_1
      \cap S_2| = |\cl_{j-1}(S_1)| - |\cl_{j-1}(S_1) \cap
      \cl_{j-1}(S_2)| = |\cl_{j-1}(S_1) \setminus \cl_{j-1}(S_2)| =
      |\cl_j(S_1 \setminus S_2)|
      = |\cl_j(R)|$. Similarly if $R = S_2 \setminus S_1$.\\
    \end{enumerate}
  \item {\em Invariant} III
    \begin{enumerate}[{III}a $|$]
    \item {\em $R$ and $R'$ are not new sets.} It is in
      $\cF_{j-1}$. Thus trivially true by induction hypothesis.
    \item {\em Only one, say $R$, is a new set.} Due to invariant IV
      induction hypothesis, Lemma~\ref{lem:setminuscard} and
      definition of $\cl_j$, it follows that invariant III is true no
      matter which of the new sets $R$ is equal to. If $R = S_1 \cap
      S_2$, $|R \cap R'| = |S_1 \cap S_2 \cap R'| = |\cl_{j-1}(S_1)
      \cap \cl_{j-1}(S_2) \cap \cl_{j-1}(R')| = |\cl_j(S_1 \cap S_2)
      \cap \cl_j(R')| = |\cl_j(R) \cap \cl_j(R')|$.  If $R = S_1
      \setminus S_2$, $|R \cap R'| = |(S_1 \setminus S_2) \cap R'| =
      |(\cl_{j-1}(S_1) \setminus \cl_{j-1}(S_2)) \cap \cl_{j-1}(R')| =
      |\cl_{j}(S_1 \cap S_2) \cap \cl_{j}(R')| = |\cl_{j}(R) \cap
      \cl_{j}(R')|$. Similarly, if $R = S_2 \setminus
      S_1$. Note $R'$ is not a new set.\\

    \item {\em $R$ and $R'$ are new sets.} By definition, the new sets
      and their path images in path label $\cl_j$ are disjoint so $|R
      \cap R'| = |\cl_j(R) \cap \cl_j(R)| = 0$. Thus case proven.
    \end{enumerate}
  \item {\em Invariant} IV
    
    Due to the condition in line~\ref{ln:3waycheck}, this invariant is
    ensured at the end of every iteration.
    % \begin{enumerate} [{Case 3.}1:]
    % \item {\em $R$, $R'$ and $R''$ are not new sets.} Trivially true
    %   by induction hypothesis.
    % \item {\em Only one, say $R$, is a new set.}  If $R = S_1 \cap
    %   S_2$, from Lemma~\ref{lem:fourpaths} and invariant III
    %   hypothesis, this case is proven. Similarly if $R$ is any of
    %   the other new sets, the case is proven by also using Lemma
    %   ~\ref{lem:setminuscard}.
    % \item {\em At least two of $R, R', R''$ are new sets.}  The new
    %   sets are disjoint hence this case is vacuously true.
    % \end{enumerate}
  \end{enumerate} %\qed
  % \vspace{-6mm}
\end{proof}

\begin{lemma}
  \label{lem:noexit1}
  If the input ICPPL $(\cF, \cl)$ to Algorithm~\ref{al:filterone} is
  feasible, then the set of hypergraph isomorphism functions that
  defines $(\cF, \cl)$'s feasibility is the same as the set that
  defines $(\cF_j, \cl_j)$'s feasibility, if any.  Secondly, for any
  iteration $j > 0$ of the {\em \bf while} loop, the {\em \bf exit}
  statement in line~\ref{ln:exit1} will not execute.
\end{lemma}
\begin{proof}\thesisspacing
  Since $(\cF,\cl)$ is feasible, by Lemma~\ref{lem:feasible}
  $(\cF_j,\cl_j)$ for every iteration $j > 0$ is
  feasible.  % Therefore,
  % every hypergraph isomorphism $\phi: supp(\cF) \rightarrow V(T)$
  % that induces $\cl$ on $\cF$ also induces $\cl_{j-1}$ and $\cl_{j}$
  % on $\cF_{j-1}$ and $\cF_{j}$ respectively, i.e.,
  % $\cl_{\phi[\cF_{j-1}]} = \cl_{j-1}$ and $\cl_{\phi[\cF_j]} =
  % \cl_j$. Thus it can be seen that for all $x \in supp(\cF)$, for
  % all $v \in V(T)$ the following hold true.
  Also, every hypergraph isomorphism $\phi: supp(\cF) \rightarrow
  V(T)$ that induces $\cl$ on $\cF$ also induces $\cl_{j}$ on
  $\cF_{j}$, i.e., $\cl_{\phi[\cF_j]} = \cl_j$. Thus it can be seen
  that for all $x \in supp(\cF)$, for all $v \in V(T)$, if $(x,v) \in
  \phi$ then $v \in \cl_{j}(S)$ for all $S \in \cF_{j}$ such that $x
  \in S$.
  % the following hold true.
  % \begin{enumerate}[i. ]
  % \item If $(x,v) \in \phi$ then $v \in \cl_{j-1}(S)$ for all $S \in
  %   \cF_{j-1}$ such that $x \in S$.
  % \item If $(x,v) \in \phi$ then $v \in \cl_{j}(S)$ for all $S \in
  %   \cF_{j}$ such that $x \in S$
  % \end{enumerate}
  In other words, filter 1 outputs a filtered path labeling that
  ``preserves''
  hypergraph isomorphisms of the original path labeling.\\
  Secondly, line~\ref{ln:exit1} will execute \iff the exit condition in
  line~\ref{ln:3waycheck}, i.e. failure of three way intersection
  preservation, becomes true in any iteration of the {\em \bf while}
  loop.  Due to Lemma~\ref{lem:invar1} Invariant IV, the exit
  condition does not occur if the input is a feasible ICPPL.%\qed

  % such that $\phi(x) = v$ where $v$ is the leaf considered in the
  % first iterations of while. Clearly, $\phi$ is a renaming of
  % vertices in hypergraph $\cF$ to those in hypergraph
  % $\cF^\cl$. Thus the following facts can be observed in every
  % iteration of the loop.

  % \begin{enumerate}[\hspace{2mm}i. ] \vspace{\topshrink}
  % \item all intersection cardinalities are preserved in this path
  %   labeling \vspace{\topshrink}
  % \item element $x$ is exclusive in a hyperedge in $\cF$ since $v$
  %   is exclusive in a hyperedge in $\cF^\cl$.
  % \end{enumerate}

  % Thus the exit condition is never rendered true after $x$ and $v$
  % are removed from their respective hyperedges. %\qed

  % \noindent
  % This proof uses mathematical induction on the number of iterations
  % $j$, $j \ge 0$, of the loop that executed without exiting. The
  % base case, $j = 0$ is obviously true since the input is an ICPPL
  % and the exit condition cannot hold true due to ICPPL property
  % (iii).  Assume the algorithm executes till the end of $j-1$th
  % iteration without exiting at line ~\ref{ln:3waycheck}. Consider
  % the $j$th iteration. From Lemma ~\ref{lem:feasible} we know that
  % $(\cF_j, \cl_j)$ and $(\cF_{j-1}, \cl_{j-1})$ are
  % feasible\remove[AS]{and from the proof in lemma lem:invar1 we know
  %   that $(\cF_{j-1}, \cl_{j-1})$ satisfies all the invariants
  %   defined in the lemma}.  Thus there exists a bijection $\phi:
  % supp(\cF) \rightarrow V(T)$ such that the induced path
  %   % labeling on $\cF_{j-1}$ $\cl_{\phi[\cF_{j-1}]} = \cl_{j-1}$.
  % labeling on $\cF_{j}$, $\cl_{\phi[\cF_{j}]}$ and on $\cF_{j-1}$,
  % $\cl_{\phi[\cF_{j-1}]}$ are equal to $\cl_{j}$ and $\cl_{j-1}$
  % respectively.  We need to prove that for any $R, R', R'' \in
  % \cF_{j}$, $|R \cap R' \cap R''| = |\cl_j(R) \cap \cl_j(R') \cap
  % \cl_j(R'')|$.  The following are the possible cases that could
  % arise. From argument above, $|\cl_j(R) \cap \cl_j(R') \cap
  % \cl_j(R'')| = |\cl_{\phi[\cF_{j}]}(R) \cap \cl_{\phi[\cF_{j}]}
  % (R') \cap \cl_{\phi[\cF_{j}]} (R'')|$

  % \begin{enumerate}[a $|$]
  % \item {\em None of the sets are new. $R, R', R'' \in \cF_{j-1}$.}
  %   We know $(\cF_{j-1}, \cl_{j-1})$ is feasible. Thus $|R \cap R'
  %   \cap R''| = |\cl_{j-1}(R) \cap \cl_{j-1}(R') \cap
  %   \cl_{j-1}(R'')| = |\cl_{j}(R) \cap \cl_{j}(R') \cap
  %   \cl_{j}(R'')|$.
  % \item {\em Only one, say $R$, is a new set.}  Let $R = S_1 \cap
  %   S_2$ ($S_1, S_2$ are defined in the proof of lemma
  %   ~\ref{lem:invar1}). Now we have $|R \cap R' \cap R''| = |S_1
  %   \cap S_2 \cap R' \cap R''| = |\cl_{j-1}(S_1) \cap \cl_{j-1}(S_2)
  %   \cap \cl_{j-1}(R') \cap \cl_{j-1}(R'')| = |\cl_{j}(R) \cap
  %   \cl_{j}(R') \cap \cl_{j}(R'')|$. Thus proven. If $R$ is any of
  %   the other new sets, the same claim can be verified using lemma
  %   ~\ref{lem:setminuscard}.
  %     % \item []{\bf Case 3:}
  % \item {\em At least two of $R, R', R''$ are new sets.}  The new
  %   sets are disjoint hence this case is vacuously true.
  % \end{enumerate}
  % %\qed \tnote[E2]{remove the induction proof. just text saying x and
  %   v are exclusive in these sets therefore the intersection
  %   cardinalities don't change thus all invariants are still true}
\end{proof}

As a result of Algorithm~\ref{al:filterone} each leaf $v$ in $T$ is such that
there is exactly one set in $\cF$ with $v$ as a vertex in the path
assigned to it.  In Algorithm~\ref{al:filtertwo} we identify elements in
$supp(\cF)$ whose images are leaves in a hypergraph isomorphism if one
exists.  Let $S \in \cF$ be such that $\cl(S)$ is a path with leaf and
$v \in V(T)$ is the unique leaf incident on it.  We define a new path
labeling $\cl_{new}$ such that $\cl_{new}(\set{x}) = \set{v}$ where
$x$ an arbitrary element from $S \setminus \bigcup_{\hS \ne S}
\hS$. In other words, $x$ is an element present in no other set in
$\cF$ except $S$. This is intuitive since $v$ is present in no other
path image under $\cl$ other than $\cl(S)$.  The element $x$ and leaf
$v$ are then removed from the set $S$ and path $\cl(S)$
respectively. After doing this for all leaves in $T$, all path images
in the new path labeling $\cl_{new}$ except leaf labels (a path that
has only a leaf is called the {\em leaf label} for the corresponding
single element hyperedge or set) are paths from a new pruned tree $T_0
= T \setminus \{v \mid v \text{ is a leaf in }
T\}$. Algorithm~\ref{al:filtertwo} is now presented with details.


%\temptext{ --- WG11 begin --- } %%%%%%%%%%%%%%%%%%%%%%%%%%%%%%%%%

% \begin{lemma}
%   \label{lem:setminuscard}
%   If $\cA$ is an ICPPA, and $(S_1, P_1),(S_2, P_2),(S_3, P_3) \in
%   \cA$, then $|S_1 \cap (S_2 \setminus S_3)| = |P_1 \cap (P_2
%   \setminus P_3)|$.
% \end{lemma}
% \begin{proof}\thesisspacing
%   $|S_1 \cap (S_2 \setminus S_3)| = |(S_1 \cap S_2) \setminus S_3| =
%   |S_1 \cap S_2| - |S_1 \cap S_2 \cap S_3|$. Due to conditions (ii)
%   and (iii) of ICPPA, $|S_1 \cap S_2| - |S_1 \cap S_2 \cap S_3| = |P_1
%   \cap P_2| - |P_1 \cap P_2 \cap P_3| = |(P_1 \cap P_2) \setminus P_3|
%   = |P_1 \cap (P_2 \setminus P_3)|$. Thus lemma is proven. %\qed
% \end{proof}



% \begin{lemma}
%   \label{lem:invar1}
%   In Algorithm \ref{al:filterone}, at the end of $j$th iteration, $j \ge 0$,
%   of the while loop of Algorithm \ref{al:filterone}, the following invariants
%   are maintained.
%   \begin{itemize}
%   \item {\em Invariant I:} $Q$ is a path in $T$ for each $(P,Q) \in
%     \Pi_j$
%   \item {\em Invariant II:} $|P|=|Q|$ for each $(P,Q) \in \Pi_j$
%   \item {\em Invariant III:} For any two $(P,Q), (P',Q') \in \Pi_j$,
%     $|P' \cap P''|=|Q' \cap Q''|$.
%   \item {\em Invariant IV:} For any three, $(P',Q'), (P'',Q''), (P, Q)
%     \in \Pi_j$, $|P' \cap P'' \cap P|=|Q' \cap Q'' \cap Q|$.
%   \end{itemize}
% \end{lemma}
% \begin{proof}\thesisspacing
%   Proof is by induction on the number of iterations, $j$. In the rest
%   of the proof, the term ``new sets'' will refer to the new sets added
%   in $j$th iteration as defined in line \ref{setbreak} of Algorithm
%   \ref{al:filterone}, i.e. the following three assignment pairs for some
%   $(P_1,Q_1), (P_2,Q_2) \in \Pi_{j-1}$ where $Q_1$ and $Q_2$ intersect
%   and share a leaf: $(P_1 \cap P_2, Q_1 \cap Q_2)$, or $(P_1 \setminus
%   P_2, Q_1 \setminus Q_2)$, or $(P_2 \setminus P_1, Q_2
%   \setminus Q_1)$.\\
%   \noindent
%   The base case, $\Pi_0 = \{(S_i,P_i) \mid i \in [m]\}$, is trivially
%   true since it is the input which is an ICPPA.  Assume the lemma is
%   true till the $j-1$ iteration. Consider $j$th
%   iteration:\\
%   \noindent
%   If $(P,Q)$, $(P',Q')$ and $(P'',Q'')$ are in $\Pi_{j}$ and
%   $\Pi_{j-1}$, all the invariants are
%   clearly true because they are from $j-1$ iteration.\\
%   If $(P,Q)$ is in $\Pi_{j}$ and not in $\Pi_{j-1}$, then it must be
%   one of the new sets added in $\Pi_j$. Since $(P_1,Q_1)$ and
%   $(P_2,Q_2)$ are from $\Pi_{j-1}$ and $Q_1,Q_2$ intersect and have a
%   common leaf, it can be verified that the
%   new sets are also paths. \\
%   By hypothesis for invariant III, invariant II also holds for $(P,Q)$
%   no matter which new set in $\Pi_j$ it
%   is.\\
%   To prove invariant III, if $(P,Q)$ and $(P',Q')$ are not in
%   $\Pi_{j-1}$, then they are both new sets and invariant III holds
%   trivially (new sets are disjoint). Next consider $(P,Q), (P',Q') \in
%   \Pi_j$ with only one of them, say $(P',Q')$, in $\Pi_{j-1}$. Then
%   $(P,Q)$ is one of the new sets added in line \ref{setbreak}. It is
%   easy to see that if $(P,Q)$ is $(P_1 \cap P_2, Q_1 \cap Q_2)$, then
%   due to invariant IV in hypothesis, invariant III becomes true in
%   this iteration. Similarly, using lemma \ref{lem:setminuscard}
%   invariant III is proven if $(P, Q)$ is $(P_1 \setminus P_2, Q_1
%   \setminus Q_2)$, or $(P_2 \setminus P_1, Q_2
%   \setminus Q_1)$.\\
%   To prove invariant IV, consider three assignments
%   $(P,Q),(P',Q'),(P'',Q'')$. If at least two of these pairs are in not
%   $\Pi_{j-1}$, then they are any two of the new sets. Note that these
%   new sets are disjoint and hence if $(P',Q'), (P'',Q'')$ are any of
%   these sets, $|P \cap P' \cap P''|=|Q \cap Q' \cap Q''|=0$ and
%   invariant IV is true. Now we consider the case if at most one of
%   $(P,Q),(P',Q'),(P'',Q'')$ is not in $\Pi_{j-1}$. If none of them are
%   not in $\Pi_{j-1}$ (i.e. all of them are in $\Pi_{j-1}$), invariant
%   IV is clearly true. Consider the case where exactly one of them is
%   not in $\Pi_{j-1}$. w.l.o.g let that be $(P,Q)$ and it could be any
%   one of the new sets. If $(P,Q)$ is $(P_1 \cap P_2, Q_1 \cap Q_2)$,
%   from lemma \ref{lem:fourpaths} and invariant III hypothesis,
%   invariant IV is proven. Similarly if $(P,Q)$ is any of the other new
%   sets, invariant IV is proven by also using lemma
%   \ref{lem:setminuscard}. %\qed

% \end{proof} 
%\temptext{ --- WG11 end --- }%%%%%%%%%%%%%%%%%%%%%%%%%%%%%%%%%


\begin{algorithm}[h]
  \caption{Leaf labeling from an ICPPL {\tt filter\_2($\cF, \cl, T$)}}
  \label{al:filtertwo} % {leafasgn}
  \begin{algorithmic}[\lndisplay]
    \STATE $\cF_0 \assign \cF$, $\cl_0(S) \assign \cl(S)$ for all $S
    \in \cF_0$ \COMMENT {Path images are such that no two path images
      share a leaf.}
    \STATE $j \assign 1$\\
    \WHILE {there is a leaf $v$ in $T$ and a unique $S_1 \in
      \cF_{j-1}$ such that $v \in \cl_{j-1}(S_1)$ }\label{uniqueleaf}
    \STATE $\cF_j \assign \cF_{j-1} \setminus \{S_1\}$\\
    \STATE for all $S \in \cF_{j-1}$ such that $S \ne S_1$ set
    $\cl_j(S) \assign
    \cl_{j-1}(S)$\\
    \STATE $X \assign S_1 \setminus \bigcup_{S \in \cF_{j-1}, S \ne S_1}S$\\
    \IF{$X$ is empty} \label{xempty} \STATE {\bf
      exit} \label{ln:exit2} \ENDIF
    \STATE $x \assign $ arbitrary element from $X$\\
    \STATE $\cF_j \assign \cF_j \cup \{\{x\}, S_1 \setminus \{x\}\} $\\
    \STATE $\cl_j(\{x\}) \assign \{v\}$\\
    \STATE $\cl_j(S_1 \setminus \{x\}) \assign \cl_{j-1}(S_1) \setminus \{v\}$\\
    \STATE $j \assign j+1$\\
    \ENDWHILE
    \STATE $\cF' \assign \cF_j$, $\cl' \assign \cl_j$\\
    \RETURN $(\cF', \cl')$
  \end{algorithmic}
\end{algorithm}


Suppose the input ICPPL $(\cF, \cl)$ is feasible, yet set $X$ in
Algorithm~\ref{al:filtertwo} is empty in some iteration of the {\bf while}
loop. This will abort our procedure of finding the hypergraph
isomorphism. The following lemma shows that this will not happen.

%\temptext{ --- WG11 begin --- }%%%%%%%%%%%%%%%%%%%%%%%%%%%%%%%%%
% \begin{lemma}
%   \label{lem:invar3}
%   In Algorithm \ref{al:filtertwo},for all $j \geq 0$, at the end of the
%   $j$th iteration the four invariants given in lemma \ref{lem:invar1}
%   are valid.
%   % Moreover, $X$ as defined in the algorithm is non-empty if this is
%   % an ICPPA.
% \end{lemma}
% \begin{proof}\thesisspacing
%   First we see that $X = S_{i_1} \setminus \bigcup_{i \ne i_1, i \in
%     I}S_i$ is non empty in every iteration for an ICPPA. Suppose $X$
%   is empty. We know that $v \in P_{i_1} \setminus \bigcup_{i \ne i_1,
%     i \in I}P_i$ since $v$ is in the unique path $P_{i_1}$. Since this
%   is an ICPPA $|S_{i_1}| = |P_{i_1}|$. For any $x \in S_{i_1}$ it is
%   contained in at least two sets due to our assumption. Let $S_{i_2}$
%   be a second set that contains $x$. We know $v \notin
%   P_{i_2}$. Therefore there cannot exist a permutation that maps
%   elements of $S_{i_2}$ to $P_{i_2}$. This contradicts our assumption
%   that this is an ICPPA. Thus $X$ cannot be empty.


%   We use mathematical induction on the number of iterations for this
%   proof. The term ``new sets'' will refer to the sets added in $\Pi_j$
%   in the $j$th iteration, i.e. $(P' \setminus \{x\},Q' \setminus
%   \{v\})$ and $(\{x\},\{v\})$ for some $(P',Q')$ in $\Pi_{j-1}$ such
%   that $v$ is a leaf and $Q'$ is the unique path
%   incident on it.\\
%   For $\Pi_0$ all invariants hold because it is output from algorithm
%   \ref{al:filterone} which is an ICPPA. Hence base case is proved.  Assume
%   the lemma holds for $\Pi_{j-1}$. Consider $\Pi_j$ and any $(P,Q) \in
%   \Pi_j$. If $(P,Q) $ is in $ \Pi_j$ and $\Pi_{j-1}$ invariants I and
%   II are true because of induction assumption. If it is only in
%   $\Pi_j$, then it is $\{(P' \setminus \{x\}),(Q' \setminus \{v\})$ or
%   $(\{x\},\{v\})$ for some $(P',Q')$ in $\Pi_{j-1}$. By definition,
%   $x$ is an element in $P'$ (as defined in the algorithm) and $v$ is a
%   leaf in $Q'$. If $(P,Q)$ is $\{(P' \setminus \{x\}),(Q' \setminus
%   \{v\})$, $Q$ is a path since only a leaf is removed from path
%   $Q'$. We know $|P'| = |Q'|$, therefore $|P' \setminus \{x\}| = |Q'
%   \setminus \{v\}|$. Hence in this case invariants I and II are
%   obvious. It is easy to see these invariants hold if $(P,Q)$ is
%   $(\{x\},\{v\})$.


%   For invariant III consider $(P_1,Q_1),(P_2,Q_2)$ in $\Pi_j$. If both
%   of them are also in $\Pi_{j-1}$, claim is proved. If one of them is
%   not in $\Pi_{j-1}$ then it has to be $\{(P' \setminus \{x\}),(Q'
%   \setminus \{v\})$ or $(\{x\},\{v\})$ for some $(P',Q')$ in
%   $\Pi_{j-1}$. Since by definition, $Q'$ is the only path with $v$ and
%   $P'$ the only set with $x$ in the previous iteration, $|P_1 \cap (P'
%   \setminus \{x\})| = |P_1 \cap P'|$ and $|Q_1 \cap (Q' \setminus
%   \{v\})| = |Q_1 \cap Q'|$ and $|P_1 \cap \{x\}| = 0, Q_1 \cap \{v\} =
%   0$. Thus invariant III is also proven.

%   \noindent
%   To prove invariant IV, consider $(P_1,Q_1),(P_2,Q_2), (P_3,Q_3)$ in
%   $\Pi_j$. If exactly one of them, say $P_3 \notin \Pi_{j-1}$, it is
%   one of the new sets. By the same argument used to prove invariant
%   III, $|P_1 \cap P_2 \cap (P' \setminus \{x\})| = |P_1 \cap P_2 \cap
%   P'|$ and $|Q_1 \cap Q_2 \cap (Q' \setminus \{x\})| = |Q_1 \cap Q_2
%   \cap Q'|$. Since $P_1, P_2, P'$ are all in $\Pi_{j-1}$, by induction
%   hypothesis $|P_1 \cap P_2 \cap P'| = |Q_1 \cap Q_2 \cap Q'|$. Also
%   $|P_1 \cap P_2 \cap \{x\}| = 0, Q_1 \cap Q_2 \cap \{v\} = 0$.  If
%   two or more of them are not in $\Pi_{j-1}$, then it can be verified
%   that $|P_1 \cap P_2 \cap P_3| = |Q_1 \cap Q_2 \cap Q_3|$ since the
%   new sets in $\Pi_j$ are either disjoint or as follows: assuming
%   $P_1, P_2 \notin \Pi_{j-1}$ and new sets are derived from $(P', Q'),
%   (P'', Q'') \in \Pi_{j-1}$ with $x_1, x_2$ exclusively in $P_1, P_2$,
%   $(\{x_1\},\{v_1\}), (\{x_2\},\{v_2\}) \in \Pi_j $ thus $v_1, v_2$
%   are exclusively in $Q_1, Q_2$ resp. it follows that $|P_1 \cap P_2
%   \cap P_3| = |(P' \setminus \{x_1\}) \cap (P'' \setminus \{x_2\})
%   \cap P_3| = |P' \cap P'' \cap P_3| = |Q' \cap Q'' \cap Q_3| = |(Q'
%   \setminus \{v_1\} \cap Q'' \setminus \{v_2\} \cap Q_3| = |Q_1 \cap
%   Q_2 \cap Q_3|$. Thus invariant IV is also proven.  %\qed
% \end{proof}

% Using algorithms \ref{al:filterone} and \ref{al:filtertwo} we prove the following
% theorem.

% \begin{theorem}
%   \label{th:perm}
%   If $\cA$ is an ICPPA, then there exists a bijection $\sigma : U
%   \rightarrow V(T)$ such that $\sigma(S_i) = P_i$ for all $i \in I$
% \end{theorem}
% \begin{proof}\thesisspacing
%   This is a contructive proof. First, the given ICPPA $\cA$ and tree
%   $T$ are given as input to Algorithm \ref{al:filterone}. This yields a
%   ``filtered'' ICPPA as the output which is input to Algorithm
%   \ref{al:filtertwo}.  It can be observed that the output of Algorithm
%   \ref{al:filtertwo} is a set of interval assignments to sets and
%   one-to-one assignment of elements of $U$ to each leaf of $T$. To be
%   precise, it would be of the form $\cB_0 = \cA_0 \cup \cL_0$. The
%   leaf assignments are defined in $\cL_0 = \{ (x_i,v_i) \mid x_i \in
%   U, v_i \in T, x_i \ne x_j, v_i \ne v_j, i \ne j, i,j \in [k] \}$
%   where $k$ is the number of leaves in $T$. The path assignments are
%   defined in $\cA_0 \subseteq \{(S_i',P_i') \mid S_i' \subseteq U_0,
%   P_i' \text{ is a path from } T_0\}$ where $T_0$ is the tree obtained
%   by removing all the leaves in $T$ and $U_0 = U \setminus \{ x \mid x
%   \text{ is assigned to a leaf in }\cL_0 \}$. Now we have a subproblem
%   of finding the permutation for the path assignment $\cA_0$ which has
%   paths from tree $T_0$ and sets from universe $U_0$. Now we repeat
%   the procedure and the path assignment $\cA_0$ and tree $T_0$ is
%   given as input to Algorithm \ref{al:filterone}. The output of this
%   algorithm is given to Algorithm \ref{al:filtertwo} to get a new union of
%   path and leaf assignments $\cB_1 = \cA_1 \cup \cL_1$ defined similar
%   to $\cB_0, \cL_0, \cA_0$. In general, the two algorithms are run on
%   path assignment $\cA_{i-1}$ with paths from tree $T_{i-1}$ to get a
%   new subproblem with path assignment $\cA_i$ and tree $T_{i}$. $T_i$
%   is the subtree of $T_{i-1}$ obtained by removing all its
%   leaves. More importantly, it gives leaf assignments $\cL_{i}$ to the
%   leaves in tree $T_{i-1}$. This is continued until we get a
%   subproblem with path assignment $\cA_{d-1}$ and tree $T_{d-1}$ for
%   some $d \le n$ which is just a path. From the last lemma we know
%   that $\cA_{d-1}$ is an ICPPA. Another observation is that an ICPPA
%   with all its tree paths being intervals (subpaths from a path) is
%   nothing but an ICPIA\cite{nsnrs09}.  Let $\cA_{d-1}$ be equal to
%   $\{(S_i'',P_i'') \mid S_i'' \subseteq U_{d-1}, P_i'' \text{ is a
%     path from } T_{d-1} \}$. It is true that the paths $P_i''$s may
%   not be precisely an interval in the sense of consecutive integers
%   because they are some nodes from a tree. However, it is easy to see
%   that the nodes of $T_{d-1}$ can be ordered from left to right and
%   ranked to get intervals $I_i$ for every path $P_i''$ as
%   follows. $I_i = \{[l,r] \mid l = \text{ the lowest rank of the nodes
%     in }P_i'', r = l+|P_i''|-1 \}$. Let asssignment $\cA_d$ be with
%   the renamed paths. $\cA_d = \{ (S_i'', I_i) \mid (S_i'', P_i'') \in
%   \cA_{d-1} \}$. What has been effectively done is renaming the nodes
%   in $T_{d-1}$ to get a tree $T_d$.  The ICPIA $\cA_d$ is now in the
%   format that the ICPIA algorithm requires which gives us the
%   permutation $\sigma' : U_{d-1} \rightarrow T_{d-1}$

% \noindent
% $\sigma'$ along with all the leaf assignments $\cL_i$ gives us the
% permutation for the original path assignment $\cA$.  More precisely,
% the permutation for tree path assignment $\cA$ is defined as
% follows. $\sigma: U \rightarrow T$ such that the following is
% maintained.
% \begin{align*}
%   \sigma(x) &= \sigma'(x),   \text{ if } x \in U_{d-1} \\
%   &= \cL_i(x), \text{ where $x$ is assigned to a leaf in a subproblem
%     $\cA_{i-1}, T_{i-1}$}
% \end{align*}

% \noindent
% To summarize, run algorithm \ref{al:filterone} and \ref{al:filtertwo} on
% $T$. After the leaves have been assigned to specific elements from
% $U$, remove all leaves from $T$ to get new tree $T_0$. The leaf
% assignments are in $\cL_0$. Since only leaves were removed $T_0$ is
% indeed a tree. Repeat the algorithms on $T_0$ to get leaf assignments
% $\cL_{1}$. Remove the leaves in $T_0$ to get $T_1$ and so on until the
% pruned tree $T_d$ is a single path. Now run ICPIA algorithm on $T_d$
% to get permutation $\sigma'$. The relation $\cL_0 \cup \cL_1 \cup
% .. \cup \cL_{d} \cup \sigma'$ gives the bijection required in the
% original problem.%\qed
% \end{proof}
%\temptext{ --- WG11 end --- }%%%%%%%%%%%%%%%%%%%%%%%%%%%%%%%%%


\begin{lemma}
  \label{lem:xnotempty}
  If the input ICPPL $(\cF, \cl)$ to Algorithm~\ref{al:filtertwo} is
  feasible, then for all iterations $j > 0$ of the {\em \bf while}
  loop, the {\em \bf exit} statement in line~\ref{ln:exit2} does not
  execute.
\end{lemma}
\begin{proof}\thesisspacing
  Assume $X$ is empty for some iteration $j > 0$. We know that $v$ is
  an element of $\cl_{j-1}(S_1)$. Since it is uniquely present in
  $\cl_{j-1}(S_1)$, it is clear that $v \in \cl_{j-1}(S_1) \setminus
  \bigcup_{(S \in \cF_{j-1}) \wedge (S \ne S_1)}\cl_{j-1}(S)$.  Note
  that for any $x \in S_1$ it is contained in at least two sets due to
  our assumption about cardinality of $X$. Let $S_2 \in \cF_{j-1}$ be
  another set that contains $x$. From the above argument, we know $v
  \notin \cl_{j-1}(S_2)$. Therefore there cannot exist a hypergraph
  isomorphism bijection that maps elements in $S_2$ to those in
  $\cl_{j-1}(S_2)$. This contradicts our assumption that the input is
  feasible. Thus $X$ cannot be empty if input is ICPPL and feasible.
  %\qed
\end{proof}

\begin{lemma}
  \label{lem:invar3}
  In Algorithm~\ref{al:filtertwo}, for all $j > 0$, at the end of the
  $j$th iteration of the {\bf while} loop the four invariants given in
  Lemma~\ref{lem:invar1} hold.
\end{lemma}
\begin{proof}\thesisspacing
  By Lemma~\ref{lem:xnotempty} we know that set $X$ will not be empty
  in any iteration of the {\em \bf while} loop if input ICPPL $(\cF,
  \cl)$ is feasible and $\cl_j$ is always computed for all $j >
  0$. Also note that removing a leaf from any path keeps the new path
  connected. Thus invariant I is obviously true. In every iteration $j
  > 0$, we remove exactly one element $x$ from one set $S$ in $\cF$
  and exactly one vertex $v$ which is a leaf from one path
  $\cl_{j-1}(S)$ in $T$. This is because as seen in
  Lemma~\ref{lem:xnotempty}, $x$ is exclusive to $S$ and $v$ is
  exclusive to $\cl_{j-1}(S)$. Due to this fact, it is clear that the
  intersection cardinality equations do not change, i.e., invariants
  II, III, IV remain true. On the other hand, if the input ICPPL is
  not feasible the invariants are vacuously true. %\qed
\end{proof}


We have seen two filtering algorithms above, namely,
Algorithm~\ref{al:filterone} for {\tt filter\_1} and
Algorithm~\ref{al:filtertwo} for {\tt filter\_2} which when executed
in that order result in a new ICPPL on the same universe $U$ and
target tree $T$. We also proved that if the input is indeed feasible,
these algorithms executed in sequence give an ICPPL that preserves \change{the
  same set}{a subset} of hypergraph isomorphisms as that of the input
ICPPL. Now we present the algorithmic characterization of a feasible
tree path labeling by way of Algorithm~\ref{al:icppl-find-isomorph}.

\temptext{Prove that the same set of hypergraph isomorphisms are in
  the filtered ICPPL? -- NOPE. we are only interested in testing
  feasibility. not necessarily giving ALL bijections.}
\begin{lemma}{\temptext{PROOF TBD.}}
  Let $(\cF_2, \cl_2)$ be the output of {\tt filter\_2} and the ICPPL
  input to {\tt filter\_1} preceeding its call is $(\cF, \cl)$ with
  target tree $T$. If a bijection $\phi:\supt{\cF_2} \rightarrow V(T)$
  is a hypergraph isomorphism of $(\cF_2, \cl_2)$, then it is a
  hypergraph isomorphism of $(\cF, \cl)$.
\end{lemma}


Algorithm~\ref{al:icppl-find-isomorph} computes a hypergraph
isomorphism $\phi$ recursively using Algorithm~\ref{al:filterone} and
Algorithm~\ref{al:filtertwo} and pruning the leaves of the input
tree. In brief, it is done as follows. If the input is feasible, the
output of {\tt filter\_1}, say $(\cF_1, \cl_1)$, gives a new ICPPL
which preserves all the original hypergraph isomorphisms.  This ICPPL
$(\cF_1, \cl_1)$ is then given to {\tt filter\_2} which gives the
second filtered ICPPL, say $(\cF_2, \cl_2)$ such that a feasible leaf
labeling is in singleton elements of $\cF_2$. These leaf labels are
the elements in $supp(\cF)$ that map to leaves in $T$ in at least one
hypergraph isomorphism of the input ICPPL $\cl$. It must be noted that
at this point, the algorithm ``chooses'' one (or more - there could be
more than one bijections that have the same leaf labels) hypergraph
isomorphism $\phi'$, from the set of all isomorphisms resulting from the
feasibility of the original input $(\cF, \cl)$. However, if there
exists one (\ie input is feasible), it will be found -- thus
performing the test required in \FTPL.

Since their pre-images have been found, all leaves in $T$ are then
pruned away. The leaf labels are removed from the path labeling
$\cl_2$ and the corresponding elements are removed from the
corresponding sets in $\cF_2$. It is now clear that we have a
subproblem with a new hypergraph $\cF'$, new tree path labeling $\cl'$
and target tree $T'$.  The tree pruning algorithm is recursively
called on $\cF', \cl', T'$. The recursive call returns the bijection
$\phi''$ for the rest of the elements in $supp(\cF)$ which along with
the leaf labels $\phi'$ computed earlier gives us a hypergraph
isomorphism $\phi$ for the input $\cF,\cl,T$.  The following lemma formalizes the
characterization of feasible path labeling.

\begin{algorithm}[h]
  \caption{{\tt get-hypergraph-isomorphism ($\cF, \cl, T$)}}
  \label{al:icppl-find-isomorph}
  \begin{algorithmic}[\lndisplay]

    \IF{$T$ is empty}
    \RETURN $\emptyset$\\
    \ENDIF
    \STATE $L \assign \{v \mid v \text{ is a leaf in }      T\}$\\
    \STATE $(\cF_1, \cl_1) \assign$ {\tt filter\_1($\cF, \cl,
      T$)}\\
    \STATE $(\cF_2, \cl_2) \assign$ {\tt filter\_2($\cF_1,
      \cl_1, T$)}\\

    \STATE $(\cF', \cl') \assign (\cF_2, \cl_2)$\\
    \STATE $\phi' \leftarrow \emptyset$

    \FOR {every $v \in L$} \STATE $\phi'(x) \assign v$ where $x \in
    \cl_2^{-1}(\{v\})$ \COMMENT {Copy the leaf labels to a one to one
      function $\phi': supp(\cF) \rightarrow L$
    }\\
    \STATE Remove $\{x\}$ and $\{v\}$ from $\cF'$, $\cl'$  appropriately\\
    \ENDFOR

    \STATE $T' \assign T \setminus L$

    \STATE $\phi'' \assign$ {\tt get-hypergraph-isomorphism($\cF',
      \cl', T'$)}
    \STATE $\phi \assign \phi'' \cup \phi'$ \\
    \RETURN $\phi$
  \end{algorithmic}
\end{algorithm}

\begin{lemma}
  \label{lem:hyperiso} %{lem:perm}
  If $(\cF, \cl)$ is an ICPPL from a tree $T$ and
  Algorithm~\ref{al:icppl-find-isomorph}, on input $\cF, \cl, T$
  returns a non-empty function, then there exists a hypergraph
  isomorphism $\phi : supp(\cF) \rightarrow V(T)$ such that the
  $\phi$-induced tree path labeling is equal to $\cl$ or $\cl_\phi =
  \cl$.
\end{lemma}
\begin{proof}\thesisspacing
  It is clear that in the end of every recursive call to
  Algorithm~\ref{al:icppl-find-isomorph}, the function $\phi'$ is
  one-to-one involving all the leaves in the input target tree $T$ (of
  the current recursive call). Moreover, by Lemma~\ref{lem:noexit1}
  and Lemma~\ref{lem:xnotempty} it is consistent with the input tree
  path labeling $\cl$ (of the current recursive call). The tree
  pruning is done by only removing leaves in each call to the function
  and is done till the tree becomes empty. Thus the returned function
  $\phi: supp(\cF) \rightarrow V(T)$ is a union of mutually exclusive
  one-to-one functions exhausting all vertices of the tree. In other
  words, it is a bijection from $supp(\cF)$ to $V(T)$ inducing the
  given path labeling $\cl$ and thus a hypergraph isomorphism. %\qed
\end{proof}

\begin{theorem}
  \label{th:charac}
  A path labeling $(\cF, \cl)$ on tree $T$ is feasible \iff it is an
  ICPPL and Algorithm~\ref{al:icppl-find-isomorph} with $(\cF, \cl,
  T)$ as input returns a non-empty function.
\end{theorem}
\begin{proof}\thesisspacing
  From Lemma~\ref{lem:hyperiso}, we know that if $(\cF, \cl)$ is an
  ICPPL and if Algorithm~\ref{al:icppl-find-isomorph} with $(\cF, \cl,
  T)$ as input returns a non-empty function, then $(\cF, \cl)$ is
  feasible.  Now consider the case where $(\cF, \cl)$ is
  feasible, \ie there exists a hypergraph isomorphism $\phi$ such
  that $\cl_\phi = \cl$. Lemma~\ref{lem:noexit1} and
  Lemma~\ref{lem:xnotempty} show us that {\tt filter\_1} and {\tt
    filter\_2} do not exit if input is feasible. Thus
  Algorithm~\ref{al:icppl-find-isomorph} returns a non-empty
  function.%\qed
\end{proof}


\section{ Computing feasible TPL with special target trees}
\label{sec:spltargettree}

\tnote{\sc refer to Problem Defs here}
Section~\ref{sec:feasible} described properties that a TPL must have
for it to be feasible. The next problem of interest is to test if a
given hypergraph is a path hypergraph with respect to a given target
tree\footnote{A larger problem would be to find if a given hypergraph
  is a path graph on any tree. This problem is not addressed in this
  thesis.}. In other words, the problem is to find out if a feasible
tree path labeling exists from a given target tree for a given
hypergraph. In this section we will see two special cases of this
problem where the target tree is from a particular family of
trees. The first one, where the tree is a path, is a shown to be
equivalent to the well studied problem of consecutive-ones in
Section~\ref{sec:icpplicpia}. The second one, where the tree is a
$k$-subdivided tree\footnote{See Section~\ref{ch:prelims} for the
  formal definition.}, has been solved using a polynomial time
algorithm. The latter problem enforces some conditions on the
hypergraph too which will be seen in Section~\ref{sec:ksubdivstar}.

\subsection{Target tree is a Path}
\label{sec:icpplicpia}

Let us consider a special case of ICPPL with the following properties.
The target tree $T$ is a path.  Hence, all path labels are can be viewed as
intervals assigned to the sets in $\cF$.  It is shown, in
\cite{nsnrs09}, that the filtering algorithms outlined above need only
preserve pairwise intersection cardinalities, and higher level
intersection cardinalities are preserved by the Helly Property of
intervals.  Consequently, the filter algorithms do not need to ever
evaluate the additional check to {\em \bf exit}.
%\begin{enumerate}
%\item Given tree $T$ is a path. Hence, all path labels are interval labels.
%\item Only pairwise intersection cardinality
%  preservation is sufficient. i.e. property (iii) in ICPPL is not enforced.
%\item The filter algorithms do not have {\em \bf exit} statements.
%\end{enumerate}
%This is called an Intersection Cardinality Preservation Interval
%Assignment (ICPIA) \cite{nsnrs09}. 
This structure and its algorithm is used in the next section for
finding tree path labeling from a $k$-subdivided star due to this
graph's close relationship with intervals.

\tnote[dire]{ELABORATE}


\subsection{Target tree is a $k$-subdivided Star}
\label{sec:ksubdivstar}

\begin{figure}[t] %[htbp] 
  \centering
  \begin{tabular}{lr}
    (a) \includegraphics{cop-tree_thesis.1}&
    (b) \includegraphics{cop-tree_thesis.2}
  \end{tabular}
  \caption{\figtabsize (a) $8$-subdivided star with 7 rays (b)
    3-subdivided star with 3 rays}
  \label{fig:kstar}
\end{figure}

In this section we consider the problem of assigning paths from a
$k$-subdivided star $T$ to a given set system $\cF$.  We consider
$\cF$ for which the overlap graph $\bO(\cF)$ is connected.  The
overlap graph is well-known from the work of
\cite{kklv10,nsnrs09,wlh02}.  We use the notation in
\cite{kklv10}. Recall from Section~\ref{ch:prelims} that hyperedges
$S$ and $S'$ are said to overlap, denoted by $S \overlap S'$, if $S$
and $S'$ have a non-empty intersection but neither of them is
contained in the other. The overlap graph $\bO(\cF)$ is a graph in
which the vertices correspond to the sets in $\cF$, and the vertices
corresponding to the hyperedges $S$ and $S'$ are adjacent if and only
if they overlap.  Note that the intersection graph of $\cF$,
$\bI(\cF)$ is different from $\bO(\cF)$ and $\bO(\cF) \subseteq
\bI(\cF)$.  A connected component of $\bO(\cF)$ is called an overlap
component of $\cF$.  An interesting property of the overlap components
is that any two distinct overlap components, say $\cO_1$ and $\cO_2$,
are such that any two sets $S_1 \in \cO_1$ and $S_2 \in \cO_2$ are
disjoint, or, w.l.o.g, all the sets in $\cO_1$ are contained within
one set in $\cO_2$.  This containment relation naturally determines a
decomposition of the overlap components into rooted containment trees.
We consider the case when there is only one rooted containment tree,
and we first present our algorithm when $\bO(\cF)$ is connected.  It
is easy to see that once the path labeling to the overlap component in
the root of the containment tree is achieved, the path labeling to the
other overlap components in the rooted containment tree is essentially
finding a path labeling when the target tree is a path: each target
path is a path that is allocated to sets in the root overlap
component.  Therefore, for the rest of this section, $\bO(\cF)$ is a
connected graph. We also assume that all hyperedges are of cardinality
at most $k+2$.

Recall from Section~\ref{ch:prelims} that a $k$-subdivided star is a
star with each edge subdivided $k$ times. Therefore, a $k$-subdivided
star has a central vertex which we call the {\em root}, and each root
to leaf path is called a {\em ray}. First, we observe that by removing
the root $r$ from $T$, we get a collection of $p$ vertex disjoint
paths of length $k+1$, $p$ being the number of leaves in $T$.  We
denote the rays by $R_1, \ldots, R_p$, and the number of vertices in
$R_i$, $i \in [p]$ is $k+2$.  Let $\seq{v_{i1},\ldots,v_{i(k+2)}=r}$
denote the sequence of vertices in $R_i$, where $v_{i1}$ is the
leaf. Note that $r$ is a common vertex to all $R_i$.

In this section the given hypergraph, the $k$-subdivided star and the
root of the star are denoted by $\cO$, $T$ and vertex $r$,
respectively.
% For each hyperedge $X \in \cO$, we will maintain a 2-tuple of
% non-negative numbers $\seq{p_1(X), p_2(X)}$.  The numbers satisfy
% the property that $p_1(X) + p_2(X) \leq |X|$, and at the end of path
% labeling, for each $X$, $p_1(X) + p_2(X) = |X|$.  This signifies the
% algorithm tracking the lengths of subpaths of the path assigned to
% $X$ from at most two rays. We also maintain another parameter called
% the {\em residue} of $X$ denoted by $s(X)=|X| - p_1(X)$. This
% signifies the residue path length that must be assigned to $X$ which
% must be from another ray. For instance, if $X$ is labeled a path
% from only one ray, then $p_1(X) = |X|$, $p_2(X) = 0$ and $s(X) = 0$.

% 
% We iteratively consider each ray from which paths will be assigned
% to hyperedges. At the beginning of each iteration hyperedges of
% $\cO$ are classifed into the following disjoint sets.
% \begin{enumerate}
% \item [$\cL_1^i$] {\em Labeled without $r$.} Those that have been
%   labeled with a path which does not contain $r$ in one of the
%   previous iterations.\\ $\cL_1^i = \set{ X \mid p_1(X) = |X| \text{
%       and } p_2(X) = 0 \text{ and } s(X) = 0, X \in \cO}$
% \item [$\cL_2^i$] {\em Labeled with $r$.} Those that have been
%   labeled with two subpaths of $\cl(X)$ containing $r$ from two
%   different rays in two previous iterations.\\ $\cL_2^i = \set{X
%     \mid 0 < p_1\left(X\right), p_2\left(X\right) < |X| \text{ and }
%     s(X) = 0, X \in \cO}$
% \item [$\cT_1^i$] {\em Type 1 / partially labeled.} Those that have
%   been labeled with one path containing $r$ from a single ray in one
%   of the previous iterations. Here, $p_1(X)$ denotes the length of
%   the subpath of $\cl(X)$ that $X$ has been so far labeled
%   with. Also, it is clear that such a path must start at $r$ and
%   must be in a ray different from the one corresponding to $p_1(X)$.\\
%   $\cT_1^i = \set{ X \mid 0 < p_1(X) < |X| \text{ and } p_2(X) = 0
%     \text{ and } s(X) > 0, X \in \cO}$
% \item [$\cT_2^i$] {\em Type 2 / not labeled.} Those that have not
%   been
%   labeled with a path in any previous iteration.\\
%   $\cT_2^i = \set{ X \mid p_1(X) = p_2(X) = 0 \text{ and } s(X) =
%     |X|, X \in \cO}$
% \end{enumerate}
% \vspace{-2mm}
% \begin{align*}
%   \cO &= \cL_1^i \cup \cL_2^i \cup \cT_1^i \cup \cT_2^i, \text{    } \forall i \in [p]\\
%   \cO_i &= \cT_1^i \cup \cT_2^i, \text{ } \forall i \in [p]
% \end{align*}


The set $\cO_i$ refers to the set of hyperedges $\cT_1^i \cup \cT_2^i$
in the $i$th iteration.  Note that $\cO_1 = \cO$.  In the $i$th
iteration, hyperedges from $\cO_i$ are assigned paths from $R_i$ using
the following rules. Also the end of the iteration, $\cL_1^{i+1},
\cL_2^{i+1}, \cT_1^{i+1}, \cT_2^{i+1}$ are set to $\cL_1^{i},
\cL_2^{i}, \cT_1^{i}, \cT_2^{i}$ respectively, along with some
case-specific changes mentioned in the rules below.

\temptext{repetition}

In this section we consider the problem of assigning paths from a
$k$-subdivided star $T$ to a given set system $\cF$ such that each set
$X \in \cF$ is of cardinality at most $k+2$.  Secondly, we present our
results only for the case when overlap graph $\bO(\cF)$ is connected.
%The
%overlap graph is well-known from the work of
%\cite{kklv10,nsnrs09,wlh02}.  We use the notation in
%\cite{kklv10}. Recall from Section~\ref{sec:prelims} that hyperedges
%$S$ and $S'$ are said to overlap, denoted by $S \overlap S'$, if $S$ and $S'$
%have a non-empty intersection but neither of them is contained in the
%other. The overlap graph $\bO(\cF)$ is a graph in which the vertices
%correspond to the sets in $\cF$, and the vertices corresponding to the
%hyperedges $S$ and $S'$ are adjacent if and only if they overlap.  Note
%that the intersection graph of $\cF$, $\bI(\cF)$ is different from
%$\bO(\cF)$ and $\bO(\cF) \subseteq \bI(\cF)$.  
A connected component of $\bO(\cF)$ is called an overlap component of
$\cF$.  An interesting property of the overlap components is that any
two distinct overlap components, say $\cO_1$ and $\cO_2$, are such
that any two sets $S_1 \in \cO_1$ and $S_2 \in \cO_2$ are disjoint,
or, w.l.o.g, all the sets in $\cO_1$ are contained within one set in
$\cO_2$.  This containment relation naturally determines a
decomposition of the overlap components into rooted containment trees.
We consider the case when there is only one rooted containment tree,
and we first present our algorithm when $\bO(\cF)$ is connected.  It
is easy to see that once the path labeling to the overlap component in
the root of the containment tree is achieved, the path labeling to the
other overlap components in the rooted containment tree is essentially
finding a path labeling when the target tree is a path: each target
path is a path that is allocated to sets in the root overlap
component.  Therefore, for the rest of this section, $\bO(\cF)$ is a
connected graph. Recall that we also consider the special case when
all hyperedges are of cardinality at most $k+2$.  By definition, a
$k$-subdivided star has a central vertex which we call the {\em root},
and each root to leaf path is called a {\em ray}.  First, we observe
that by removing the root $r$ from $T$, we get a collection of $p$
vertex disjoint paths of length $k+1$, $p$ being the number of leaves
in $T$.  We denote the rays by $R_1, \ldots, R_p$, and the number of
vertices in $R_i$, $i \in [p]$ is $k+2$.  Let
$\seq{v_{i1},\ldots,v_{i(k+2)}=r}$ denote the sequence of vertices in
$R_i$, where $v_{i1}$ is the leaf. Note that $r$ is a common vertex to
all $R_i$.
  


\subsection{Description of the Algorithm}
In this section the given hypergraph $\cF$, the $k$-subdivided star
and the root of the star are denoted by $\cO$, $T$ and vertex $r$,
respectively.  In particular, note that the vertices of $\cO$
correspond to the sets in $\cF$, and the edges correspond to the
overlap relation.

\noindent
For each hyperedge $X \in \cO$, we will maintain a 2-tuple of non-negative
 numbers $\seq{p_1(X), p_2(X)}$.  The numbers satisfy the property that
 $p_1(X) + p_2(X) \leq |X|$, and at the end of path labeling, for each
 $X$, $p_1(X) + p_2(X) = |X|$.  This signifies the algorithm tracking
 the lengths of subpaths of the path assigned to $X$ from at most two
 rays. We also maintain another parameter called the {\em residue} of
 $X$ denoted by $s(X)=|X| - p_1(X)$. This signifies the residue path
 length that must be assigned to $X$ which must be from another
 ray. For instance, if $X$ is labeled a path from only one ray, then
 $p_1(X) = |X|$, $p_2(X) = 0$ and $s(X) = 0$.

 \xnoindent The algorithm proceeds in iterations, and in the $i$-th
 iteration, $i > 1$, a single hyperedge $X$ that overlaps with a
 hyperedge that has been assigned a path is considered.  At the
 beginning of each iteration hyperedges of $\cO$ are classifed into
 the following disjoint sets.
 \begin{enumerate}
 \item [$\cL_1^i$] {\em Labeled without $r$.} Those that have been
   labeled with a path which does not contain $r$ in one of the
   previous iterations.\\  $\cL_1^i = \set{ X \mid p_1(X) = |X| \text{ and
     } p_2(X) = 0 \text{ and } s(X) = 0, X \in \cO}$
 \item [$\cL_2^i$] {\em Labeled with $r$.} Those that have been labeled
   with two subpaths of $\cl(X)$ containing $r$ from two different rays
   in two previous iterations.\\ $\cL_2^i = \set{X \mid 0 < p_1\left(X\right),
     p_2\left(X\right) < |X|=p_1(X)+p_2(X) \text{ and } s(X) = 0, X \in \cO}$
   \item [$\cT_1^i$] {\em Type 1 / partially labeled.} Those that have
     been labeled with one path containing $r$ from a single ray in one
     of the previous iterations. Here, $p_1(X)$ denotes the length of
     the subpath of $\cl(X)$ that $X$ has been so far labeled
     with. \\
     $\cT_1^i = \set{ X \mid 0 < p_1(X) < |X| \text{ and } p_2(X) = 0
       \text{ and } s(X) = |X|-p_1(X), X \in \cO}$
   \item [$\cT_2^i$] {\em Type 2 / not labeled.} Those that have not been
     labeled with a path in any previous iteration.\\
     $\cT_2^i = \set{ X \mid p_1(X) = p_2(X) = 0 \text{ and } s(X) = |X|,
       X \in \cO}$
 \end{enumerate}
% \vspace{-2mm}
% \begin{align*}
%   \cO &= \cL_1^i \cup \cL_2^i \cup \cT_1^i \cup \cT_2^i, \text{    } \forall i \in [p]\\
%   \cO_i &= \cT_1^i \cup \cT_2^i, \text{    } \forall i \in [p]
% \end{align*}

\noindent
The set $\cO_i$ refers to the set of hyperedges $\cT_1^i \cup \cT_2^i$
in the $i$th iteration.  Note that $\cO_1 = \cO$.  In the $i$th
iteration, hyperedges from $\cO_i$ are assigned paths from $T$ using
the following rules. Also the end of the iteration, $\cL_1^{i+1},
\cL_2^{i+1}, \cT_1^{i+1}, \cT_2^{i+1}$ are set to $\cL_1^{i},
\cL_2^{i}, \cT_1^{i}, \cT_2^{i}$ respectively, along with some
case-specific changes mentioned in the rules below.

\noindent
\begin{enumerate}[I.]
\item {\bf Iteration 1:} Let $S=\{X_1,\ldots,X_s\}$ denote the
  super-marginal hyperedges from $\cO_1$.  If $|S|=s \neq p$, then
  exit reporting failure.  Else, assign to each $X_j \in S$, the path
  from $R_j$ such that the path contains the leaf in $R_j$.  This path
  is refered to as $\cl(X_j)$.  Set $p_1(X_j)=|X|, p_2(X_j)=s(X_j)=0$.
  Hyperedges in $S$ are not added to $\cO_2$ but are added to
  $\cL_1^2$ and all other hyperedges are added to
  $\cO_2$.
\item {\bf Iteration $i$:} Let $X$ be a hyperedge from $\cO_i$ such
  that there exists $Y \in \cL_1^i \cup \cL_2^i$ and $X \overlap
  Y$. Further let $Z \in \cL_1^i \cup L_2^i$ such that $Z \overlap Y$.
  If $X \in \cT_2^i$, and if there are multiple $Y$ candidates then
  any $Y$ is selected.  On the other hand, if $X \in \cT_1^i$, then
  $X$ has a partial path assignment, $\cl'(X)$ from a previous
  iteration, say from ray $R_j$. Then, $Y$ is
  chosen such that $X \cap Y$ has a non-empty intersection with a ray
  different from $R_j$.  The key things that are done in assigning a
  path to $X$ are as follows. The {\em end} of path $\cl(Y)$ where
  $\cl(X)$ would overlap is found, and then based on this the
  existence of a feasible assignment is decided.  It is important to
  note that since $X \overlap Y$, $\cl(X) \overlap \cl(Y)$ in any
  feasible assignment.  Therefore, the notion of the {\em end} at
  which $\cl(X)$ and $\cl(Y)$ overlap is unambiguous, since for any
  path, there are two end points.
  \begin{enumerate}
  \item \label{iendpoint} {\em End point of $\cl(Y)$ where $\cl(X)$ overlaps
      depends on $X \cap Z$:} If $X \cap Z
    \neq \emptyset$, then $\cl(X)$ has an overlap of $|X \cap Y|$ at that
    end of $\cl(Y)$ at which $\cl(Y)$ and $\cl(Z)$ overlap.  If $X
    \cap Z = \emptyset$, then $\cl(X)$ has an overlap of $|X \cap Y|$ at
    that end of $\cl(Y)$ where $\cl(Y)$ and $\cl(Z)$ do not intersect.
  \item {\em Any path of length $s(X)$ at the appropriate end contains
      $r$:} If $X \in \cT_1^i$ then after finding the appropriate end
    as in step~\ref{iendpoint} this the unique path of length $s(X)$
    should end at $r$.  If not, we exit reporting failure.  Else,
    $\cl(X)$ is computed as union of $\cl'(X)$ and this path. If any
    three-way intersection cardinality is violated with this new
    assignment, then exit, reporting failure.  Otherwise, $X$ is added
    to $\cL_2^{i+1}$.  On the other hand, if $X \in \cT_2^i$, then
    after step~\ref{iendpoint}, $\cl(X)$ or $\cl'(X)$ is unique up to
    the root and including it. Clearly, the vertices $\cl(X)$ or
    $\cl'(X)$ contains depends on $|X|$ and $|X \cap Y|$.  If any
    three way intersection cardinality is violated due to this
    assignment, exit, reporting failure.  Otherwise,
    $p_1(X)$ is updated as the length of the assigned path, and $s(X)
    = |X|-p_1(X)$.  If $s(X) > 0$, then $X$ is added to $\cT_1^{i+1}$.
    If $s(X)=0$, then $X$ is added to $\cL_1^{i+1}$.
  \item {\em The unique path of length $s(X)$ overlapping at the
      appropriate end of $Y$ does not contain $r$:} In this case,
    $\cl(X)$ is updated to include this path.  If any three way
    intersection cardinality is violated, exit, reporting failure.
    Otherwise, update $p_1(X)$ and $p_2(X)$ are appropriate, $X$ is
    added to $\cL_1^{i+1}$ or $\cL_2^{i+1}$, as appropriate.
  \end{enumerate}
\end{enumerate}

\noindent {\bf Proof of Correctness and Analysis of Running Time:} It
is clear that the algorithm runs in polynomial time, as at each step,
at most three-way intersection cardinalities need to be checked.
Further, finding super-marginal hyperedges can also be done in
polynomial time, as it involves considering the overlap regions and
checking if the inclusion partial order contains a single minimal
element.  In particular, once the super-marginal edges are identified,
each iteration involes finding the next hyperedge to consider, and
testing for a path to that hyperedge.  To identify the next hyperedge
to consider, we consider the breadth first layering of the hyperedges
with the zeroeth layer consisting of the super-marginal hyperedges.
Since $\cO$ is connected, it follows that all hyperedges of $\cO$ will
be considered by the algorithm.  Once a hyperedge is considered, the
path to be assigned to it can also be computed in constant time.  In
particular, in the algorithm the path to be assigned to $X$ depends on
$\cl(Y), \cl(Z)$, $s(X)$ and the presence or absence of $r$ in the
candidate partial path $\cl'(X)$.  Therefore, once the super-marginal
edges are identified, the running time of the algorithm is linear in
the size of the input.  By the technique used for constructing prime
matrices \cite{wlh02}, the super-marginal edges can be found in linear
time in the input size.  Therefore, the algorithm can be implemented
to run in linear time in the input size.

\noindent
The proof of correctness uses the following main properties:
\begin{enumerate}
\item The $k$-subdivided star has a very symmetric structure.  This
  symmetry is quantified based on the following observation -- either
  there are no feasible path labelings of $\cO$ using paths from $T$,
  or there are exactly $p!$ feasible path labelings.  In other words,
  there is either no feasible assignment, or effectively a unique
  assignment modulo symmetry.
\item The $p$ super-marginal hyperedges, if they exist, will each be
  assigned a path from distinct rays, and each such path contains the
  leaf.
\item For a candidate hyperedge $X$, the partial path assignment
  $\cl'(X)$ is decided by its overlap with $\cl(Y)$ and cardinality of
  intersection with $\cl(Z)$.
\end{enumerate}
These properties are formalized as follows:
\begin{lemma}
  \label{lem:sup-mar}
  If $X \in \cF$ is super-marginal and $\cl$ is a feasible tree path
  labeling to tree $T$, then $\cl(X)$ will contain a leaf in $T$.
\end{lemma}
\begin{proof}\thesisspacing
  Suppose $X \in \cF$ is super-marginal and $(\cF, \cl)$ is a feasible
  path labeling from $T$.  Assume $\cl(X)$ does not have a leaf.  Let
  $R_i$ be one of the rays (or the only ray) $\cl(X)$ is part of.
  Since $X$ is in a connected overlap component, there exists $Y_1 \in
  \cF$ and $X \nsubseteq Y_1$ such that $Y_1 \overlap X$ and $Y_1$ has
  at least one vertex closer to the leaf in $R_i$ than any vertex in
  $X$. Similarly with the same argument there exists $Y_2 \in \cF$
  with same subset and overlap relation with $X$ except it has has at
  least one vertex farther away from the leaf in $R_i$ than any vertex
  in $X$. Clearly $Y_1 \cap X$ and $Y_2 \cap X$ cannot be part of same
  inclusion chain which contradicts that assumption $X$ is
  super-marginal. Thus the claim is proved.\qed
\end{proof}
\begin{lemma}
  If $\cO$ does not have any super-marginal edges, then in any
  feasible path labeling $\cl$ of $\cO$ with paths from $T$ is such
  that, for any hyperedge $X$ for which $\cl(X)$ contains a leaf, $|X|
  \geq k+3$.
\end{lemma}
\begin{proof}\thesisspacing
  The proof of this lemma is by contradiction.  Let $X$ be a
  hyperedges such that $|X| \leq k+2$ and that $\cl(X)$ has a leaf.
  This implies that the overlap regions with $X$, which are captured by
  the overlap regions with $\cl(X)$, will form a single inclusion
  chain. This shows that $X$ is a marginal hyperedge which
  contradicts the assumption that $\cO$ does not have super-marginal
  hyperedges. \qed
\end{proof}
This lemma is used to prove the next lemma for the case when for all
$X \in \cO$, $|X| \leq k+2$.  The proof is left out as it just uses
the previous lemma and the fact that the hyperedges in $X$ have at
most $k+2$ elements.
\begin{lemma}
  If there is a feasible path labeling for $\cO$ in $T$, then there
  are exactly $p$ super-marginal hyperedges.
\end{lemma}
These lemmas now are used to prove the following theorem.
\begin{theorem}
  Given $\cO$ and a $k$-subdivided star $T$, the above algorithm
  decides correctly if there is a feasible path labeling $\cl$.
\end{theorem}
\begin{proof}\thesisspacing {\em Outline.}
%   If the algorithm outputs a path labeling $\cl$, then it is clear
%   that it is an ICPPL. The reason is that the algorithm checks that
%   three-way intersection cardinalities are preserved in each iteration
%   which ensures \icpplpr~\ref{pr:iii}. Moreover, it is clear that
%   $\cl(X)$ for any $X \in \cO$ is computed by maintaining
%   \icpplpr~\ref{pr:i} and \icpplpr~\ref{pr:ii}. For such a labeling
%   $\cl$, the proof that it is feasible is by induction on $k$. What
%   needs to be shown is that Algorithm~\ref{al:icppl-find-isomorph}
%   successfully runs on input $(\cO, \cl)$. In base case $k=0$, $T$ is
%   a star. The claim is clear by observing that after Filter 1 and one
%   iteration of Filter 2, all the leaves have found their pre-images
%   from support $\cO$.  Therefore, in the induction step, it is clear
%   that after Filter 1 and one iteration of Filter 2, the leaves are
%   assigned pre-images.  Removing the leaves from $T$ and the
%   pre-images from support of $\cO$, results in an ICPPL to a
%   $(k-1)$-subdivided star.  Now we apply the induction hypothesis, and
%   we get a isomorphism between the hypergraph $\cO$ and $\cO^\cl$.
  If the algorithm outputs a path labeling $\cl$, then it is clear
  that it is an ICPPL. The reason is that the algorithm checks that
  three-way intersection cardinalities are preserved in each iteration
  which ensures ICPPL \icpplpr~\ref{pr:iii}. Moreover, it is clear
  that $\cl(X)$ for any $X \in \cO$ is computed by maintaining ICPPL
  \icpplpr~\ref{pr:i} and ICPPL \icpplpr~\ref{pr:ii}. For such a
  labeling $\cl$, the proof that it is feasible is by induction on
  $k$. What needs to be shown is that
  Algorithm~\ref{al:icppl-find-isomorph} successfully runs on input
  $(\cO, \cl)$. In base case $k=0$, $T$ is a star. Also every set is
  at most size 2 ($k+2$) size and thus overlaps are at most 1. If two
  paths share a leaf in {\tt filter\_1} one must be of length 2 and
  the other of length 1. Thus the exit condition is not met. Further,
  it is also clear that the exit condition in {\tt filter\_2} is also
  not met. Thus claim proven for base case.  Now assume the claim to
  be true when target tree is a $(k-1)$-subdivided star. Consider the
  case of a $k$-subdivided star.  We can show that after {\tt
    filter\_1} and one iteration of a modified {\tt filter\_2} {\em
    all} leaves are assigned pre-images.  Removing the leaves from $T$
  and the pre-images from support of $\cO$, results in an ICPPL to a
  $(k-1)$-subdivided star.  Now we apply the induction hypothesis, and
  we get an isomorphism between the hypergraphs $\cO$ and $\cO^\cl$.

  \noindent
  In the reverse direction if there is a feasible path labeling $\cl$,
  then we know that $\cl$ is unique up to isomorphism.  Therefore,
  again by induction on $k$ it follows that the algorithm finds $\cl$.
  \qed
\end{proof}


% \noindent
% Consider the overlap graph $\bO(\cF)$ of the given hypergraph
% $\cF$. Let $S_{sm} \in \cF$ be such that it is a super-marginal
% hyperedge.  Algorithm~\ref{al:ktree-label} uses $S_{sm}$ along with
% the overlap graph $\bO(\cF)$ to calculate the feasible tree path
% labeling to the $k$-subdivded tree $T$.
%
% \begin{algorithm}[h]
%   \caption{{\tt compute-ksubtree-path-labeling($X, \cF, T$)}}
%   \label{al:ktree-label}
%   \begin{algorithmic}[\lndisplay]
%     \IF{$X = S_{sm}$}
%     \STATE -- TBD --\\
%     \ELSE
%     \STATE -- TBD --\\
%     \ENDIF
%
%   \end{algorithmic}
% \end{algorithm}


% \subsection{temp section from nsnsr09}

% \begin{algorithm}
%   \caption{Basic step in an algorithm to find an ICPIA for a prime
%     matrix $M'$}
%   \label{ds-algo}
%   ICPIA(Set $S$, Integer $p > 0$)

%   \noindent
%   /* {\tt $S \cap S^i \not = \phi$ for some $i \in \{1,\ldots,p\}$,
%     but $S \not\subseteq S^i$, $S^i \not\subseteq S$.  \\ Assigns to
%     $S$ an interval $I$ such that $\{I^1,\ldots,I^p,I\}$ forms an
%     ICPIA for $\{S^1,\ldots,S^p,S\}$.} */
%   \begin{algorithmic}
%     \STATE Let $|S \cap S^i| = z$. \\
%     \STATE Let $I_l$ be the interval such that $|I_l \cap I^i| = z$, $|I_l| = |S|$ and the $z$ common elements are the smallest elements of $I^i$. \\
%     \STATE Let $I_r$ be the interval such that $|I_r \cap I^i| = z$, $|I_r| = |S|$, and the $z$ common elements are the largest elements of $I^i$. \\
%     \IF {$p == 1$}
%     \STATE Assign $I_l$ to $S$ \\
%     \STATE /* {\tt In this case, $I_r$ could also be assigned to $S$.  This will yield the {\bf other} ICPIA} */\\
%     \ELSE \IF{$|I_l \cap I^q| = |S \cap S^q|$ for each $q \in
%       \{1,\ldots,p\}$}
%     \STATE Assign $I_l$ to $S$ and exit.\\
%     \ENDIF \IF{$|I_r \cap I^q| = |S \cap S^q|$ for each $q \in
%       \{1,\ldots,p\}$}
%     \STATE Assign $I_r$ to $S$ and exit.\\
%     \ENDIF \ENDIF \STATE Report no ICPIA and exit.
%   \end{algorithmic}
% \end{algorithm}


% \begin{theorem}
%   Algorithm~\ref{ds-algo} outputs an ICPIA to a prime matrix $M'$
%   \iff there is an ICPIA for $M'$.
% \end{theorem}
% \begin{proof}
%   The only-if part of the theorem is straightforward.  We now show
%   that if there is an ICPIA for $M'$, then Algorithm~\ref{ds-algo}
%   will indeed discover it.  The key fact is that in $M'$ for each
%   set $S$, there is another set $T \in M'$ such that $S \cap T \not
%   = \phi$, and $S$ and $T$ are not contained in each other.  Due to
%   this fact, there are exactly two ICPIAs for $M'$.  The two
%   distinct ICPIAs differ based on the interval assigned to $S_1$,
%   see Algorithm ~\ref{ds-algo}.  If $I_l$ is assigned to $S_1$, then
%   we get one, and the other ICPIA is obtained by assigning $I_r$ to
%   $S_1$.  For each subsequent set, say $S^j$, the interval to be
%   assigned is forced.  It is forced due to the fact that the
%   interval assigned to $S^j$ is based on the interval assigned to
%   $S^i$, where $S^i \cap S^j \not = \phi$, and $S^i \not\subseteq
%   S^j$, and $S^j \not\subseteq S^i$.  Given the fact that the
%   algorithm is an exact implementation of these observations, it
%   follows that Algorithm~\ref{ds-algo} finds an ICPIA if there is
%   one.
% \end{proof}





\section{ TPL with no restrictions}
\label{sec:norestraint}

{\tt abstract begin} it is known that if the given tree is a path a
feasible assignment can be found in polynomial time, and we observe
that it can actually be done in logspace. \tnote{[TRUE?]}  {\tt
  abstract end}

In Section \ref{prelims} we present the necessary preliminaries, in
Section \ref{feasible} we present our characterization of feasible
tree path assignments, and in Section \ref{decompos} we present the
characterizing subproblems for finding a bijection between $U$ and
$V(T)$ such that sets map to tree paths. Finally, in Section
\ref{complexity} we conclude by showing that Tree Path Assignment is
GI-Complete, and also observe that Consecutive Ones Testing is in
Logspace.


%%%%%%%%%%%%%%%%%%%%%%%%%%%%%%%%%%%%% DUMMYTOC \subsection{Introduction}

\annote{ We refer to this as the Tree Path Assignment problem for an
  input $(\cF,T)$ pair.}{Add this ``pair'' as a definition in
  ch:prelims}

\remove{It is an interesting fact that for a matrix with the COP, the
  intersection graph of the corresponding set system is an interval
  graph.  A similar connection to a subclass of chordal graphs, and
  this subclass contains interval graphs, exists for the
  generalization of COP.  In this case, the intersection graph of the
  corresponding set system must be a path graph. Chordal graphs are of
  great significance, extensively studied, and have several
  applications.  One of the well known and interesting properties of a
  chordal graphs is its connection with intersection graphs
  cite{mcg04}. For every chordal graph, there exists a tree and a
  family of subtrees of this tree such that the intersection graph of
  this family is isomorphic to the chordal graph
  cite{plr70,gav78,bp93}.  Certain format of these trees are called
  clique trees cite{apy92} of the graph which is a compact
  representation of the chordal graph. There has also been work done
  on the generalization of clique trees to clique hypergraphs
  cite{km02}.  If the chordal graph can be represented as the
  intersection graph of paths in a tree, then the graph is called path
  graph cite{mcg04} or path chordal graph.  Therefore, it is clear
  that if there is a bijection from $U$ to $V(T)$ such that the sets
  map to paths, then the intersection graph of the set system is
  indeed a path chordal graph.  This is, however, only a necessary
  condition and can be checked efficiently, as path graph recognition
  is polynomial solvable cite{gav78,aas93}.  Indeed, it is possible to
  construct a set system and tree, such that the intersection graph is
  a path chordal graph, but there is no bijection between $U$ and
  $V(T)$ such that the sets map to paths.  This connection indeed
  suggests that our problem is indeed as hard as path graph
  isomorphism.  Further path graph isomorphism is known be
  isomorphism-complete, see for example cite{kklv10}. } In the second
part of this paper, we decompose our search for a bijection between
$U$ and $V(T)$ into subproblems.  Each subproblem is on a set system
in which for each set, there is another set in the set system with
which the intersection is {\em strict}- there is a non-empty
intersection, but neither is contained in the other.  This is in the
spirit of results in \cite{wlh02,nsnrs09} where to test for COP in a
given matrix, the COP problem is solved on an equivalent set of prime
matrices.  \annote{Our decomposition localizes the challenge of path
  graph isomorphism to two problems.}{HOW?}

Finally, we show that Tree Path Assignment is isomorphism-complete.
We also point out Consecutive Ones Testing is in Logspace from two
different results in the literature \cite{kklv10, mcc04}. To the best
of our knowledge this observation has not been made earlier.


\subsection{Finding an assignment of tree paths to a set
  system} \label{decompos} In the previous section we have shown that
the problem of finding a Tree Path Asssignment to an input $(\cF,T)$
is equivalent to finding an ICPPA to $\cF$ in tree $T$.  In this
section we characterize those set systems that have an ICPPA in a
given tree.  As a consequence of this characterization we identify two
sub-problems that must be solved to obtain an ICPPA.  We do not solve
the problem and in the next section show that finding an ICPPA in a
given tree is GI-Complete.

\noindent
A set system can be concisely represented by a binary matrix where the
row indices denote the universe of the set system and the column
indices denote each of the sets. Let the binary matrix be $M$ with
order $n \times m$, the set system be $\cF = \{S_i \mid i \in [m]\}$,
universe of set system $U = \{x_1, \dots ,x_n\}$. If $M$ represents
$\cF$, $|U| = n, |\cF| = m$. Thus $(i,j)$th element of $M$, $M_{ij} =
1$ \iff $x_i \in S_j$. If $\cF$ has a feasible tree path assignment
(ICPPA) $\cA = \{(S_i,P_i) \mid i \in [m]\}$, then we say its
corresponding matrix $M$ has an ICPPA. Conversly we say that a matrix
$M$ has an ICPPA if there exists an ICPPA $\cA$ as defined
above.\\
\noindent
We now define the strict intersection graph or overlap graph of
$\cF$. This graph occurs at many places in the literature, see for
example \cite{kklv10, wlh02, nsnrs09}.  The vertices of the graph
correspond to the sets in $\cF$.  An edge is present between vertices
of two sets \iff the corresponding sets have a nonempty intersection
and none is contained in the other. Formally, intersection graph is
$G_f = (V_f, E_f)$ such that $V_f = \{v_i \mid S_i \in \cF\}$ and $E_f
= \{(v_i, v_j) \mid S_i \cap S_j \ne \O \text{ and }S_i \nsubseteq
S_j, S_j \nsubseteq S_i \}$.  We use this graph to decompose $M$ as
described in \cite{wlh02,nsnrs09}.  A prime sub-matrix of $M$ is
defined as the matrix formed by a set of columns of $M$ which
correspond to a connected component of the graph $G_f$.  Let us denote
the prime sub-matrices by $M_1,\ldots,M_p$ each corresponding to one
of the $p$ components of $G_f$. Clearly, two distinct matrices have a
distinct set of columns.  Let $col(M_i)$ be the set of columns in the
sub-matrix $M_i$.  The support of a prime sub-matrix $M_i$ is defined
as $supp(M_i) = \displaystyle \bigcup_{j \in col(M_i)}S_j$.  Note that
for each $i$, $supp(M_i) \subseteq U$.  For a set of prime
sub-matrices $X$ we define
$supp(X) = \displaystyle \bigcup_{M \in X} supp(M)$. \\


\noindent
Consider the relation $\preccurlyeq$ on the prime sub-matrices $M_1,
\ldots, M_p$ defined as follows:
\begin{equation}
  \nonumber \{(M_i,M_j) | \mbox{ A set } S \in
  M_i \mbox{ is contained in a set } S' \in M_j\} \cup \{(M_i,M_i) | 1
  \leq i \leq p\} 
\end{equation}

\noindent
This relation is the same as that defined in \cite{nsnrs09}. The prime
submatrices and the above relation can be defined for any set
system. We will use this structure of prime submatrices to present our
results on an an ICPPA for a set system $\cF$. Recall the following
lemmas, and theorem that $\preccurlyeq$ is a partial order, from
\cite{nsnrs09}.

\begin{lemma} \label{lem:containment} Let $(M_i,M_j) \in
  \preccurlyeq$.  Then there is a set $S' \in M_j$ such that for each
  $S \in M_i$, $S \subseteq S'$.
\end{lemma}
\begin{lemma}
  For each pair of prime sub-matrices, either $(M_i,M_j) \not\in
  \preccurlyeq$ or $(M_j,M_i) \not\in \preccurlyeq$.
  % If $(M_i,M_j) \in \preccurlyeq$ and $(M_j,M_i) \in \preccurlyeq$,
  % then $i = j$ and $|M_i| = 1$.
\end{lemma}
\begin{lemma}
  If $(M_i,M_j) \in \preccurlyeq $ and $(M_j,M_k) \in \preccurlyeq$,
  then $(M_i,M_k) \in \preccurlyeq$.
\end{lemma}
\begin{lemma}
  If $(M_i,M_j) \in \preccurlyeq$ and $(M_i,M_k) \in \preccurlyeq$,
  then either $(M_j,M_k) \in \preccurlyeq$ or $(M_k,M_j) \in
  \preccurlyeq$.
\end{lemma}
\begin{theorem} \label{thm:partitionold} $\preccurlyeq$ is a partial
  order on the set of prime sub-matrices of $M$.  Further, it uniquely
  partitions the prime sub-matrices of $M$ such that on each set in
  the partition $\preccurlyeq$ induces a total order.
\end{theorem}
For the purposes of this paper, we refine the total order mentioned in
Theorem \ref{thm:partitionold}. We do this by identifying an in-tree
rooted at each maximal upper bound under $\preccurlyeq$.  Each of
these in-trees will be on disjoint vertex sets, which in this case
would be disjoint sets of prime-submatrices.  The in-trees are
specified by selecting the appropriate edges from the Hasse diagram
associated with $\preccurlyeq$.  Let $\cI$ be the following set:
\begin{align*}
  \cI = \{ (M_i,M_j) \in \preccurlyeq \mid \nexists M_k s.t. M_i
  \preccurlyeq M_k, M_k \preccurlyeq M_j \} \cup \{ (M_i,M_i), i \in
  [p] \}
\end{align*}

\begin{theorem} \label{thm:partition} Consider the directed graph $X$
  whose vertices correspond to the prime sub-matrices, and the edges
  are given by $\cI$.  Then, $X$ is a vertex disjoint collection of
  in-trees and the root of each in-tree is a maximal upper bound in
  $\preccurlyeq$.
\end{theorem}
\begin{proof}\thesisspacing
  To observe that $X$ is a collection of in-trees, we observe that for
  vertices corresponding to maximal upper bounds, no out-going edge is
  present in $I$.  Secondly, for each other element, exactly one
  out-going edge is chosen, and for the minimal lower bound, there is
  no in-coming edge.  Consequently, $X$ is acyclic, and since each
  vertex has at most one edge leaving it, it follows that $X$ is a
  collection of in-trees, and for each in-tree, the root is a maximal
  upper bound in $\preccurlyeq$.  Hence the theorem.
\end{proof}
Let the partition of $X$ given by Theorem \ref{thm:partition} be
$\{X_1,\ldots,X_r\}$.  Further, each in-tree itself can be layered
based on the distance from the root.  The root is considered to be at
level zero. For $j \geq 0$, Let $X_{i,j}$ denote the set of prime
matrices in level $j$ of in-tree $X_i$.

\begin{lemma}
  \label{lem:subicppa}
  Let $M$ be a matrix and let $X$ be the directed graph whose vertices
  are in correspondence with the prime submatrices of $M$.  Further
  let $\{X_1,\ldots,X_r\}$ be the partition of $X$ into in-trees as
  defined above.  Then, matrix $M$ has an ICPPA in tree $T$ \iff $T$
  can be partitioned into vertex disjoint subtrees $\{T_1, T_2, \dots
  T_r\}$ such that, for each $1 \leq i \leq r$, the set of prime
  sub-matrices corresponding to vertices in $X_i$ has an ICPPA in
  $T_i$.
\end{lemma}
\begin{proof}\thesisspacing
  Let us consider the reverse direction first.  Let us assume that $T$
  can be partitioned into $T_1, \ldots, T_r$ such that for each $1
  \leq i \leq r $, the set of prime sub-matrices corresponding to
  vertices in $X_i$ has an ICPPA in $T_i$.  It is clear from the
  properties of the partial order $\preccurlyeq$ that these ICPPAs
  naturally yield an ICPPA of $M$ in $T$.  The main property used in
  this inference is that for each $1 \leq i \neq j \leq r$, $supp(X_i)
  \cap supp(X_j) = \phi$.

\noindent
To prove the forward direction, we show that if $M$ has an ICPPA, say
$\cA$, in $T$, then there exists a partition of $T$ into vertex
disjoint subtree $T_1, \ldots, T_r$ such that for each $1 \leq i \leq
r$, the set of prime sub-matrices corresponding to vertices in $X_i$
has an ICPPA in $T_i$.  For each $1 \leq i \leq r$, we define based on
$\cA$ a subtree $T_i$ corresponding to $X_i$.  We then argue that the
trees thus defined are vertex disjoint, and complete the proof.
Consider $X_i$ and consider the prime sub-matrix in $X_{i,0}$.
Consider the paths assigned under $\cA$ to the sets in the prime
sub-matrix in $X_{i,0}$.  Since the component in $G_f$ corresponding
to this matrix is a connected component, it follows that union of
paths assigned to this prime-submatrix is a subtree of $T$.  We call
this sub-tree $T_i$.  All other prime-submatrices in $X_i$ are
assigned paths in $T_i$ since $\cA$ is an ICPPA, and the support of
other prime sub-matrices in $X_i$ are contained in the support of the
matrix in $X_{i,0}$.  Secondly, for each $1 \leq i \neq j \leq r$,
$supp(X_i) \cap supp(X_j) = \phi$, and since $\cA$ is an ICPPA, it
follows that $T_i$ and $T_j$ are vertex disjoint.  Finally, since $|U|
= |V(T)|$, it follows that $T_1, \ldots, T_r$ is a partition of $T$
into vertex disjoint sub-trees such that for each $1 \leq i \leq r$,
the set of matrices corresponding to nodes in $X_i$ has an ICPPA in
$T_i$.  Hence the lemma.
\end{proof}
The essence of the following lemma is that an ICPPA only needs to be
assigned to the prime sub-matrix corresponding to the root of each
in-tree, and all the other prime sub-matrices only need to have an
Intersection Cardinality Preserving Interval Assignments (ICPIA).
Recall, an ICPIA is an assignment of intervals to sets such that the
cardinality of an assigned interval is same as the cardinality of the
interval, and the cardinality of intersection of any two sets is same
as the cardinality of the intersection of the corresponding intervals.
It is shown in \cite{nsnrs09} that the existence of an ICPIA is a
necessary and sufficient condition for a matrix to have COP.  We
present the pseudo-code to test if $M$ has an ICPPA in $T$.
\begin{lemma} \label{lem:rooticppa} Let $M$ be a matrix and let $X$ be
  the directed graph whose vertices are in correspondence with the
  prime submatrices of $M$.  Further let $\{X_1,\ldots,X_r\}$ be the
  partition of $X$ into in-trees as defined earlier in this section.
  Let $T$ be the given tree and let $\{T_1, \ldots, T_r\}$ be a given
  partition of $T$ into vertex disjoint sub-trees.  Then, for each $1
  \leq i \leq r$, the set of matrices corresponding to vertices of
  $X_i$ has an ICPPA in $T_i$ if and only if the matrix in $X_{i,0}$
  has an ICPPA in $T_i$ and all other matrices in $X_i$ have an {\bf
    {\em ICPIA}}.
\end{lemma}
\begin{proof}\thesisspacing
  The proof is based on the following fact- $\preccurlyeq$ is a
  partial order and $X$ is a digraph which is the disjoint union of
  in-trees.  Each edge in the in-tree is a containment relationship
  among the supports of the corresonding sub-matrices. Therefore, any
  ICPPA to a prime sub-matrix that is not the root is contained in a
  path assigned to the sets in the parent matrix.  Consequently, any
  ICPPA to the prime sub-matrix that is not at the root is an ICPIA,
  and any ICPIA can be used to construct an ICPPA to the matrices
  corresponding to nodes in $X_i$ provided the matrix in the root has
  an ICPPA in $T_i$.  Hence the lemma.
\end{proof}
Lemma \ref{lem:subicppa} and Lemma \ref{lem:rooticppa} point out two
algorithmic challenges in finding an ICPPA for a given set system
$\cF$ in a tree $T$.  Given $\cF$, finding $X$ and its partition
$\{X_1,\ldots,X_r\}$ into in-trees can be done in polynomial time.  On
the other hand, as per lemma \ref{lem:subicppa} we need to parition
$T$ into vertex disjoint sub-trees $\{T_1, \ldots, T_r\}$ such that
for each $i$, the set of matrices corresponding to nodes in $X_i$ have
an ICPPA in $T_i$.  This seems to be a challenging step, and it must
be remarked that this step is easy when $T$ itself is a path, as each
individual $T_i$ would be sub-paths.  The second algorithmic challenge
is identified by lemma \ref{lem:rooticppa} which is to assign an ICPPA
from a given tree to the matrix associated with the root node of
$X_i$.
\begin{algorithm}[h]
  \caption{Algorithm to find an ICPPA for a matrix $M$ on tree $T$:
    $main\_ICPPA(M, T$)}
  \label{al:icppa-main}
  \begin{algorithmic}
    \STATE Identify the prime sub-matrices. This is done by constructing the strict overlap graph and identify connected components.  Each connected component yields a prime sub-matrix.   \\
    \STATE Construct the partial order $\preccurlyeq$ on the set of prime sub-matrices.  \\
    \STATE Construct the partition $X_1,\ldots,X_r$ of the prime
    sub-matrices induced by $\preccurlyeq$ \\
    \STATE For each $1 \leq i \leq r$, Check if all matrices except
    those in $X_{i,0}$ has an ICPIA.  If a matrix does not have ICPIA
    exit with a negative answer.  To check for the existence of ICPIA,
    use the result in \cite{nsnrs09}.  \STATE Find a partition of
    $T_1, \ldots, T_r$ such that matrices in $X_{i,0}$ has an ICPPA in
    $T_i$.  If not such partition exists, exit with negative answer.
  \end{algorithmic}
\end{algorithm}


\section{Complexity of Tree Path Assignment-A Discussion}
\label{complexity}
Recall that the input to the Tree Path Assignment question is an order
pair $(\cF,T)$ where $\cF$ is a family of subsets of an universe $U$,
and $T$ is a tree such that $|V(T)| = |U|$.  The question is to come
up with a bijection from $U$ to $V(T)$ such that the image of each set
in $\cF$ is a path in $T$.


\subsection{Consecutive Ones Testing is in Logspace}
While Tree Path Assignment is isomorphism-complete, it is polynomial
time solvable when the given tree is a path.  Indeed, in this case we
encounter a restatement of matrices with the COP.  The known
approaches to testing for COP fall into two categories: those that
provide a witness when the input matrix does not have COP, and those
that do not provide a witness.  The first linear time algorithm for
testing COP for a binary matrix was using a data structure called PQ
trees, which represent all COP orderings of $M$, invented by
\cite{bl76}. There is a PQ tree for a matrix \iff the matrix has COP.
Indeed, this is an algorithmic characterization of the consecutive
ones property and the absence of the PQ-tree does not yield any
witness to the reason for failure.  A closely related data structure
is the generalized PQ tree in \cite{mcc04}.  In generalized PQ tree
the P and Q nodes are called prime and linear nodes. Aside from that,
it has a third type of node called degenerate nodes which is present
only if the set system does not have COP \cite{mcc04}.  Using the idea
of generalized PQ tree, \cite{mcc04} proves that checking for
bipartiteness in the certain incomparability graph is sufficient to
check for COP.  \cite{mcc04} invented a certificate to confirm when a
binary matrix does not have COP.  \cite{mcc04} describes a graph
called incompatibility graph of a set system $\cF$ which has vertices
$(a,b), a \ne b$ for every $a, b \in U$, $U$ being the universe of the
set system. There are edges $((a,b),(b,c))$ and $((b,a),(c,b))$ if
there is a set $S \in \cF$ such that $a, c \in S$ and $b \notin S$. In
other words the vertices of an edge in this graph represents two
orderings that cannot occur in a consecutive ones ordering of $\cF$.
\begin{theorem}[Theorem 6.1, \cite{mcc04}]
  Let $\cF$ be an arbitrary set family on domain $V$. Then $\cF$ has
  the consecutive ones property if and only if its incompatibility
  graph is bipartite, and if it does not have the consecutive ones
  property, the incompatibility graph has an odd cycle of length at
  most $n+3$.
\end{theorem}
This theorem gives a certificate as to why a given matrix does not
have COP.  Similarly, the approach of testing for an ICPIA in
\cite{nsnrs09} also gives a different certificate- a prime sub-matrix
that does not have an ICPIA.  Further, the above theorem can be used
to check if a given matrix has COP in logspace by checking if its
incompatibility graph is bipartite. \cite{rei84} showed that checking
for bipartiteness can be done in logspace. Thus we conclude that
consecutive ones testing can be done in logspace.

\noindent
More recently, \cite{kklv10} showed that interval graph isomorphism
can be done in logspace. Their paper proves that a canon for interval
graphs can be calculated in logspace using an interval hypergraph
representation of the interval graph with each hyperedge being a set
to which an interval shall be assigned by the canonization
algorithm. An overlap graph (subgraph of intersection graph, edges
define only strict intersections and no containment) of the hyperedges
of the hypergraph is created and canons are computed for each overlap
component. The overlap components define a tree like relation due to
the fact that two overlap components are such that either all the
hyperedges of one is disjoint from all in the other, or all of them
are contained in one hyperedge in the other. This is similar to the
containment tree defined in \cite{nsnrs09} and in this paper. Finally
the canon for the whole graph is created using logspace tree
canonization algorithm from \cite{sl92}. The interval labelling done
in this process of canonization is exactly the same as the problem of
assigning feasible intervals to a set system, and thus the problem of
finding a COP ordering in a binary matrix \cite{nsnrs09}.

\begin{theorem}[Theorem 4.7, \cite{kklv10}]
  \label{th:canonlabel}
  Given an interval hypergraph $\cH$, a canonical interval labeling
  $l_H$ for $H$ can be computed in FL.
\end{theorem}


We present the following reduction to see that COP testing is indeed
in logspace. Given a binary matrix $M$ of order $n \times m$, let $S_i
= \{j \mid M[j,i]=1 \}$. Let $\cF = \{S_i \mid i \in [m] \}$ be this
set system. Construct a hypergraph $\cH$ with its vertex set being
$\{1, 2, \dots n\}$. The edge set of $\cH$ is isomorphic to
$\cF$. Thus every edge in $\cH$ represents a set in the given set
system $\cF$. Let this mapping be $\pi: E(\cH) \rightarrow \cF$. It is
easy to see that if $M$ has COP, then $\cH$ is an interval
hypergraph. From theorem \ref{th:canonlabel}, it is clear that the
interval labeling $l_{\cH}: V(\cH) \rightarrow [n]$ can be calculated
in logspace. Construct sets $I_i = \{ \l_{\cH}(x) \mid x \in E, E \in
E(\cH), \pi(E) = S_i\}$, for all $i \in [m]$. Since $\cH$ is an
interval hypergraph, $I_i$ is an interval for all $i \in [m]$, and is
the interval assigned to $S_i$ if $M$ has COP.

Now we have the following corollary.
\begin{corollary}
  \label{cor:coplog}
  If a binary matrix $M$ has COP then the interval assignments to each
  of its columns can be calculated in FL.
\end{corollary}

\noindent
Finally, we conclude by asking about the complexity of Tree Path
Assignment restricted to other subclasses of trees.  In particular, is
Tree Path Assignment in caterpillars easier than Tree Path assignment
in general trees.



% \noindent
% It has been long known that interval graph recognition is in
% logspace\cite{rei84}. Recently interval graph isomorphism was also
% shown to be logspace decidable using a logspace canonization
% algorithm by \cite{kklv10}.  This result is built on top of logspace
% results of undirected graph connectivity \cite{rei08}, logspace
% tractability using a certain logical formalism called FP+C and
% modular decomposition of interval graphs\cite{lau10} etc.  Interval
% graphs are closely connected to binary matrices with COP. The
% maximal clique vertex incidence matrix (matrix with rows
% representing maximal cliques and columns representing vertices of a
% graph) has COP on columns \iff the graph is an interval
% graph\cite{fg65}. This follows from the interval graph
% characterization by \cite{gh64}. Due to this close relation it is
% natural to see if consecutive ones property can be
% tested in logspace. \\
% \noindent
% We also explore some extensions of the interval assignment problem
% in \cite{nsnrs09}, namely caterpillar path assignment problem.

% We present a logspace algorithm here that uses the ICPIA
% characterization of binary matrices with COP (set system associated
% with such a matrix)\cite{nsnrs09}.


%%%%%%%%%%%%%%%%%%%%%%%%%%%%%%%%%%%%% DUMMYTOC Assignment\tnote{Merge
%%%%%%%%%%%%%%%%%%%%%%%%%%%%%%%%%%%%%   with the original charac
%%%%%%%%%%%%%%%%%%%%%%%%%%%%%%%%%%%%%   section}}
% \label{feasible}
% Consider a path assignment $\cA = \{(S_i, P_i) \mid S_i \in \cF, P_i
% \text{ is a path from $T$}, i \in [m]\}$ to a set system $\cF =
% \{S_i \mid S_i \subseteq U, i \in [m]\}$, were $T$ is a given tree,
% $U$ is the set system's universe and $m$ is the number of sets in
% $\cF$. We call $\cA$ an {\em Intersection Cardinality Preserving
%   Path Assignment (ICPPA)} if it has the following properties.

% \begin{enumerate}
% \item [i.]  $|S_i| = |P_i|$ for all $i \in [m]$
% \item [ii.] $|S_i \cap S_j| = |P_i \cap P_j|$ for all $i,j \in [m]$
%   % \item [iii.] $|\bigcup_{i \in J} S_i| = |\bigcup_{i \in J} P_i|$
%   %   for all $J \subseteq I$\footnote{TBD: this is needed now to
%   %     fix
%   %     some proofs}
% \item [iii.] $|S_i \cap S_j \cap S_k| = |P_i \cap P_j \cap P_k|$ for
%   all $i,j,k \in [m]$
% \end{enumerate}

% \begin{lemma}
%   \label{lem:setminuscard}
%   If $\cA$ is an ICPPA, and $(S_1, P_1),(S_2, P_2),(S_3, P_3) \in
%   \cA$, then $|S_1 \cap (S_2 \setminus S_3)| = |P_1 \cap (P_2
%   \setminus P_3)|$.
% \end{lemma}
% \begin{proof}
%   $|S_1 \cap (S_2 \setminus S_3)| = |(S_1 \cap S_2) \setminus S_3| =
%   |S_1 \cap S_2| - |S_1 \cap S_2 \cap S_3|$. Due to conditions (ii)
%   and (iii) of ICPPA, $|S_1 \cap S_2| - |S_1 \cap S_2 \cap S_3| =
%   |P_1 \cap P_2| - |P_1 \cap P_2 \cap P_3| = |(P_1 \cap P_2)
%   \setminus P_3| = |P_1 \cap (P_2 \setminus P_3)|$. Thus lemma is
%   proven. %\qed
% \end{proof}

% \begin{lemma}
%   \label{lem:fourpaths} Consider four paths in a tree $Q_1, Q_2,
%   Q_3, Q_4$ such that they have nonempty pairwise intersection and
%   $Q_1, Q_2$ share a leaf. Then there exists distinct $i, j, k \in
%   \{1,2,3,4\}$ such that, $Q_1 \cap Q_2 \cap Q_3 \cap Q_4 = Q_i \cap
%   Q_j \cap Q_k$.
% \end{lemma}
% \begin{proof}\thesisspacing
%   {\em Case 1:} w.l.o.g, consider $Q_3 \cap Q_4$ and let us call it
%   $Q$. This is clearly a path (intersection of two paths is a path).
%   %   Since $Q_1, Q_2$ share a leaf, the following are paths $Q_1
%   %   \setminus Q_2$, $Q_2 \setminus Q_1$, $Q_1 \cap Q_2$ and they
%   %   are
%   %   mutually disjoint.
%   Suppose $Q$ does not intersect with $Q_1 \setminus Q_2$, i.e. $Q
%   \cap (Q_1 \setminus Q_2) = \O$. Then $Q \cap Q_1 \cap Q_2 = Q \cap
%   Q_2$. Similarly, if $Q \cap (Q_2 \setminus Q_1) = \O$, $Q \cap Q_1
%   \cap Q_2 = Q \cap Q_1$. Thus it is clear that if the intersection
%   of any two paths does not intersect with any of the set
%   differences of the remaining two paths, the claim in the lemma is
%   true.
%   %   Note that $Q_1 \setminus Q_2$ and $Q_2
%   %   \setminus Q_1$ are paths because $Q_1, Q_2$ share a leaf.\\
%   {\em Case 2:} Let us consider the compliment of the previous
%   case. i.e. the intersection of any two paths intersects with both
%   the set differences of the other two. First let us consider $Q
%   \cap (Q_1 \setminus Q_2) \ne \O$ and $Q \cap (Q_1 \setminus Q_2)
%   \ne \O$, where $Q = Q_3 \cap Q_4$. Since $Q_1$ and $Q_2$ share a
%   leaf, there is exactly one vertex at which they branch off from
%   the path $Q_1 \cap Q_2$ into two paths $Q_1 \setminus Q_2$ and
%   $Q_2 \setminus Q_1$. Let this vertex be $v$. It is clear that if
%   path $Q_3 \cap Q_4$, must intersect with paths $Q_1 \setminus Q_2$
%   and $Q_2 \setminus Q_1$, it must contain $v$ since these are paths
%   from a tree. Moreover, $Q_3 \cap Q_4$ intersects with $Q_1 \cap
%   Q_2$ at exactly $v$ and only at $v$ which means that $Q_1 \cap
%   Q_2$ does not intersect with $Q_3 \setminus Q_4$ or $Q_4 \setminus
%   Q_3$ which contradicts initial condition of this case. Thus this
%   case cannot occur and case 1 is the only possible scenario. \\
%   Thus lemma is proven %\qed
% \end{proof}


% \noindent
% In the remaining part of this section we show that a path assignment
% is feasible if and only if it is an ICPPA.  One direction of this
% claim is clear: that is a path assignment is feasible, then all
% intersection cardinalities are preserved, that is the path
% assignment is an ICPPA.  The reason is that a feasible path
% assignment has an associated bijection between $U$ and $V(T)$ such
% that the sets map to paths.  The rest of the section is devoted to
% constructively proving that it is sufficient, for the existence of a
% bijection, for a path assignment to be an ICPPA.  At a top-level,
% the constructive approaches refine the path assignment iteratively,
% such that at the end of each iteration we have a path assignment,
% and finally we have a family of bijections.  First we present and
% then prove the correctness of Algorithm \ref{al:filterone}.  This algorithm
% refines the path assignment by considering pairs of paths that share
% a leaf.

% \begin{algorithm}[h]
%   \caption{Permutations from an ICPPA $\{(S_i,P_i) | i \in I\}$}
%   \label{al:filterone}
%   \begin{algorithmic}
%     \STATE Let $\Pi_0=\{(S_i,P_i)| i \in I\}$\\
%     \STATE $j = 1$;
%     \label{shareleaf} \WHILE {There is $(P_1,Q_1), (P_2,Q_2) \in
%       \Pi_{j-1}$ with $Q_1$ and $Q_2$ having a common leaf} \STATE
%     $\Pi_j= \Pi_{j-1} \setminus \{(P_1,Q_1),(P_2,Q_2)\}$;
%     \label{setbreak}\STATE $\Pi_j = \Pi_j \cup \{(P_1 \cap P_2,Q_1
%     \cap Q_2), (P_1 \setminus P_2,Q_1 \setminus Q_2), (P_2 \setminus
%     P_1, Q_2 \setminus Q_1)\}$; \STATE $j = j+1$; \ENDWHILE \STATE
%     $\Pi = \Pi_j$; \STATE Return $\Pi$;
%   \end{algorithmic}
% \end{algorithm}

% \begin{lemma}
%   \label{lem:invar1}
%   In Algorithm \ref{al:filterone}, at the end of $j$th iteration, $j \ge
%   0$, of the while loop of Algorithm \ref{al:filterone}, the following
%   invariants are maintained.
%   \begin{itemize}
%   \item {\em Invariant I:} $Q$ is a path in $T$ for each $(P,Q) \in
%     \Pi_j$
%   \item {\em Invariant II:} $|P|=|Q|$ for each $(P,Q) \in \Pi_j$
%   \item {\em Invariant III:} For any two $(P,Q), (P',Q') \in \Pi_j$,
%     $|P' \cap P''|=|Q' \cap Q''|$.
%   \item {\em Invariant IV:} For any three, $(P',Q'), (P'',Q''), (P,
%     Q) \in \Pi_j$, $|P' \cap P'' \cap P|=|Q' \cap Q'' \cap Q|$.
%   \end{itemize}
% \end{lemma}
% \begin{proof}
%   Proof is by induction on the number of iterations, $j$. In the
%   rest of the proof, the term ``new sets'' will refer to the new
%   sets added in $j$th iteration as defined in line \ref{setbreak} of
%   Algorithm \ref{al:filterone}, i.e. the following three assignment pairs
%   for some $(P_1,Q_1), (P_2,Q_2) \in \Pi_{j-1}$ where $Q_1$ and
%   $Q_2$ intersect and share a leaf: $(P_1 \cap P_2, Q_1 \cap Q_2)$,
%   or $(P_1 \setminus P_2, Q_1 \setminus Q_2)$, or $(P_2 \setminus
%   P_1, Q_2
%   \setminus Q_1)$.\\
%   \noindent
%   The base case, $\Pi_0 = \{(S_i,P_i) \mid i \in [m]\}$, is
%   trivially true since it is the input which is an ICPPA.  Assume
%   the lemma is true till the $j-1$ iteration. Consider $j$th
%   iteration:\\
%   \noindent
%   If $(P,Q)$, $(P',Q')$ and $(P'',Q'')$ are in $\Pi_{j}$ and
%   $\Pi_{j-1}$, all the invariants are
%   clearly true because they are from $j-1$ iteration.\\
%   If $(P,Q)$ is in $\Pi_{j}$ and not in $\Pi_{j-1}$, then it must be
%   one of the new sets added in $\Pi_j$. Since $(P_1,Q_1)$ and
%   $(P_2,Q_2)$ are from $\Pi_{j-1}$ and $Q_1,Q_2$ intersect and have
%   a common leaf, it can be verified that the
%   new sets are also paths. \\
%   By hypothesis for invariant III, invariant II also holds for
%   $(P,Q)$ no matter which new set in $\Pi_j$ it
%   is.\\
%   To prove invariant III, if $(P,Q)$ and $(P',Q')$ are not in
%   $\Pi_{j-1}$, then they are both new sets and invariant III holds
%   trivially (new sets are disjoint). Next consider $(P,Q), (P',Q')
%   \in \Pi_j$ with only one of them, say $(P',Q')$, in
%   $\Pi_{j-1}$. Then $(P,Q)$ is one of the new sets added in line
%   \ref{setbreak}. It is easy to see that if $(P,Q)$ is $(P_1 \cap
%   P_2, Q_1 \cap Q_2)$, then due to invariant IV in hypothesis,
%   invariant III becomes true in this iteration. Similarly, using
%   lemma \ref{lem:setminuscard} invariant III is proven if $(P, Q)$
%   is $(P_1 \setminus P_2, Q_1 \setminus Q_2)$, or $(P_2 \setminus
%   P_1, Q_2
%   \setminus Q_1)$.\\
%   To prove invariant IV, consider three assignments
%   $(P,Q),(P',Q'),(P'',Q'')$. If at least two of these pairs are in
%   not $\Pi_{j-1}$, then they are any two of the new sets. Note that
%   these new sets are disjoint and hence if $(P',Q'), (P'',Q'')$ are
%   any of these sets, $|P \cap P' \cap P''|=|Q \cap Q' \cap Q''|=0$
%   and invariant IV is true. Now we consider the case if at most one
%   of $(P,Q),(P',Q'),(P'',Q'')$ is not in $\Pi_{j-1}$. If none of
%   them are not in $\Pi_{j-1}$ (i.e. all of them are in $\Pi_{j-1}$),
%   invariant IV is clearly true. Consider the case where exactly one
%   of them is not in $\Pi_{j-1}$. w.l.o.g let that be $(P,Q)$ and it
%   could be any one of the new sets. If $(P,Q)$ is $(P_1 \cap P_2,
%   Q_1 \cap Q_2)$, from lemma \ref{lem:fourpaths} and invariant III
%   hypothesis, invariant IV is proven. Similarly if $(P,Q)$ is any of
%   the other new sets, invariant IV is proven by also using lemma
%   \ref{lem:setminuscard}. %\qed

% \end{proof}

% \noindent
% It can be observed that the output of algorithm \ref{al:filterone} is such
% that every leaf is incident on at most a single path in the new set
% of assignments. This is due to the loop condition at line
% \ref{shareleaf}. Let $v_1$ be the leaf incident on path
% $P_i$. Assign to it any one element from $S_i \setminus \bigcup_{i
%   \ne j} S_j$. Remove $(S_i, P_i)$ from assignments and add
% $(\{x_1\}, \{v_1\}), (S_i \setminus \{x_1\}, P_i \setminus
% \{v_1\})$. Now all assignments except single leaf assignments are
% paths from the subtree $T_0 = T \setminus \{v \mid v \text{ is a
%   leaf in } T\}$.


% \begin{algorithm}[h]
%   \caption{Leaf assignments from an ICPPA $\{(S_i,P_i) | i \in I\}$}
%   \label{al:filtertwo}
%   \begin{algorithmic}
%     \STATE Let $\Pi_0=\{(S_i,P_i)| i \in [m]\}$. Paths are such that
%     no
%     two paths $P_i, P_j, i \ne j$ share a leaf.\\
%     \STATE $j = 1$\\
%     \WHILE {there is a leaf $v$ and a unique $(S_{i_1}, P_{i_1})$
%       such that $v \in P_{i_1}$}
%     \STATE $\Pi_j=   \Pi_{j-1} \setminus \{(S_{i_1}, P_{i_1})\}$\\
%     \STATE $X = S_{i_1} \setminus \bigcup_{i \ne i_1, i \in I}S_i$
%     \IF{$X$ is empty} \STATE exit \ENDIF; \STATE Let $x = $
%     arbitrary element from $X$ \vspace{2mm}

%     \STATE $\Pi_j = \Pi_j \cup \{(S_{i_1} \setminus \{x\}),(P_{i_1}
%     \setminus
%     \{v\}), (\{x\},\{v\})\}$\\
%     \STATE $j = j+1$\\
%     \ENDWHILE
%     \STATE $\Pi = \Pi_j$\\
%     \STATE Return $\Pi$\\
%   \end{algorithmic}
% \end{algorithm}

% \begin{lemma}
%   \label{lem:invar3}
%   In Algorithm \ref{al:filtertwo},for all $j \geq 0$, at the end of the
%   $j$th iteration the four invariants given in lemma
%   \ref{lem:invar1} are valid.
%   %   Moreover, $X$ as defined in the algorithm is non-empty if this
%   %   is
%   %   an ICPPA.
% \end{lemma}
% \begin{proof}
%   First we see that $X = S_{i_1} \setminus \bigcup_{i \ne i_1, i \in
%     I}S_i$ is non empty in every iteration for an ICPPA. Suppose $X$
%   is empty. We know that $v \in P_{i_1} \setminus \bigcup_{i \ne
%     i_1, i \in I}P_i$ since $v$ is in the unique path
%   $P_{i_1}$. Since this is an ICPPA $|S_{i_1}| = |P_{i_1}|$. For any
%   $x \in S_{i_1}$ it is contained in at least two sets due to our
%   assumption. Let $S_{i_2}$ be a second set that contains $x$. We
%   know $v \notin P_{i_2}$. Therefore there cannot exist a
%   permutation that maps elements of $S_{i_2}$ to $P_{i_2}$. This
%   contradicts our assumption that this is an ICPPA. Thus $X$ cannot
%   be empty.


%   We use mathematical induction on the number of iterations for this
%   proof. The term ``new sets'' will refer to the sets added in
%   $\Pi_j$ in the $j$th iteration, i.e. $(P' \setminus \{x\},Q'
%   \setminus \{v\})$ and $(\{x\},\{v\})$ for some $(P',Q')$ in
%   $\Pi_{j-1}$ such that $v$ is a leaf and $Q'$ is the unique path
%   incident on it.\\
%   For $\Pi_0$ all invariants hold because it is output from
%   algorithm \ref{al:filterone} which is an ICPPA. Hence base case is
%   proved.  Assume the lemma holds for $\Pi_{j-1}$. Consider $\Pi_j$
%   and any $(P,Q) \in \Pi_j$. If $(P,Q) $ is in $ \Pi_j$ and
%   $\Pi_{j-1}$ invariants I and II are true because of induction
%   assumption. If it is only in $\Pi_j$, then it is $\{(P' \setminus
%   \{x\}),(Q' \setminus \{v\})$ or $(\{x\},\{v\})$ for some $(P',Q')$
%   in $\Pi_{j-1}$. By definition, $x$ is an element in $P'$ (as
%   defined in the algorithm) and $v$ is a leaf in $Q'$. If $(P,Q)$ is
%   $\{(P' \setminus \{x\}),(Q' \setminus \{v\})$, $Q$ is a path since
%   only a leaf is removed from path $Q'$. We know $|P'| = |Q'|$,
%   therefore $|P' \setminus \{x\}| = |Q' \setminus \{v\}|$. Hence in
%   this case invariants I and II are obvious. It is easy to see these
%   invariants hold if $(P,Q)$ is $(\{x\},\{v\})$.


%   For invariant III consider $(P_1,Q_1),(P_2,Q_2)$ in $\Pi_j$. If
%   both of them are also in $\Pi_{j-1}$, claim is proved. If one of
%   them is not in $\Pi_{j-1}$ then it has to be $\{(P' \setminus
%   \{x\}),(Q' \setminus \{v\})$ or $(\{x\},\{v\})$ for some $(P',Q')$
%   in $\Pi_{j-1}$. Since by definition, $Q'$ is the only path with
%   $v$ and $P'$ the only set with $x$ in the previous iteration,
%   $|P_1 \cap (P' \setminus \{x\})| = |P_1 \cap P'|$ and $|Q_1 \cap
%   (Q' \setminus \{v\})| = |Q_1 \cap Q'|$ and $|P_1 \cap \{x\}| = 0,
%   Q_1 \cap \{v\} = 0$. Thus invariant III is also proven.

%   \noindent
%   To prove invariant IV, consider $(P_1,Q_1),(P_2,Q_2), (P_3,Q_3)$
%   in $\Pi_j$. If exactly one of them, say $P_3 \notin \Pi_{j-1}$, it
%   is one of the new sets. By the same argument used to prove
%   invariant III, $|P_1 \cap P_2 \cap (P' \setminus \{x\})| = |P_1
%   \cap P_2 \cap P'|$ and $|Q_1 \cap Q_2 \cap (Q' \setminus \{x\})| =
%   |Q_1 \cap Q_2 \cap Q'|$. Since $P_1, P_2, P'$ are all in
%   $\Pi_{j-1}$, by induction hypothesis $|P_1 \cap P_2 \cap P'| =
%   |Q_1 \cap Q_2 \cap Q'|$. Also $|P_1 \cap P_2 \cap \{x\}| = 0, Q_1
%   \cap Q_2 \cap \{v\} = 0$.  If two or more of them are not in
%   $\Pi_{j-1}$, then it can be verified that $|P_1 \cap P_2 \cap P_3|
%   = |Q_1 \cap Q_2 \cap Q_3|$ since the new sets in $\Pi_j$ are
%   either disjoint or as follows: assuming $P_1, P_2 \notin
%   \Pi_{j-1}$ and new sets are derived from $(P', Q'), (P'', Q'') \in
%   \Pi_{j-1}$ with $x_1, x_2$ exclusively in $P_1, P_2$,
%   $(\{x_1\},\{v_1\}), (\{x_2\},\{v_2\}) \in \Pi_j $ thus $v_1, v_2$
%   are exclusively in $Q_1, Q_2$ resp. it follows that $|P_1 \cap P_2
%   \cap P_3| = |(P' \setminus \{x_1\}) \cap (P'' \setminus \{x_2\})
%   \cap P_3| = |P' \cap P'' \cap P_3| = |Q' \cap Q'' \cap Q_3| = |(Q'
%   \setminus \{v_1\} \cap Q'' \setminus \{v_2\} \cap Q_3| = |Q_1 \cap
%   Q_2 \cap Q_3|$. Thus invariant IV is also proven.  %\qed
% \end{proof}

% Using algorithms \ref{al:filterone} and \ref{al:filtertwo} we prove the
% following theorem.

% \begin{theorem}
%   \label{th:perm}
%   If $\cA$ is an ICPPA, then there exists a bijection $\sigma : U
%   \rightarrow V(T)$ such that $\sigma(S_i) = P_i$ for all $i \in I$
% \end{theorem}
% \begin{proof}
%   This is a contructive proof. First, the given ICPPA $\cA$ and tree
%   $T$ are given as input to Algorithm \ref{al:filterone}. This yields a
%   ``filtered'' ICPPA as the output which is input to Algorithm
%   \ref{al:filtertwo}.  It can be observed that the output of Algorithm
%   \ref{al:filtertwo} is a set of interval assignments to sets and
%   one-to-one assignment of elements of $U$ to each leaf of $T$. To
%   be precise, it would be of the form $\cB_0 = \cA_0 \cup
%   \cL_0$. The leaf assignments are defined in $\cL_0 = \{ (x_i,v_i)
%   \mid x_i \in U, v_i \in T, x_i \ne x_j, v_i \ne v_j, i \ne j, i,j
%   \in [k] \}$ where $k$ is the number of leaves in $T$. The path
%   assignments are defined in $\cA_0 \subseteq \{(S_i',P_i') \mid
%   S_i' \subseteq U_0, P_i' \text{ is a path from } T_0\}$ where
%   $T_0$ is the tree obtained by removing all the leaves in $T$ and
%   $U_0 = U \setminus \{ x \mid x \text{ is assigned to a leaf in
%   }\cL_0 \}$. Now we have a subproblem of finding the permutation
%   for the path assignment $\cA_0$ which has paths from tree $T_0$
%   and sets from universe $U_0$. Now we repeat the procedure and the
%   path assignment $\cA_0$ and tree $T_0$ is given as input to
%   Algorithm \ref{al:filterone}. The output of this algorithm is given to
%   Algorithm \ref{al:filtertwo} to get a new union of path and leaf
%   assignments $\cB_1 = \cA_1 \cup \cL_1$ defined similar to $\cB_0,
%   \cL_0, \cA_0$. In general, the two algorithms are run on path
%   assignment $\cA_{i-1}$ with paths from tree $T_{i-1}$ to get a new
%   subproblem with path assignment $\cA_i$ and tree $T_{i}$. $T_i$ is
%   the subtree of $T_{i-1}$ obtained by removing all its leaves. More
%   importantly, it gives leaf assignments $\cL_{i}$ to the leaves in
%   tree $T_{i-1}$. This is continued until we get a subproblem with
%   path assignment $\cA_{d-1}$ and tree $T_{d-1}$ for some $d \le n$
%   which is just a path. From the last lemma we know that $\cA_{d-1}$
%   is an ICPPA. Another observation is that an ICPPA with all its
%   tree paths being intervals (subpaths from a path) is nothing but
%   an ICPIA\cite{nsnrs09}.  Let $\cA_{d-1}$ be equal to
%   $\{(S_i'',P_i'') \mid S_i'' \subseteq U_{d-1}, P_i'' \text{ is a
%     path from } T_{d-1} \}$. It is true that the paths $P_i''$s may
%   not be precisely an interval in the sense of consecutive integers
%   because they are some nodes from a tree. However, it is easy to
%   see that the nodes of $T_{d-1}$ can be ordered from left to right
%   and ranked to get intervals $I_i$ for every path $P_i''$ as
%   follows. $I_i = \{[l,r] \mid l = \text{ the lowest rank of the
%     nodes in }P_i'', r = l+|P_i''|-1 \}$. Let asssignment $\cA_d$ be
%   with the renamed paths. $\cA_d = \{ (S_i'', I_i) \mid (S_i'',
%   P_i'') \in \cA_{d-1} \}$. What has been effectively done is
%   renaming the nodes in $T_{d-1}$ to get a tree $T_d$.  The ICPIA
%   $\cA_d$ is now in the format that the ICPIA algorithm requires
%   which gives us the permutation $\sigma' : U_{d-1} \rightarrow
%   T_{d-1}$

%   \noindent
%   $\sigma'$ along with all the leaf assignments $\cL_i$ gives us the
%   permutation for the original path assignment $\cA$.  More
%   precisely, the permutation for tree path assignment $\cA$ is
%   defined as follows. $\sigma: U \rightarrow T$ such that the
%   following is maintained.
%   \begin{align*}
%     \sigma(x) &= \sigma'(x),   \text{ if } x \in U_{d-1} \\
%     &= \cL_i(x), \text{ where $x$ is assigned to a leaf in a
%       subproblem $\cA_{i-1}, T_{i-1}$}
%   \end{align*}

%   \noindent
%   To summarize, run algorithm \ref{al:filterone} and \ref{al:filtertwo} on
%   $T$. After the leaves have been assigned to specific elements from
%   $U$, remove all leaves from $T$ to get new tree $T_0$. The leaf
%   assignments are in $\cL_0$. Since only leaves were removed $T_0$
%   is indeed a tree. Repeat the algorithms on $T_0$ to get leaf
%   assignments $\cL_{1}$. Remove the leaves in $T_0$ to get $T_1$ and
%   so on until the pruned tree $T_d$ is a single path. Now run ICPIA
%   algorithm on $T_d$ to get permutation $\sigma'$. The relation
%   $\cL_0 \cup \cL_1 \cup .. \cup \cL_{d} \cup \sigma'$ gives the
%   bijection required in the original problem.%\qed
% \end{proof}


