\documentclass[synopsis,MS]             % MS thesis dissertation.
              {iitmdiss}

\usepackage{times}             % Font?
\usepackage{t1enc}             % TI encoding??  
                            %%% Graphics related.                 %

\usepackage{../lib/mylatexlib}


\begin{document}

%%%%%%%%%%%%%%%%%%%%%%%%%%%%%%%%%%%%%%%%%%%%%%%%%%%%%%%%%%%%%%%%%%%%%%
% Title page
\title{\mythesistitle}

\author{\myname}

\date{\MakeUppercase {\mysubmissionmonth}}
\department{\MakeUppercase {\mydept}}

%\nocite{*}
\maketitle

%%%%%%%%%%%%%%%%%%%%%%%%%%%%%%%%%%%%%%%%%%%%%%%%%%%%%%%%%%%%%%%%%%%%%%
% Certificate
\certificate

\vspace*{0.5in}

\noindent 
This is to certify that the thesis titled {\bf \mythesistitle},
submitted by {\bf \myname}, to the Indian Institute of Technology,
Madras, for the award of the degree of {\bf \mydegree}, is
a bona fide record of the research work done by her under our
supervision.  The contents of this thesis, in full or in parts, have
not been submitted to any other Institute or University for the award
of any degree or diploma.

\vspace*{1.5in}

\begin{singlespacing}
  \hspace*{-0.25in}
  \parbox{2.5in}{
    \noindent {\bf Prof.~\myadvisor} \\
    \noindent Research Guide \\
    \noindent Associate Professor \\
    \noindent Dept. of \mydept\\
    \noindent IIT Madras -- 600 036} \hfill
  \parbox{1.25in}{               % {1.75in}
    \vspace*{0.75in} 
    \noindent Chennai\\          %{\em Place:}
    \noindent \mysubmissiondate} %{\em Date:}
\end{singlespacing}




%%%%%%%%%%%%%%%%%%%%%%%%%%%%%%%%%%%%%%%%%%%%%%%%%%%%%%%%%%%%%%%%%%%%%%
% Acknowledgements
\acknowledgements

\tnote[a]{work in progress.}

%%%%%%%%%%%%%%%%%%%%%%%%%%%%%%%%%%%%%%%%%%%%%%%%%%%%%%%%%%%%%%%%%%%%%%
% Abstract

\begin{abstract}
  \noindent {\bf Keywords}: \hspace*{0.5em} \parbox[t]{4.4in}{consecutive
    ones property, algorithmic graph theory, hypergraph isomorphism,
    interval labeling} \vspace*{24pt}

\def \tem {}
  \noindent 
  \annote[a]{In this paper, we explore a natural generalization of
  results on binary matrices with the |\tem consecutive ones property|. 
  We consider the following constraint satisfaction problem. Given (i)
  a set system $\F \subseteq$ $(2^{U} \setminus \emptyset)$ of a
  finite set $U$ of cardinality $n$, (ii) a tree $T$ of size $n$ and
  (iii) a bijection called |\tem tree path labeling|, $\cl$ mapping the
  sets in $\cF$ to paths in $T$, does there exist at least one
  bijection $\phi:U \rightarrow V(T)$ such that for each $S \in \cF$,
  $\{\phi(x) \mid x \in S\} = \cl(S)$?  A tree path labeling of a set
  system is called |\tem feasible| if there exists such a bijection
  $\phi$.  We present an algorithmic characterization of feasible tree
  path labeling. COP is a special instance of tree path labeling
  problem when $T$ is a path.  We conclude with a polynomial time
  algorithm to find a feasible tree path labeling of a given set
  system when $T$ is a |\tem $k$-subdivided star|, set system has a
  single containment tree of overlap components and set size is
  limited to at most $k+2$. }{this is the abstract of the
  paper. expand to fit thesis.}
\end{abstract}


%\pagebreak

%%%%%%%%%%%%%%%%%%%%%%%%%%%%%%%%%%%%%%%%%%%%%%%%%%%%%%%%%%%%%%%%%
% Table of contents etc.

\begin{singlespace}
\tableofcontents
\thispagestyle{empty}

\listoftables
\addcontentsline{toc}{chapter}{LIST OF TABLES}
\listoffigures
\addcontentsline{toc}{chapter}{LIST OF FIGURES}
\end{singlespace}


%%%%%%%%%%%%%%%%%%%%%%%%%%%%%%%%%%%%%%%%%%%%%%%%%%%%%%%%%%%%%%%%%%%%%%
% Abbreviations
\abbreviations

\noindent 
\begin{tabbing}
xxxxxxxxxxx \= xxxxxxxxxxxxxxxxxxxxxxxxxxxxxxxxxxxxxxxxxxxxxxxx \kill
\textbf{COP}   \> Consecutive Ones Property \\
\textbf{COT}   \> Consecutive Ones Testing \\
\textbf{ICPIA}   \> Intersection Cardinality Preservation Interval Assignment \\
\textbf{ICPPL}   \> Intersection Cardinality Preserved Path Labeling \\

\end{tabbing}

\pagebreak

%%%%%%%%%%%%%%%%%%%%%%%%%%%%%%%%%%%%%%%%%%%%%%%%%%%%%%%%%%%%%%%%%%%%%%
% Notation

\chapter*{\centerline{NOTATION}}
\addcontentsline{toc}{chapter}{NOTATION}

\begin{singlespace}
\begin{tabbing}
xxxxxxxxxxx \= xxxxxxxxxxxxxxxxxxxxxxxxxxxxxxxxxxxxxxxxxxxxxxxx \kill
\textbf{$2^{U}$}  \> Powerset of set $U$ \\
\end{tabbing}
\end{singlespace}

\pagebreak
\clearpage

% The main text will follow from this point so set the page numbering
% to arabic from here on.
\pagenumbering{arabic}


%%%%%%%%%%%%%%%%%%%%%%%%%%%%%%%%%%%%%%%%%%%%%%%%%%
% Introduction.


\chapter{Introduction}
\label{chap:intro}
\tnote[a]{ Some things to remember - have this somewhere.

  \begin{itemize}
  \item Examples: Illustration of the problem examples. both COP and
    extension
  \item Basic preliminaries: (definitions theorems etc) needed if any
    - graph theory
  \item Organization: Outline of document
  \end{itemize}
  
  Abstract must be a brief about what results we have and how it fits
  in the body of research.

  \begin{enumerate}
  \item \emph{Area:} - Broad to Specialized. i.e. Combinatorial
    algorithms -> Matrix reorganization -> general data reorganization
    (interval assignment) -> path assignment
  \item \emph{Class of problems:} say, data reorganization.
  \item \emph{Nature of results:} Is a generalization. We have a
    Polynomial algorithm for a subset of the generalization.
  \end{enumerate}

Introduction is a magnified version of Abstract with some brief
citations etc. mini survey. Here talk about canonization how it
connects these problems to graph isomorphism. Introduce our problem,
explain why it is important etc. How canonization is a
posisbility. Brief about results obtained.

Someone who doesn't want to go through the whole thesis must get the
gist of the whole document from the introduction chapter. that should
be the outlook. 

once this is done, the rest of the chapters will fall into place.

{\em Survey chapter} is the full fledged expansion of the survey in
introduction with details, observations, theorems etc.

  {\em The main themes} of the thesis that must not be left out:
  \begin{enumerate}
  \item connection of COP to {\bf set systems} and thus to general
    data bases. \cite{nsnrs09}
  \item {\bf graph isomorphism} by canonization \cite{kklv10}
  \item {\bf certificate for a problem} \cite{mcc04}
  \end{enumerate}

Also see purple book notes.
}

\section{to area}
\section{to problem}
\section{to your thesis}
\section{Organization of document}


   
\chapter{Consecutive Ones property}
\tnote[a]{survey of cop includes mainly 
\begin{enumerate}
\item \cite{d08phd} (perhaps minus chap 4?)
\item \cite{nsnrs09}
\item \cite{mcc04}
\item \cite{kklv10}
\end{enumerate}
.. to name a few. 

sections will be as follows:}
\section{Characterization of COP}

\tnote[a]{\cite{at72}, \cite{mcc04}, prime submatrix \cite{wlh02}?}
\section{Recognition of COP}
\subsection{\tnote[a]{Polynomial time algorithms}}
\tnote[a]{ (pq tree, pqr, icpia)}
\subsection{\tnote[a]{Tucker's submatrices, Dom's algo to find them}}
\subsection{\tnote[a]{McConnell's certificate}}


\section{Alteration to matrices to get COP}
\tnote[a]{algorithms to make a matrix into one with COP
  (min/max-cos-r/c) \cite{d08phd} }
\section{brief analysis of hard problems with input having COP (chap
  5, 6 in dom)}
\label{sec:briefanal}
\tnote[a]{Do I need this?}
\section{Complexity of certain COP variations}
\tnote[a]{maybe combine with Section~\ref{sec:briefanal}\\
complexity - poly, logspace}


\chapter{Other problems related to COP }
\tnote[a]{(see literature)}





\chapter{Research - Tree path labeling of path hypergraphs}
\tnote[a]{Extension to trees. ( stuff in latin paper,)}

\tnote[a]{UPDATE TO FINAL VERSION. BELOW IS OLDER VERSION WITH
  INCOMPLETE SEC 4.}
\section{Introduction}
Consecutive ones property (COP) of binary matrices is a widely studied
combinatorial problem. The problem is to rearrange rows (columns) of a
binary matrix in such a way that every column (row) has its $1$s occur
consecutively. If this is possible the matrix is said to have the COP.
This problem has several practical applications in diverse fields
including scheduling \cite{hl06}, information retrieval \cite{k77} and
computational biology \cite{abh98}.  Further, it is a tool in graph
theory \cite{mcg04} for interval graph recognition, characterization
of hamiltonian graphs, and in integer linear programming
\cite{ht02,hl06}.  Recognition of COP is polynomial time solvable by
several algorithms. PQ trees \cite{bl76}, variations of PQ trees
\cite{mm09,wlh01,wlh02,mcc04}, ICPIA \cite{nsnrs09} are the main ones.

%\noindent
The problem of COP testing can be easily seen as a simple constraint
satisfaction problem involving a system of sets from a universe. Every
column of the binary matrix can be converted into a set of integers
which are the indices of the rows with $1$s in that column. When
observed in this context, if the matrix has the COP, a reordering of
its rows will result in sets that have only consecutive integers. In
other words, the sets after reordering are intervals. Indeed the
problem now becomes finding interval assignments to the given set
system \cite{nsnrs09} with a single permutation of the universe (set
of row indices) which permutes each set to its interval. The result in
\cite{nsnrs09} characterizes interval assignments to the sets which
can be obtained from a single permutation of the rows.  They show that
for each set, the cardinality of the interval assigned to it must be
same as the cardinality of the set, and the intersection cardinality
of any two sets must be same as the interesction cardinality of the
corresponding intervals.  While this is naturally a necessary
condition, \cite{nsnrs09} shows this is indeed sufficient.  Such an
interval assignment is called an Intersection Cardinality Preserving
Interval Assignment (ICPIA).  Finally, the idea of decomposing a given
0-1 matrix into prime matrices to check for COP is adopted from
\cite{wlh02} to test if an ICPIA exists for a given set system.

\noindent 
{\bf Our Work.}  A natural generalization of the interval
assignment problem is feasible tree path labeling problem of a set
system. The problem is defined as follows -- given a set system $\cF$
from a universe $U$ and a tree $T$, does there exist a bijection from
$U$ to the vertices of $T$ such that each set in the system maps to a
path in $T$.  We refer to this as the {\em tree path labeling problem}
for an input set system, target tree pair -- $(\cF,T)$. As a special
case if the tree $T$ is a path, the problem becomes the interval
assignment problem.  We focus on the question of generalizing the
notion of an ICPIA \cite{nsnrs09} to characterize feasible path
assignments.  We show that for a given set system $\cF$, a tree $T$,
and an assignment of paths from $T$ to the sets, there is a feasible
bijection between $U$ and $V(T)$ if and only if all intersection
cardinalities among any three sets (not necessarily distinct) is same
as the intersection cardinality of the paths assigned to them and the
input runs a filtering algorithm (described in this paper)
successfully.  This characterization is proved constructively and it
gives a natural data structure that stores all the relevant feasible
bijections between $U$ and $V(T)$.  Further, the filtering algorithm
is also an efficient algorithm to test if a tree path labeling to the
set system is feasible.  This generalizes the result in
\cite{nsnrs09}.

%\noindent 
\tnote[a]{the following text could be in Further Study/Conclusion}It
is an interesting fact that for a matrix with the COP, the
intersection graph of the corresponding set system is an interval
graph.  A similar connection to a subclass of chordal graphs and a
superclass of interval graphs exists for the generalization of COP.
In this case, the intersection graph of the corresponding set system
must be a {\em path graph}. Chordal graphs are of great significance,
extensively studied, and have several applications.  One of the well
known and interesting properties of a chordal graphs is its connection
with intersection graphs \cite{mcg04}. For every chordal graph, there
exists a tree and a family of subtrees of this tree such that the
intersection graph of this family is isomorphic to the chordal graph
\cite{plr70,gav78,bp93}.  These trees when in a certain format, are
called {\em clique trees} \cite{apy92} of the chordal graph. This is a
compact representation of the chordal graph. There has also been work
done on the generalization of clique trees to clique hypergraphs
\cite{km02}.  If the chordal graph can be represented as the
intersection graph of paths in a tree, then the graph is called path
graph \cite{mcg04}.  Therefore, it is clear that if there is a
bijection from $U$ to $V(T)$ such that for every set, the elements in
it map to vertices of a unique path in $T$, then the intersection
graph of the set system is indeed a path graph.  However, this is only
a necessary condition and can be checked efficiently because path
graph recognition is polynomial time
solvable\cite{gav78,aas93}. Indeed, it is possible to construct a set
system and tree, such that the intersection graph is a path graph, but
there is no bijection between $U$ and $V(T)$ such that the sets map to
paths. Path graph isomorphism is known be isomorphism-complete, see
for example \cite{kklv10}. An interesting area of research would be to
see what this result tells us about the complexity of the tree path
labeling problem (not covered in this paper).

%\noindent
In the later part of this paper, we focus on a new special case of the
tree path labeling problem. Here the set system is such that every set
is at most $k+2$ in size and for every pair of sets in it there exists
a sequence of sets between them with consecutive sets in this sequence
having a strict intersection -- i.e., non-empty intersection with
neither being contained in the other. Moreover, the given tree is a
{\em $k$-subdivided star}. We demonstrate a polynomial time algorithm
to find a feasible path labeling in this case.


% In the later part of this paper, we decompose our search for a
% bijection between $U$ and $V(T)$ into subproblems.  Each subproblem is
% on a set subsystem in which for each set, there is another set in the
% set subsystem with which the intersection is {\em strict}, i.e., there
% is a non-empty intersection, but neither is contained in the other.
% This is in the spirit of results in \cite{wlh02,nsnrs09} where to test
% for the COP in a given matrix, the COP problem is solved on an
% equivalent set of prime matrices.

\noindent {\bf Roadmap.} The necessary preliminaries with definitions
etc. are presented in
Section~\ref{sec:prelims}. Section~\ref{sec:feasible} documents the
characterization of a feasible path labeling and finally,
Section~\ref{sec:ksubdivstar} describes a polynomial time algorithm to
find the tree path labeling of a given set system from a given
$k$-subdivided tree.

\section{Preliminaries} 
\label{sec:prelims} 

This section states definitions and basic facts necessary in the scope
of this document.

\def\xnoindent {}

The set $\F \subseteq (2^{U} \setminus \emptyset)$ is a {\em set
  system} of a universe $U$ with $|U| = n$.  The {\em support} of a
set system $\F$ denoted by $supp(\cF)$ is the union of all the sets in
$\F$; $supp(\F) = \bigcup_{S \in \F}S$. For the purposes of this
paper, a set system is required to ``cover'' the universe; $
supp(\cF) = U$.
%\vspace{\prelimspace}

\xnoindent The graph $T$ represents a {\em target tree} with same
number of vertices as elements in $U$; $|V(T)| = n$.  A {\em path
  system}\, $\cP$ is a set system of paths from $T$; $\cP \subseteq \{P
\mid P \subseteq V, \text{ } T[P] \text{ is a path} \}$.
%\vspace{\prelimspace}

% \xnoindent
% A set system $\cF$ can be alternatively represented by a {\em
%   hypergraph} $\H_\cF$ whose vertex set is $supp(\cF)$ and hyperedges
% are the sets in $\cF$. This is a known representation for interval
% systems in literature \cite{bls99,kklv10}.  We extend this definition here to
% path systems.  
\xnoindent
A set system $\cF$ can be alternatively represented by a {\em
  hypergraph}\, $\cF_H$ whose vertex set is $supp(\cF)$ and hyperedges
are the sets in $\cF$. This is a known representation for interval
systems in literature \cite{bls99,kklv10}.  We extend this definition here to
path systems. Due to the equivalence of set system and hypergraph in
the scope of this paper, we drop the
subscript $_H$ in the notation and refer to both the structures by $\cF$.
%\vspace{\prelimspace}


\xnoindent Two hypergraphs $\cF'$, $\cF''$ are said to be {\em
  isomorphic} to each other, denoted by $\cF' \cong \cF''$, iff there
exists a bijection $\phi: supp(\cF') \rightarrow supp(\cF'')$ such
that for all sets $A \subseteq supp(\cF')$, $A$ is a hyperedge in
$\cF'$ iff $B$ is a hyperedge in $\cF''$ where $B = \{\phi(x) \mid x
\in A\}$ \cite{kklv10}. This is called {\em hypergraph
  isomorphism}. \tnote[a]{also extend $\phi$ to hyperedges -- see if
  required }
%\vspace{\prelimspace}


\xnoindent The {\em intersection graph}\, $\bI(\cF)$ of a hypergraph
$\cF$ is a graph such that its vertex set has a bijection to $\cF$ and
there exists an edge between two vertices iff their corresponding
hyperedges have a non-empty intersection \cite{mcg04}.
%\vspace{\prelimspace}


%\noindent 
If the intersection graphs of two hypergraphs are isomorphic, $\bI(\cF)
\cong \bI(\cP)$ where $\cP$ is also a path system, then the bijection
$\cl: \cF \rightarrow \cP$ due to this isomorphism is called a {\em
  path labeling} of the hypergraph $\cF$. To illustrate further, let
$\cg: V(\cF) \rightarrow V(\cP)$ be the above mentioned isomorphism
where $V(\cF)$ and $V(\cP)$ are the vertex sets that represent the
hyperedges for each hypergraph respectively, $V(\cF) = \{ v_S \mid S
\in \cF\}$ and $V(\cP) = \{ v_P \mid P \in \cP\}$. Then the path
labeling $\cl$ is defined as follows: $\cl(S_1) = P_1$ iff $\cg
(v_{S_1}) = v_{P_1}$. The path system $\cP$ may be alternatively
denoted in terms of $\cF$ and $\cl$ as $\cF^\cl$. In most scenarios in
this paper, what is given are the pair $(\cF, \cl)$ and the target
tree $T$; hence this notation will be used more often.

\xnoindent If $\cF \cong \cP$ where $\cP$ is a path system, then $\cF$
is called a {\em path hypergraph} and $\cP$ is called {\em path
  representation} of $\cF$. If this isomorphism is $\phi: supp(\cF)
\rightarrow V(T)$, then it is clear that there is an {\em induced path
labeling} $\cl_\phi: \cF \rightarrow \cP$ to the set system;
$\cl_\phi(S) = \set{y \mid y = \phi(x), x \in S}$ for all $S \in \cF$. Recall that
$supp(\cP) = V(T)$.
%\vspace{\prelimspace}


% A graph $G$ is a {\em path graph} if it is isomorphic to the intersection graph
% $\bI(\cP)$ of a path system $\cP$. 
% This
% isomorphism gives a bijection $\cl': V(G) \rightarrow \cP$. Moreover,
% for the purposes of this paper, we require that in a path labeling,
% $supp(\cP) = V(T)$.  If graph $G$ is also isomorphic to $\bI(\cF)$ for some hypergraph
% $\cF$, then clearly there is a bijection $\cl: \cF \rightarrow \cP$
% such that $\cl(S) = \cl'(v_S)$ where $v_S$ is the vertex corresponding
% to set $S$ in $\bI(\cF)$ for any $S \in \cF$. This bijection $\cl$ is
% called the {\em path labeling} of the hypergraph $\cF$ and the path
% system $\cP$ may be alternatively denoted as $\cF^\cl$.
%\vspace{\prelimspace}

\xnoindent
A path labeling $(\cF, \cl)$ is defined to be {\em
  feasible} if
$\cF \cong \cF^\cl$ and this hypergraph isomorphism $\phi: supp(\cF)
\rightarrow supp(\cF^\cl)$ induces a path labeling $\cl_\phi: \cF
\rightarrow \cF^\cl$ such that $\cl_\phi = \cl$. 
%\vspace{\prelimspace}

\xnoindent An {\em overlap graph}\, $\bO(\cF)$ of a hypergraph $\cF$ is
a graph such that its vertex set has a bijection to $\cF$ and there
exists an edge between two of its vertices iff their corresponding
hyperedges overlap. Two hyperedges $S$ and $S'$ are said to {\em
  overlap}, denoted by $S \overlap S'$, if they have a non-empty
intersection and neither is contained in the other; $S \overlap S'
\text{ iff } S \cap S' \ne \emptyset, S \nsubseteq S', S' \nsubseteq
S$. Thus $\bO(\cF)$ is a spanning subgraph of $\bI(\cF)$ and not
necessarily connected. Each connected component of $\bO(\cF)$ is
called an {\em overlap component}.
% If there are $d$ overlap components
% in $\bO(\cF)$, the set subsystems are denoted by $\cO_1, \cO_2, \ldots
% \cO_d$. Clearly $\cO_i \subseteq \F, i \in [d]$. For any $i, j \in [d]$,
% it can be verified that one of the following is true.
% \begin{enumerate}[a) ]
% \item $supp(\cO_i)$ and $supp(\cO_j)$ are disjoint
% \item $supp(\cO_i)$ is a subset of a set in $\cO_j$
% \item $supp(\cO_j)$ is a subset of a set in $\cO_i$
% \end{enumerate}
%\vspace{\prelimspace}

\xnoindent
A hyperedge $S \in \cF$ is called {\em marginal} if for all $S'
\overlap S$, the overlaps $S \cap S'$ form a single inclusion chain
\cite{kklv10}. Additionally, if $S$ is such that it is contained in no
other hyperedge in $\cF$, i.e., it is inclusion maximal then it is called
{\em super-marginal}.

\xnoindent A {\em star} graph is a complete bipartite graph $K_{1,p}$
which is clearly a tree and $p$ is the number of leaves. The vertex
with maximum degree is called the {\em center} of the star and the
edges are called {\em rays} of the star. A {\em $k$-subdivided star}
is a star with all its rays subdivided exactly $k$ times. The
definition of a {\em ray of a $k$-subdivided star} is extended to the path
from the center to a leaf. It is clear that all rays are of length $k+2$.


\section{Characterization of Feasible Tree Path  Labeling} 
\label{sec:feasible} 

In this section we give an algorithmic characterization of a
feasibility of tree path labeling.
% Consider a path labeling $\cl: \cF \rightarrow \cP$ for set system $\cF$
% and path system $\cP$ on the given tree $T$. We
% call $\cl$ an {\em Intersection Cardinality Preserving Path Labeling
%   (ICPPL)} if it has the following properties.
Consider a path labeling $(\cF, \cl)$ on the
given tree $T$. We call $(\cF, \cl)$ an {\em Intersection Cardinality
  Preserving Path Labeling (ICPPL)} if it has the following
properties.

%\vspace{\topshrink}

\begin{enumerate}[{(\icpplpr\ }i) \ \ \ ]
\item \label{pr:i} $|S| = |\cl(S)|$ \ \ \ \ \ \ \ \ \ \ \ \ \ \ \ \ \ \ \ \ \ \ \ \
  \ \ \ \ \ \ \ \ \ \ \ \ \ \ \ for all $S \in \cF$
%  \vspace{\topshrink}
\item \label{pr:ii}$|S_1 \cap S_2| = |\cl(S_1) \cap \cl(S_2)|$ \ \ \ \ \ \ \ \ \ \
  \ \ \ \ \ \ \ \ \ \ for all distinct
  $S_1, S_2 \in \cF$
%  \vspace{\topshrink}
\item \label{pr:iii}$|S_1 \cap S_2 \cap S_3| = |\cl(S_1) \cap \cl(S_2) \cap
  \cl(S_3)|$ \ \ \ for all distinct $S_1, S_2, S_3 \in
  \cF$
\end{enumerate}


\noindent
The following lemma is useful in subsequent arguments. 
\begin{lemma}
  \label{lem:setminuscard}
  If $\cl$ is an ICPPL, and $S_1, S_2, S_3 \in \cF$, then $|S_1 \cap
  (S_2 \setminus S_3)| = |\cl(S_1) \cap (\cl(S_2) \setminus
  \cl(S_3))|$.
\end{lemma}
\begin{proof}%[Proof of Lemma~\ref{lem:setminuscard}]
  Let $P_i = \cl(S_i)$, for all $1 \le i \le  3$.
  $|S_1 \cap (S_2 \setminus S_3)| = |(S_1 \cap S_2) \setminus S_3| =
  |S_1 \cap S_2| - |S_1 \cap S_2 \cap S_3|$. Due to properties (ii)
  and (iii) of ICPPL, $|S_1 \cap S_2| - |S_1 \cap S_2 \cap S_3| = |P_1
  \cap P_2| - |P_1 \cap P_2 \cap P_3| = |(P_1 \cap P_2) \setminus P_3|
  = |P_1 \cap (P_2 \setminus P_3)|$. Thus lemma is proven. \qed
\end{proof}

\def \xnoindent  {}

\xnoindent In the remaining part of this section we show that $(\cF,
\cl)$ is feasible if and only if it is an ICPPL and
Algorithm~\ref{al:icppl-find-isomorph} returns a non-empty
function. Algorithm~\ref{al:icppl-find-isomorph} recursively does two levels of
filtering of $(\cF, \cl)$ to make it simpler while retaining the set
of isomorphisms, if any, between $\cF$ and $\cF^\cl$.
% One direction of this claim isclear: that if a path
% labeling is feasible, then all intersection cardinalities are
% preserved, i.e. the path labeling is an ICPPL. Algorithm~\ref{perms}
% \annote[a]{has no premature exit condition hence any input will go
%   through it}{Prove that the filtered sets has ICPPL iff input PL
%   has ICPPL?}. Algorithm~\ref{leafasgn} has an exit condition at
% line~\ref{xempty}. It can be easily verified that $X$ cannot be
% empty if $\cl$ is a feasible path labeling. The reason is that a
% feasible path labeling has an associated bijection between
% $supp(\cF)$ and $V(T)$ \remove[a]{i.e. $supp(\cF^{\cl})$} such that
% the sets map to paths, ``preserving'' the path labeling.  The rest
% of the section is devoted to constructively proving that it is
% sufficient for a path labeling to be an ICPPL and pass the two
% filtering algorithms.  To describe in brief, the constructive
% approaches refine an ICPPL iteratively, such that at the end of each
% iteration we have a ``filtered'' path labeling, and finally we have
% a path labeling that defines a family of bijections from $supp(\cF)$
% to $V(T)$\remove[a]{ i.e. $supp(\cF^{\cl})$}.
First, we present Algorithm~\ref{perms} or {\tt filter\_1}, and prove
its correctness.  This algorithm refines the path labeling by
processing pairs of paths in $\cF^\cl$ that share a leaf until no two
paths in the new path labeling share any leaf.

\begin{algorithm}[h]
  \caption{Refine ICPPL {\tt filter\_1($\cF, \cl, T$)}}
  \label{perms}
  \begin{algorithmic}[\lndisplay]
    \STATE $\cF_0 \assign \cF$, $\cl_0(S) \assign \cl(S)$ for all $S \in \cF_0$\\
    \STATE $j \assign 1$\\
    \WHILE {there is $S_1, S_2 \in \cF_{j-1}$ such that
      $\cl_{j-1}(S_1)$ and $\cl_{j-1}(S_2)$ have a common leaf in
      $T$}\label{shareleaf} \STATE $\cF_j \assign (\cF_{j-1} \setminus
    \{S_1, S_2\})
    \cup \{S_1 \cap S_2, S_1 \setminus S_2, S_2 \setminus S_1 \}$ \label{setbreak} 
    \COMMENT {Remove $S_1$, $S_2$ and add the ``filtered'' sets}
    \STATE {\bf for} every $S \in \cF_{j-1}$ s.t. $S \ne S_1$ and $S \ne
    S_2$ {\bf do} $\cl_j(S) \assign \cl_{j-1}(S)$ {\bf end for}\\

    \STATE $\cl_j(S_1 \cap S_2) \assign \cl_{j-1}(S_1) \cap
    \cl_{j-1}(S_2)$
    \COMMENT {Carry forward the path labeling for all existing sets other than
      $S_1$, $S_2$}
    \STATE $\cl_j(S_1 \setminus S_2) \assign \cl_{j-1}(S_1) \setminus
    \cl_{j-1}(S_2)$ 
    \COMMENT {Define path labeling for new sets}
    \STATE $\cl_j(S_2 \setminus S_1) \assign \cl_{j-1}(S_2) \setminus
    \cl_{j-1}(S_1)$

    \IF{$(\cF_j, \cl_j)$ does not satisfy (\icpplpr~\ref{pr:iii}) of ICPPL}
    \label{ln:3waycheck}
    \STATE {\bf exit} \label{ln:exit1} \\
    \ENDIF

    \STATE $j \assign j+1$\\
    \ENDWHILE
    \STATE $\cF' \assign \cF_j$, $\cl' \assign \cl_j$\\
    \RETURN $(\cF', \cl')$
  \end{algorithmic}
\end{algorithm}

\begin{lemma} 
 \label{lem:feasible} 
 In Algorithm~\ref{perms}, if input $(\cF, \cl)$ is a feasible path
 assignment then at the end of $j$th iteration of the {\bf while}
 loop, $j \ge 0$, $(\cF_j, \cl_j)$ is a feasible path assignment.
\end{lemma}
\begin{proof}%[Proof of Lemma~\ref{lem:feasible}]
  We will prove this by mathematical induction on the number of
  iterations. The base case $(\cF_0, \cl_0)$ is feasible since it is
  the input itself which is given to be feasible. Assume the lemma is
  true till $j-1$th iteration. i.e. every hypergraph isomorphism
  $\phi: supp\left(\cF_{j-1}\right) \rightarrow V\left(T \right)$ that
  defines $(\cF, \cl)$'s feasibility, is such that the induced path
  labeling on $\cF_{j-1}$, $\cl_{\phi[{\cF_{j-1}}]}$ is equal to
  $\cl_{j-1}$. We will prove that $\phi$ is also the bijection that
  makes $(\cF_j, \cl_j)$ feasible. Note that $supp(\cF_{j-1}) =
  supp(\cF_{j})$ since the new sets in $\cF_j$ are created from basic
  set operations to the sets in $\cF_{j-1}$. For the same reason and
  $\phi$ being a bijection, it is clear that when applying the $\phi$
  induced path labeling on $\cF_j$, $ \cl_{\phi[{\cF_{j}}]}(S_1
  \setminus S_2) = \cl_{\phi[{\cF_{j-1}}]}(S_1) \setminus
  \cl_{\phi[{\cF_{j-1}}]}(S_2)$. Now observe that $ \cl_j(S_1
  \setminus S_2) = \cl_{j-1}(S_1) \setminus \cl_{j-1}(S_2) =
  \cl_{\phi[{\cF_{j-1}}]}(S_1) \setminus
  \cl_{\phi[{\cF_{j-1}}]}(S_2)$. Thus the induced path labeling
  $\cl_{\phi[{\cF_{j}}]} = \cl_{j}$. Therefore lemma is proven.  \qed
\end{proof}

\begin{lemma}
  \label{lem:invar1} In Algorithm~\ref{perms}, at the end of $j$th
  iteration, $j \ge 0$, of the {\bf while} loop, the following
  invariants are maintained.
  \begin{enumerate}[I {\ }] %\vspace{\topshrink}
  \item $\cl_j(R)$ is a path in $T$, \ \ \ \ \ \ \ \ \ \ \ \ \ \ \ \ \
    \ \ \ \ \ \ \ \ \ \ for all $R \in \cF_j$%\vspace{\topshrink}
  \item $|R| = |\cl_j(R)|$, \ \ \ \ \ \ \ \ \ \ \ \ \ \ \ \ \ \ \ \ \
    \ \ \ \ \ \ \ \ \ \ \ \ \ \ \ for all $R \in
    \cF_j$%\vspace{\topshrink}
  \item $|R \cap R'| = |\cl_j(R) \cap \cl_j(R')|$, \ \ \ \ \ \ \ \ \ \
    \ \ \ \ \ \ \ \ \ \ for all $R, R' \in \cF_j$%\vspace{\topshrink}
  \item $|R \cap R' \cap R''|=|\cl_j(R) \cap \cl_j(R') \cap
    \cl_j(R'')|$, \ \ \ for all $R, R', R'' \in \cF_j$
  \end{enumerate}
\end{lemma}

\begin{proof}
 Proof is by induction on the number of iterations, $j$. In this
  proof, the term ``new sets'' will refer to the sets added to $\cF_j$
  in $j$th iteration in line~\ref{setbreak} of Algorithm~\ref{perms},
  $S_1 \cap S_2, S_1 \setminus S_2, S_2 \setminus S_1$ and its
  images in $\cl_j$ where $\cl_{j-1}(S_1)$
  and $\cl_{j-1}(S_2)$ intersect and share a leaf.\\
  The invariants are true in the base case $(\cF_0, \cl_0)$, since it
  is the input ICPPL.  Assume the lemma is true till the $j-1$th
  iteration. Let us consider the possible cases for each of the above invariants for
  the $j$th iteration.

  \xnoindent
 \begin{enumerate}[\xbullet]
  \item {\em Invariant} I/II
    \begin{enumerate}[{I/II}a $|$] % \textbullet 
    \item {\em $R$ is not a new set.} It is in $\cF_{j-1}$. Thus
      trivially true by induction hypothesis.
    \item {\em $R$ is a new set.} If $R$ is in $\cF_{j}$ and not in
      $\cF_{j-1}$, then it must be one of the new sets added in
      $\cF_j$. In this case, it is clear that for each new set, the
      image under $\cl_j$ is a path since by definition the chosen
      sets $S_1$, $S_2$ are from $\cF_{j-1}$ and due to the while loop
      condition, $\cl_{j-1}(S_1)$, $\cl_{j-1}(S_2)$ have a
      common leaf. Thus invariant I is proven.\\
      Moreover, due to induction hypothesis of invariant III and the
      definition of $l_j$ in terms of $l_{j-1}$, invariant II is
      indeed true in the $j$th iteration for any of the new sets.  If
      $R = S_1 \cap S_2$, $|R| = |S_1 \cap S_2| = |\cl_{j-1}(S_1) \cap
      \cl_{j-1}(S_2)| = |\cl_j(S_1 \cap S_2)| = |\cl_j(R)|$.
      If $R = S_1 \setminus S_2$, $|R| = |S_1 \setminus S_2| = |S_1| -
      |S_1 \cap S_2| = |\cl_{j-1}(S_1)| - |\cl_{j-1}(S_1) \cap
      \cl_{j-1}(S_2)| = |\cl_{j-1}(S_1) \setminus \cl_{j-1}(S_2)| =
      |\cl_j(S_1 \setminus S_2)|
      = |\cl_j(R)|$. Similarly if $R = S_2 \setminus S_1$.\\
    \end{enumerate}
  \item {\em Invariant} III
    \begin{enumerate}[{III}a $|$]
    \item {\em $R$ and $R'$ are not new sets.} It is in
      $\cF_{j-1}$. Thus trivially true by induction hypothesis.
    \item {\em Only one, say $R$, is a new set.} Due to invariant IV
      induction hypothesis, Lemma~\ref{lem:setminuscard} and
      definition of $\cl_j$, it follows that invariant III is true no
      matter which of the new sets $R$ is equal to. If $R = S_1 \cap
      S_2$, $|R \cap R'| = |S_1 \cap S_2 \cap R'| = |\cl_{j-1}(S_1)
      \cap \cl_{j-1}(S_2) \cap \cl_{j-1}(R')| = |\cl_j(S_1 \cap S_2)
      \cap \cl_j(R')| = |\cl_j(R) \cap \cl_j(R')|$.  If $R = S_1
      \setminus S_2$, $|R \cap R'| = |(S_1 \setminus S_2) \cap R'| =
      |(\cl_{j-1}(S_1) \setminus \cl_{j-1}(S_2)) \cap \cl_{j-1}(R')| =
      |\cl_{j}(S_1 \cap S_2) \cap \cl_{j}(R')| = |\cl_{j}(R) \cap
      \cl_{j}(R')|$. Similarly, if $R = S_2 \setminus
      S_1$. Note $R'$ is not a new set.\\

    \item {\em $R$ and $R'$ are new sets.} By definition, the new
      sets and their path images in path label $\cl_j$ are disjoint so
      $|R \cap R'| = |\cl_j(R) \cap \cl_j(R)| = 0$. Thus case proven.
    \end{enumerate}
  \item {\em Invariant} IV
    
    Due to the condition in line~\ref{ln:3waycheck}, this invariant is
    ensured at the end of every iteration.
%     \begin{enumerate} [{Case 3.}1:]
%     \item {\em $R$, $R'$ and $R''$ are not new sets.} Trivially
%       true by induction hypothesis.
%     \item {\em Only one, say $R$, is a new set.}
%       If $R = S_1 \cap S_2$,  from Lemma~\ref{lem:fourpaths} and
%       invariant III hypothesis,  this case is proven. Similarly if $R$
%       is any of the other new  sets, the case is proven by also using
%       Lemma ~\ref{lem:setminuscard}.
%     \item {\em At least two of $R, R', R''$ are new sets.}
%       The new sets are disjoint hence this case is vacuously true.
%     \end{enumerate}
  \end{enumerate} \qed
%\vspace{-6mm} 

\end{proof}

\begin{lemma}
  \label{lem:noexit1}
  If the input ICPPL $(\cF, \cl)$ to Algorithm~\ref{perms} is
  feasible, then the set of hypergraph isomorphism functions that
  defines $(\cF, \cl)$'s feasibility is the same as the set that
  defines $(\cF_j, \cl_j)$'s feasibility, if any.  Secondly, for any
  iteration $j > 0$ of the {\em \bf while} loop, the {\em \bf exit}
  statement in line~\ref{ln:exit1} will not execute.
\end{lemma}
\begin{proof}
  Since $(\cF,\cl)$ is feasible, by Lemma~\ref{lem:feasible}
  $(\cF_j,\cl_j)$ for every iteration $j > 0$ is feasible.  % Therefore,
%   every hypergraph isomorphism $\phi: supp(\cF) \rightarrow V(T)$ that
%   induces $\cl$ on $\cF$ also induces $\cl_{j-1}$ and $\cl_{j}$ on
%   $\cF_{j-1}$ and $\cF_{j}$ respectively, i.e., $\cl_{\phi[\cF_{j-1}]}
%   = \cl_{j-1}$ and $\cl_{\phi[\cF_j]} = \cl_j$. Thus it can be seen
%   that for all $x \in supp(\cF)$, for all $v \in V(T)$ the following
%   hold true.
  Also, every hypergraph isomorphism $\phi: supp(\cF) \rightarrow
  V(T)$ that induces $\cl$ on $\cF$ also induces $\cl_{j}$ on
  $\cF_{j}$, i.e., $\cl_{\phi[\cF_j]} = \cl_j$. Thus it can be seen
  that for all $x \in supp(\cF)$, for all $v \in V(T)$, if $(x,v) \in
  \phi$ then $v \in \cl_{j}(S)$ for all $S \in \cF_{j}$ such that $x
  \in S$.
% the following
%   hold true.
%   \begin{enumerate}[i. ]
%   \item If $(x,v) \in \phi$ then $v \in \cl_{j-1}(S)$ for all $S \in
%     \cF_{j-1}$ such that $x \in S$.
%   \item If $(x,v) \in \phi$ then $v \in \cl_{j}(S)$ for all $S \in
%     \cF_{j}$ such that $x \in S$
%   \end{enumerate}
  In other words, filter 1 outputs a filtered path labeling that
  ``preserves''
  hypergraph isomorphisms of the original path labeling.\\
  Secondly, line~\ref{ln:exit1} will execute iff the exit condition in
  line~\ref{ln:3waycheck}, i.e. failure of three way intersection
  preservation, becomes true in any iteration of the {\em \bf while}
  loop.  Due to Lemma~\ref{lem:invar1} Invariant IV, the exit
  condition does not occur if the input is a feasible ICPPL.\qed

%   such that $\phi(x) = v$ where $v$ is the leaf considered in the
%   first iterations of while. Clearly, $\phi$ is a renaming of
%   vertices in hypergraph $\cF$ to those in hypergraph $\cF^\cl$. Thus
%   the following facts can be observed in every iteration of the loop.

%   \begin{enumerate}[\hspace{2mm}i. ] \vspace{\topshrink}
%   \item all intersection cardinalities are preserved in this path
%     labeling \vspace{\topshrink}
%   \item element $x$ is exclusive in a hyperedge in $\cF$ since $v$ is
%     exclusive in a hyperedge in $\cF^\cl$.
%   \end{enumerate}

%   Thus the exit condition is never rendered true after $x$ and $v$ are
%   removed from their respective hyperedges. \qed

% \noindent
% This proof uses mathematical induction on the number
%   of iterations $j$, $j \ge 0$, of the loop that executed
%   without exiting. The base case, $j = 0$ is obviously true since the
%   input is an ICPPL and the exit condition cannot hold true due to
%   ICPPL property (iii).  Assume the algorithm executes till the end
%   of $j-1$th iteration without exiting at line
%  ~\ref{ln:3waycheck}. Consider the $j$th iteration. From Lemma
%  ~\ref{lem:feasible} we know that $(\cF_j, \cl_j)$ and $(\cF_{j-1},
%   \cl_{j-1})$ are feasible\remove[AS]{and from the proof in lemma
%     lem:invar1 we know that $(\cF_{j-1}, \cl_{j-1})$ satisfies all the
%     invariants defined in the lemma}.  Thus there exists a bijection
%   $\phi: supp(\cF) \rightarrow V(T)$ such that the induced path
%   % labeling on $\cF_{j-1}$ $\cl_{\phi[\cF_{j-1}]} = \cl_{j-1}$.
%   labeling on $\cF_{j}$, $\cl_{\phi[\cF_{j}]}$ and on $\cF_{j-1}$,
%   $\cl_{\phi[\cF_{j-1}]}$ are equal to $\cl_{j}$ and $\cl_{j-1}$
%   respectively.  We need to prove that for any $R, R', R'' \in
%   \cF_{j}$, $|R \cap R' \cap R''| = |\cl_j(R) \cap \cl_j(R') \cap
%   \cl_j(R'')|$.
%   The following are the possible cases that could arise. From argument
%   above, $|\cl_j(R) \cap \cl_j(R') \cap \cl_j(R'')| =
%   |\cl_{\phi[\cF_{j}]}(R) \cap \cl_{\phi[\cF_{j}]} (R') \cap
%   \cl_{\phi[\cF_{j}]} (R'')|$

%   \begin{enumerate}[a $|$]
%   \item {\em None of the sets are new. $R, R', R'' \in \cF_{j-1}$.}
%     We know $(\cF_{j-1}, \cl_{j-1})$ is feasible. Thus $|R \cap R'
%     \cap R''| = |\cl_{j-1}(R) \cap \cl_{j-1}(R') \cap \cl_{j-1}(R'')|
%     = |\cl_{j}(R) \cap \cl_{j}(R') \cap \cl_{j}(R'')|$.
%   \item {\em Only one, say $R$, is a new set.}  Let $R = S_1 \cap S_2$
%     ($S_1, S_2$ are defined in the proof of lemma
%    ~\ref{lem:invar1}). Now we have $|R \cap R' \cap R''| = |S_1 \cap
%     S_2 \cap R' \cap R''| = |\cl_{j-1}(S_1) \cap \cl_{j-1}(S_2) \cap
%     \cl_{j-1}(R') \cap \cl_{j-1}(R'')| = |\cl_{j}(R) \cap \cl_{j}(R')
%     \cap \cl_{j}(R'')|$. Thus proven. If $R$ is any of the other new
%     sets, the same claim can be verified using lemma
%    ~\ref{lem:setminuscard}.
%     % \item []{\bf Case 3:}
%   \item {\em At least two of $R, R', R''$ are new sets.}  The new sets
%     are disjoint hence this case is vacuously true.
%   \end{enumerate}
%   \qed
%   \tnote[E2]{remove the induction proof. just text saying x and v are
%     exclusive in these sets therefore the intersection cardinalities
%     don't change thus all invariants are still true} 
\end{proof}

\xnoindent As a result of Algorithm~\ref{perms} each leaf $v$ in $T$
is such that there is exactly one set in $\cF$ with $v$ as a vertex in
the path assigned to it.  In Algorithm~\ref{leafasgn} we identify
elements in $supp(\cF)$ whose images are leaves in a hypergraph
isomorphism if one exists.  Let $S \in \cF$ be such that $\cl(S)$ is a
path with leaf and $v \in V(T)$ is the unique leaf incident on it.  We
define a new path labeling $\cl_{new}$ such that $\cl_{new}(\set{x}) =
\set{v}$ where $x$ an arbitrary element from $S \setminus \bigcup_{\hS
  \ne S} \hS$. In other words, $x$ is an element present in no other
set in $\cF$ except $S$. This is intuitive since $v$ is present in no
other path image under $\cl$ other than $\cl(S)$.  The element $x$ and
leaf $v$ are then removed from the set $S$ and path $\cl(S)$
respectively. After doing this for all leaves in $T$, all path images
in the new path labeling $\cl_{new}$ except leaf labels (a path that
has only a leaf is called the {\em leaf label} for the corresponding
single element hyperedge or set) are paths from a new pruned tree $T_0
= T \setminus \{v \mid v \text{ is a leaf in }
T\}$. Algorithm~\ref{leafasgn} is now presented with details.


\begin{algorithm}[h]
  \caption{Leaf labeling from an ICPPL {\tt filter\_2($\cF, \cl, T$)}}
  \label{leafasgn}
  \begin{algorithmic}[\lndisplay]
    \STATE $\cF_0 \assign \cF$, $\cl_0(S) \assign \cl(S)$ for all $S \in \cF_0$
    \COMMENT {Path images are such that no two path images share a
      leaf.}
    \STATE $j \assign 1$\\
    \WHILE {there is a leaf $v$ in $T$ and a unique $S_1 \in
      \cF_{j-1}$ such that $v \in \cl_{j-1}(S_1)$ }\label{uniqueleaf}
    \STATE $\cF_j \assign \cF_{j-1} \setminus \{S_1\}$\\
    \STATE for all $S \in \cF_{j-1}$ such that $S \ne S_1$ set
    $\cl_j(S) \assign
    \cl_{j-1}(S)$\\
    \STATE $X \assign S_1 \setminus \bigcup_{S \in \cF_{j-1}, S \ne S_1}S$\\
    \IF{$X$ is empty} \label{xempty} \STATE {\bf exit} \label{ln:exit2} \ENDIF
    \STATE $x \assign $ arbitrary element from $X$\\
    \STATE $\cF_j \assign \cF_j \cup \{\{x\}, S_1 \setminus \{x\}\} $\\
    \STATE $\cl_j(\{x\}) \assign \{v\}$\\
    \STATE $\cl_j(S_1 \setminus \{x\}) \assign \cl_{j-1}(S_1) \setminus \{v\}$\\
    \STATE $j \assign j+1$\\
    \ENDWHILE
    \STATE $\cF' \assign \cF_j$, $\cl' \assign \cl_j$\\
    \RETURN $(\cF', \cl')$
  \end{algorithmic}
\end{algorithm}

\xnoindent
Suppose the input ICPPL $(\cF, \cl)$ is feasible, yet set $X$ in
Algorithm~\ref{leafasgn} is empty in some iteration of the {\bf while}
loop. This will abort our procedure of finding the hypergraph
isomorphism. The following lemma shows that this will not happen.

\begin{lemma}
  \label{lem:xnotempty}
  If the input ICPPL $(\cF, \cl)$ to Algorithm~\ref{leafasgn} is
  feasible, then for all iterations $j > 0$ of the {\em \bf while}
  loop, the {\em \bf exit} statement in line~\ref{ln:exit2} does not
  execute.
\end{lemma}
\begin{proof}
  Assume $X$ is empty for some iteration $j > 0$. We know that $v$ is
  an element of $\cl_{j-1}(S_1)$. Since it is uniquely present in
  $\cl_{j-1}(S_1)$, it is clear that $v \in \cl_{j-1}(S_1) \setminus
  \bigcup_{(S \in \cF_{j-1}) \wedge (S \ne S_1)}\cl_{j-1}(S)$.  Note
  that for any $x \in S_1$ it is contained in at least two sets due to
  our assumption about cardinality of $X$. Let $S_2 \in \cF_{j-1}$ be
  another set that contains $x$. From the above argument, we know $v
  \notin \cl_{j-1}(S_2)$. Therefore there cannot exist a hypergraph
  isomorphism bijection that maps elements in $S_2$ to those in
  $\cl_{j-1}(S_2)$. This contradicts our assumption that the input is
  feasible. Thus $X$ cannot be empty if input is ICPPL and feasible.
  \qed
\end{proof}

\begin{lemma}
  \label{lem:invar3}
  In Algorithm~\ref{leafasgn}, for all $j > 0$, at the end of the
  $j$th iteration of the {\bf while} loop the four invariants given in
  Lemma~\ref{lem:invar1} hold.
\end{lemma}
\begin{proof}
  By Lemma~\ref{lem:xnotempty} we know that set $X$ will not be empty
  in any iteration of the {\em \bf while} loop if input ICPPL $(\cF,
  \cl)$ is feasible and $\cl_j$ is always computed for all $j >
  0$. Also note that removing a leaf from any path keeps the new path
  connected. Thus invariant I is obviously true. In every iteration $j
  > 0$, we remove exactly one element $x$ from one set $S$ in $\cF$
  and exactly one vertex $v$ which is a leaf from one path
  $\cl_{j-1}(S)$ in $T$. This is because as seen in
  Lemma~\ref{lem:xnotempty}, $x$ is exclusive to $S$ and $v$ is
  exclusive to $\cl_{j-1}(S)$. Due to this fact, it is clear that the
  intersection cardinality equations do not change, i.e., invariants
  II, III, IV remain true. On the other hand, if the input ICPPL is
  not feasible the invariants are vacuously true. \qed
\end{proof}

% \textcolor{cyan}{
% \begin{lemma}
%   \label{lem:notfeasibleexit}
%   \tnote[a]{IS THIS CORRECT?}  If input ICPPL $(\cF, \cl)$ is not
%   feasible, then in one of the recursive calls to Algorithm --3--, the
%   {\em \bf exit} statement in line x in Algorithm --1 or line y in
%   Algorithm --2 will get executed.
% \end{lemma}
% }


\xnoindent We have seen two filtering algorithms above, namely,
Algorithm~\ref{perms} {\tt filter\_1} and Algorithm~\ref{leafasgn}
{\tt filter\_2} which when executed serially repectively result in a
new ICPPL on the same universe $U$ and tree $T$. We also proved that
if the input is indeed feasible, these algorithms do indeed output the
filtered ICPPL. Now we present the algorithmic characterization of a
feasible tree path labeling by way of Algorithm~\ref{al:icppl-find-isomorph}.

\xnoindent Algorithm~\ref{al:icppl-find-isomorph} computes a
hypergraph isomorphism $\phi$ recursively using Algorithm~\ref{perms}
and Algorithm~\ref{leafasgn} and pruning the leaves of the input
tree. In brief, it is done as follows. Algorithm~\ref{leafasgn} gives
us the leaf labels in $\cF_2$, i.e., the elements in $supp(\cF)$ that
map to leaves in $T$, where $(\cF_2, \cl_2)$ is the output of
Algorithm~\ref{leafasgn}. All leaves in $T$ are then pruned away. The
leaf labels are removed from the path labeling $\cl_2$ and the
corresponding elements are removed from the corresponding sets in
$\cF_2$. This tree pruning algorithm is recursively called on the
altered hypergraph $\cF'$, path label $\cl'$ and tree $T'$. The
recursive call returns the bijection $\phi''$ for the rest of the
elements in $supp(\cF)$ which along with the leaf labels $\phi'$ gives
us the hypergraph isomorphism $\phi$.  The following lemma formalizes
the characeterization of feasible path labeling.

\begin{algorithm}[h]
  \caption{{\tt get-hypergraph-isomorphism($\cF, \cl, T$)}}
  \label{al:icppl-find-isomorph}
  \begin{algorithmic}[\lndisplay]

    \IF{$T$ is empty}
    \RETURN $\emptyset$\\
    \ENDIF
    \STATE $L \assign \{v \mid v \text{ is a leaf in }      T\}$\\
    \STATE $(\cF_1, \cl_1) \assign$ {\tt filter\_1($\cF, \cl,
      T$)}\\
    \STATE $(\cF_2, \cl_2) \assign$ {\tt filter\_2($\cF_1,
      \cl_1, T$)}\\

    \STATE $(\cF', \cl') \assign (\cF_2, \cl_2)$\\
    \STATE $\phi' \leftarrow \emptyset$

    \FOR {every $v \in L$}
    \STATE $\phi'(x) \assign v$ where $x \in \cl_2^{-1}(\{v\})$
    \COMMENT {Copy the leaf labels to a one to one function $\phi':
      supp(\cF) \rightarrow L$
      }\\
    \STATE Remove $\{x\}$ and $\{v\}$ from $\cF'$, $\cl'$  appropriately\\
    \ENDFOR

    \STATE $T' \assign T \setminus L$

    \STATE $\phi'' \assign$ {\tt
      get-hypergraph-isomorphism($\cF', \cl', T'$)}
    \STATE $\phi \assign \phi'' \cup \phi'$ \\
    \RETURN $\phi$
  \end{algorithmic}
\end{algorithm}

\begin{lemma}
  \label{lem:hyperiso}  %{lem:perm}
  If $(\cF, \cl)$ is an ICPPL from a tree $T$ and
  Algorithm~\ref{al:icppl-find-isomorph}, {\tt
    get-hypergraph-isomorphism ($\cF, \cl, T$)} returns a non-empty
  function, then there exists a hypergraph isomorphism $\phi :
  supp(\cF) \rightarrow V(T)$ such that the $\phi$-induced tree path
  labeling is equal to $\cl$ or $\cl_\phi = \cl$.
\end{lemma}
\begin{proof}
  It is clear that in the end of every recursive call to
  Algorithm~\ref{al:icppl-find-isomorph}, the function $\phi'$ is one
  to one involving all the leaves in the tree passed as input to that
  recursive call. Moreover, by Lemma~\ref{lem:noexit1} and
  Lemma~\ref{lem:xnotempty} it is consistent with the tree path
  labeling $\cl$ passed. The tree pruning is done by only removing
  leaves in each call to the function and is done till the tree
  becomes empty. Thus the returned function $\phi: supp(\cF)
  \rightarrow V(T)$ is a union of mutually exclusive one to one
  functions exhausting all vertices of the tree. In other words, it is
  a bijection from $supp(\cF)$ to $V(T)$ inducing the given path
  labeling $\cl$ and thus a hypergraph isomorphism. \qed
\end{proof}

\begin{theorem}
  \label{th:charac}
  A path labeling $(\cF, \cl)$ on tree $T$ is feasible iff it is an
  ICPPL and Algorithm~\ref{al:icppl-find-isomorph} with $(\cF, \cl,
  T)$ as input returns a non-empty function.
\end{theorem}
\begin{proof}
  From Lemma~\ref{lem:hyperiso}, we know that if $(\cF, \cl)$ is an
  ICPPL and Algorithm~\ref{al:icppl-find-isomorph} with $(\cF, \cl,
  T)$ as input returns a non-empty function, $(\cF, \cl)$ is feasible.
  Now consider the case where $(\cF, \cl)$ is feasible. i.e. there
  exists a hypergraph isomorphism $\phi$ such that $\cl_\phi =
  \cl$. Lemma~\ref{lem:noexit1} and Lemma~\ref{lem:xnotempty} show us
  that filter 1 and filter 2 do not exit if input is feasible. Thus
  Algorithm~\ref{al:icppl-find-isomorph} returns a non-empty function.\qed
\end{proof}


\subsection{ICPPL when given tree is a path}
\label{subsec:icpplicpia}
\xnoindent
Consider a special case of ICPPL with the following properties.
\begin{enumerate}
\item Given tree $T$ is a path. Hence, all path labels are interval labels.
\item Only pairwise intersection cardinality
  preservation is sufficient. i.e. property (iii) in ICPPL is not enforced.
\item The filter algorithms do not have {\em \bf exit} statements.
\end{enumerate}
This is called an Intersection Cardinality Preservation Interval
Assignment (ICPIA) \cite{nsnrs09}. This structure and its algorithm is
used in the next section for finding tree path labeling from a
$k$-subdivided star due to this graph's close relationship with
intervals. 


\section{Testing for feasible path assignments to $k$-subdivided star}
\label{sec:ksubdivstar}
In this section we consider the problem of assigning paths from a
$k$-subdivided star $T$ to a given set system $\cF$.  We consider
$\cF$ for which the overlap graph $\bO(\cF)$ is connected.  The
overlap graph is well-known from the work of
\cite{kklv10,nsnrs09,wlh02}.  We use the notation in
\cite{kklv10}. Recall from Section~\ref{sec:prelims} that hyperedges
$S$ and $S'$ are said to overlap, denoted by $S \overlap S'$, if $S$ and $S'$
have a non-empty intersection but neither of them is contained in the
other. The overlap graph $\bO(\cF)$ is a graph in which the vertices
correspond to the sets in $\cF$, and the vertices corresponding to the
hyperedges $S$ and $S'$ are adjacent if and only if they overlap.  Note
that the intersection graph of $\cF$, $\bI(\cF)$ is different from
$\bO(\cF)$ and $\bO(\cF) \subseteq \bI(\cF)$.  A connected component
of $\bO(\cF)$ is called an overlap component of $\cF$.  An interesting
property of the overlap components is that any two distinct overlap
components, say $\cO_1$ and $\cO_2$, are such that any two sets $S_1
\in \cO_1$ and $S_2 \in \cO_2$ are disjoint, or, w.l.o.g, all the sets
in $\cO_1$ are contained within one set in $\cO_2$.  This containment
relation naturally determines a decomposition of the overlap
components into rooted containment trees.  We consider the case when
there is only one rooted containment tree, and we first present our
algorithm when $\bO(\cF)$ is connected.  It is easy to see that once
the path labeling to the overlap component in the root of the
containment tree is achieved, the path labeling to the other overlap
components in the rooted containment tree is essentially finding a
path labeling when the target tree is a path: each target path is a
path that is allocated to sets in the root overlap component.
Therefore, for the rest of this section, $\bO(\cF)$ is a connected
graph. We also assume that all hyperedges are of cardinality at most
$k+2$.


\xnoindent
Recall from Section~\ref{sec:prelims} that a $k$-subdivided star is a
star with each edge subdivided $k$ times.  Therefore, a $k$-subdivided
star has a central vertex which we call the {\em root}, and each root
to leaf path is called a {\em ray}.  First, we observe that by
removing the root $r$ from $T$, we get a collection of $p$ vertex
disjoint paths of length $k+1$, $p$ being the number of leaves in $T$.
We denote the rays by $R_1, \ldots, R_p$, and the number of vertices in
$R_i$, $i \in [p]$ is $k+2$.  Let
$\seq{v_{i1},\ldots,v_{i(k+2)}=r}$ denote the sequence of vertices in
$R_i$, where $v_{i1}$ is the leaf. Note that $r$ is a common vertex to all $R_i$.
  

\xnoindent In this section the given hypergraph, the
$k$-subdivided star and the root of the star are denoted by $\cO$, $T$ and vertex
$r$, respectively.
% For each hyperedge $X \in \cO$, we will maintain a 2-tuple of non-negative
% numbers $\seq{p_1(X), p_2(X)}$.  The numbers satisfy the property that
% $p_1(X) + p_2(X) \leq |X|$, and at the end of path labeling, for each
% $X$, $p_1(X) + p_2(X) = |X|$.  This signifies the algorithm tracking
% the lengths of subpaths of the path assigned to $X$ from at most two
% rays. We also maintain another parameter called the {\em residue} of
% $X$ denoted by $s(X)=|X| - p_1(X)$. This signifies the residue path
% length that must be assigned to $X$ which must be from another
% ray. For instance, if $X$ is labeled a path from only one ray, then
% $p_1(X) = |X|$, $p_2(X) = 0$ and $s(X) = 0$.

% \xnoindent
% We iteratively consider each ray from which paths will be assigned to
% hyperedges. At the beginning of each iteration hyperedges of $\cO$ are
% classifed into the following disjoint sets.
% \begin{enumerate}
% \item [$\cL_1^i$] {\em Labeled without $r$.} Those that have been
%   labeled with a path which does not contain $r$ in one of the
%   previous iterations.\\  $\cL_1^i = \set{ X \mid p_1(X) = |X| \text{ and
%     } p_2(X) = 0 \text{ and } s(X) = 0, X \in \cO}$
% \item [$\cL_2^i$] {\em Labeled with $r$.} Those that have been labeled
%   with two subpaths of $\cl(X)$ containing $r$ from two different rays
%   in two previous iterations.\\ $\cL_2^i = \set{X \mid 0 < p_1\left(X\right),
%     p_2\left(X\right) < |X| \text{ and } s(X) = 0, X \in \cO}$
%   \item [$\cT_1^i$] {\em Type 1 / partially labeled.} Those that have
%     been labeled with one path containing $r$ from a single ray in one
%     of the previous iterations. Here, $p_1(X)$ denotes the length of
%     the subpath of $\cl(X)$ that $X$ has been so far labeled
%     with. Also, it is clear that such a path must start at $r$ and
%     must be in a ray different from the one corresponding to $p_1(X)$.\\
%     $\cT_1^i = \set{ X \mid 0 < p_1(X) < |X| \text{ and } p_2(X) = 0
%       \text{ and } s(X) > 0, X \in \cO}$
%   \item [$\cT_2^i$] {\em Type 2 / not labeled.} Those that have not been
%     labeled with a path in any previous iteration.\\
%     $\cT_2^i = \set{ X \mid p_1(X) = p_2(X) = 0 \text{ and } s(X) = |X|,
%       X \in \cO}$
% \end{enumerate}
% \vspace{-2mm}
% \begin{align*}
%   \cO &= \cL_1^i \cup \cL_2^i \cup \cT_1^i \cup \cT_2^i, \text{    } \forall i \in [p]\\
%   \cO_i &= \cT_1^i \cup \cT_2^i, \text{    } \forall i \in [p]
% \end{align*}

\xnoindent
The set $\cO_i$ refers to the set of hyperedges $\cT_1^i \cup \cT_2^i$ in
the $i$th iteration.  Note that $\cO_1 = \cO$.  In the $i$th
iteration, hyperedges from $\cO_i$ are assigned paths from $R_i$ using
the following rules. Also the end of the iteration, $\cL_1^{i+1}, \cL_2^{i+1}, \cT_1^{i+1},
\cT_2^{i+1}$ are set to $\cL_1^{i}, \cL_2^{i}, \cT_1^{i},
\cT_2^{i}$ respectively, along with some case-specific changes mentioned in the
rules below.

\noindent
Suppose the given hypergraph has a feasible path labeling to the given
tree $T$. Let this labeling be $\cl$. Now we present an algorithm that
lists the sequence of edges that maintain a set of properties
necessary for them to have a feasible tree path labeling.
\begin{enumerate}[I.]
\item {\bf Step 1}
  \begin{enumerate}
  \item {\bf {There are type 1
        edges, $\mathbf{|\cT_1^i| > 0}$}:} 
% , and for each such $X$ assign the
%     path in $R_i$ of length $s(X)$ starting at $v_{i,k+2}=r$.  $X$ is
%     then removed from $\cO_i$, and $p_2(X)=s(X)$, and $s(X)=0$.If any
%     intersection cardinality is violated because of this assignment,
%     EXIT reporting the non-existence of a feasible assignment.
  {If there is only one $\cT_1^i$ hyperedge, label it to the unique
    path in $R_i$ of length $s(X)$ starting at $v_{i(k+2)}=r$. If
    there are more than one $\cT_1^i$ hyperedges, for any pair of
    hyperedges $S_1, S_2 \in \cT_1^i$, one of the following is
    true. Let $\cl'(X) \subseteq R_j$ denote the subpath of length
    $p_1(X)$ that has been assigned to $X$ from a previously
    considered ray $R_j$.}
  \begin{enumerate}[{\em Case }1. ]
  \item $|\cl'(S_1) \cap \cl'(S_2)| = 1$ and $|S_1 \cap S_2| > 1$
    i.e. they are in different rays and their residues must contain
    their intersection. Hence assign both to $R_i$.
  \item $|\cl'(S_1) \cap \cl'(S_2)| > 1$ and $|S_1 \cap S_2| >
    |\cl'(S_1) \cap \cl'(S_2)|$ i.e. they are in the same rays and
    their residues must contain their intersection. Hence assign both
    to $R_i$.
  \item $|\cl'(S_1) \cap \cl'(S_2)| > 1$ and $|S_1 \cap S_2| =
    |\cl'(S_1) \cap \cl'(S_2)|$. i.e. the intersection cardinality has
    been met. Hence assign only $S_1$ to $R_i$ and $S_2$ must be
    assigned to some $i \ge p$.
  \end{enumerate}
  Remove the labeled edge or edges from $\cO_i$ and add it to
  $\cO_{i+1}$. Also do not add it to $\cT_1^{i+1}$ and add to
  $\cL_2^{i+1}$.
\item {\bf {There are no type-1 edges nor} super-marginal type-2
    hyperedges:} If all type-2 edges of length at most $k+1$, EXIT by
  saying there is no ICPPL.  Otherwise, Find an $X$ of type-2 such
  that $|X| \geq k+2$, assign it the unique path starting at
  $v_{i(k+1)}=r$ of length $k+2$, and set $p_1(X)=k+2,
  s(X)=|X|-(k+2)$, assign it the unique path starting at
  $v_{i(k+1)}=r$ of length $k+2$, mark this as a type-1 hyperedge, and
  add it to $\cO_{i+1}$ after removing it from $\cO_i$.
  % \remove[a]{Next, iteratively, consider each type-1 edge $X \in
  %   \cO_i$, and if it intersects a previously assigned $Y$ such that
  %   $\cl(Y) \subseteq R_i$, assign a path of length $s(X)$
  %   containing $v_{i(k+2)}=r$, set $p_2(X)=s(X), s(X)=0$, and remove
  %   $X$ from $\cO_i$.}\tnote[a]{covered in step 2}

\item {\bf There are super-marginal type-2 hyperedges {and no type-1
      edges:}} Let $X$ be a super-marginal hyperedge. If $|X| \leq
  k+2$, then $X$ is assigned the path of length $|X|$ starting at
  $v_{i1}$, and $p_1(X) = |X|, p_2(X)=0$. $X$ is removed from $\cO_i$
  and not added to $\cO_{i+1}$.  If $|X| > k+2$, then $p_1(X)=k+2$,
  assign it the unique path starting at $v_{i(k+1)}=r$ of length $k+2$
  and $s(X) = |X| - (k+2)$.  In this case, $X$ is classified as a
  Type-1 edge, removed from $\cO_i$, and added to $\cO_{i+1}$.
\end{enumerate}
\item {\bf Step 2:} Iteratively, a hyperedge $X$ is selected from
  $\cO_i$ that has an overlap with one of the hyperedges $Y$ such that
  $\cl(Y) \subseteq R_i$, and a unique path is assigned to $X$.  The
  path, say $U(X) \subseteq R_i$, that is assigned can be decided
  unambiguously since the $X \overlap Y$, and all intersection
  cardinalities can be preserved only at one of the ends of $\cl(Y)$.
  
  \noindent
  Let $\cl(X)$ denote the unique path assigned to $X$.  If $X$ is a
  type-2 hyperedge and if the unique path of length $|X|$ does not
  contain $r$, then $p_1(X) = |X|, p_2(X)=0$, and $X$ is removed from
  $\cO_i$.  In the case when $\cl(X)$ has to contain $r$, then
  $p_1(X)$ is the length of the path, $p_2(X)=0$, and $s(X) =
  |X|-p_1(X)$. Further $X$ is classified as a type-1 hyperedge, added
  to $\cO_{i+1}$ and removed from $\cO_i$.  In the case when $X$ is a
  type-1 hyperedge, then we check if $U(X)$, which is of length $s(X)$
  contains $r$. If it does, then we assign $\cl(X) \leftarrow \cl(X)
  \cup U(X)$, remove $X$ from $\cO_i$ and do not add it to
  $\cO_{i+1}$.  If not, then we {\bf Exit} reporting that an
  assignment cannot be found.  The iteration ends when no hyperedge in
  $\cO_i$ has an overlap with a hyperedge assigned to $R_i$.
\end{enumerate}

\noindent
The order in which they are assigned is the sequence of hyperedges
with the following properties. Note that if such a sequence cannot be
computed, $\cO$ cannot have a feasible path labeling from $T$.

\noindent
\tnote[a]{need to formalize the below properties} In the following
lemmas we identify a set of necessary conditions for $\cF$ to have an
ICPPL in the $k$-subdivided star $T$.  If during the execution of the
above algorithm, one of these necessary conditions is not satisfied,
the algorithm exits reporting the non-existence of an ICPPL.
\begin{lemma}
  Let all hyperedges in $\cO_i$ be type-1 edges.  Then there is a
  maximal subset $\ccT_i \subseteq \cO_i$ with the following
  properties:
  \begin{enumerate}
  \item $\ccT_i$ form an inclusion chain.
  \item For all $X \in \ccT_i$, $s(X) \leq k+2$, and There is a an $X
    \in \ccT_i$ such that $s(X)=k+2$.
  \end{enumerate}
\end{lemma}
\begin{lemma}
  If there are no \marginal type-2 edges in $\cO_i$, then there exists
  at least $p-i$ type-2 hyperedges $X \in \cO_i$ such that $|X| \geq
  k+2$.
  % is a exist type-2 hyperedges of length at least $k+2$, and there
  % must be at least $r-i$ such type-2 hyperedge, basically at least
  % one per ray.  Each hyperedge contained in such an hyperedges of
  % length at least $k+2$, is transitively related under the overlap
  % relation to one that gets assigned a path that contains
  % $r$. Basically, the proof will argue that otherwise, there will be
  % a \marginal type-2 hyperedge, or that it is infeasible.  Also, it
  % is here we assume that we assume that the size of the two
  % universes is the same.
\end{lemma}
\begin{lemma}
  At the end of Step-1 in the $i$-th iteration, if one hyperedge $X$
  of type-2 is such that $\cl(X) \subseteq R_i$, then all other
  hyperedges in $\cO_i$ are connected to $X$ in the overlap component.
\end{lemma}
\begin{lemma}
  At the end of Step-2, If control has exit at any time, there is no
  ICPPL. If control has not exit, then $R_i$ is saturated.  No
  hyperedge of $\cO_{i+1}$ will get a path from $R_i$ in the future
  iterations.  No type-2 hyperedge of $\cO_{i+1}$ will get a path from
  $R_i$.  Basically $R_i$ is done.
\end{lemma}
\begin{lemma}
  Finally, we need to prove that the assignment is an ICPPL. Secondly,
  if there is a permutation then maps sets to paths, then it is indeed
  an ICPPL, and our algorithm will basically find it.  It is a unique
  assignment upto permutation of the leaves.
\end{lemma}


\begin{lemma}
  \label{lem:sup-mar}
  If $X \in \cF$ is super-marginal and $(\cF, \cl)$ is a feasible
  tree path labeling to tree $T$, then $\cl(X)$ will
  contain a leaf in $T$.\\
  On the otherhand, if $\cl(X)$ has a leaf in
  $T$, $X$ is marginal but may not be super-marginal.
\end{lemma}
\begin{proof}
  Suppose $X \in \cF$ is super-marginal and $(\cF, \cl)$ is a feasible
  path labeling from $T$.  Assume $\cl(X)$ does not have a leaf.  Let
  $R_i$ be one of the rays (or the only ray) $\cl(X)$ is part of.
  Since $X$ is in a connected overlap component, there exists $Y_1 \in
  \cF$ and $X \nsubseteq Y_1$ such that $Y_1 \overlap X$ and $Y_1$ has
  at least one vertex closer to the leaf in $R_i$ than any vertex in
  $X$. Similarly with the same argument there exists $Y_2 \in \cF$
  with same subset and overlap relation with $X$ except it has has at
  least one vertex farther away from the leaf in $R_i$ than any vertex
  in $X$. Clearly $Y_1 \cap X$ and $Y_2 \cap X$ cannot be part of same
  inclusion chain which contradicts that assumption $X$ is
  super-marginal. Thus the claim is proven.\qed
\end{proof}

\noindent
Consider the overlap graph $\bO(\cF)$ of the given hypergraph
$\cF$. Let $S_{sm} \in \cF$ be such that it is a super-marginal
hyperedge.  Algorithm~\ref{al:ktree-label} uses $S_{sm}$ along with
the overlap graph $\bO(\cF)$ to calculate the feasible tree path
labeling to the $k$-subdivded tree $T$.

\begin{algorithm}[h]
  \caption{{\tt compute-ksubtree-path-labeling($X, \cF, T$)}}
  \label{al:ktree-label}
  \begin{algorithmic}[\lndisplay]
    \IF{$X = S_{sm}$}
    \STATE -- TBD --\\
    \ELSE 
    \STATE -- TBD --\\
    \ENDIF

 \end{algorithmic}
\end{algorithm}


% \subsection{temp section from nsnsr09}

% \begin{algorithm}
%   \caption{Basic step in an algorithm to find an ICPIA for a prime
%     matrix $M'$}
%   \label{ds-algo}
%   ICPIA(Set $S$, Integer $p > 0$)

% \noindent
% /* {\tt $S \cap S^i \not = \phi$ for some $i \in \{1,\ldots,p\}$, but
%   $S \not\subseteq S^i$, $S^i \not\subseteq S$.  \\ Assigns to $S$ an
%   interval $I$ such that $\{I^1,\ldots,I^p,I\}$ forms an ICPIA for
%   $\{S^1,\ldots,S^p,S\}$.} */
% \begin{algorithmic}
%   \STATE Let $|S \cap S^i| = z$. \\
%   \STATE Let $I_l$ be the interval such that $|I_l \cap I^i| = z$, $|I_l| = |S|$ and the $z$ common elements are the smallest elements of $I^i$. \\
%   \STATE Let $I_r$ be the interval such that $|I_r \cap I^i| = z$, $|I_r| = |S|$, and the $z$ common elements are the largest elements of $I^i$. \\
%   \IF {$p == 1$}
%   \STATE Assign $I_l$ to $S$ \\
%   \STATE /* {\tt In this case, $I_r$ could also be assigned to $S$.  This will yield the {\bf other} ICPIA} */\\
%   \ELSE \IF{$|I_l \cap I^q| = |S \cap S^q|$ for each $q \in
%     \{1,\ldots,p\}$}
%   \STATE Assign $I_l$ to $S$ and exit.\\
%   \ENDIF \IF{$|I_r \cap I^q| = |S \cap S^q|$ for each $q \in
%     \{1,\ldots,p\}$}
%   \STATE Assign $I_r$ to $S$ and exit.\\
%   \ENDIF \ENDIF \STATE Report no ICPIA and exit.
% \end{algorithmic}
% \end{algorithm}


% \begin{theorem}
%   Algorithm~\ref{ds-algo} outputs an ICPIA to a prime matrix $M'$ iff
%   there is an ICPIA for $M'$.
% \end{theorem}
% \begin{proof}
%   The only-if part of the theorem is straightforward.  We now show
%   that if there is an ICPIA for $M'$, then Algorithm~\ref{ds-algo}
%   will indeed discover it.  The key fact is that in $M'$ for each set
%   $S$, there is another set $T \in M'$ such that $S \cap T \not =
%   \phi$, and $S$ and $T$ are not contained in each other.  Due to this
%   fact, there are exactly two ICPIAs for $M'$.  The two distinct
%   ICPIAs differ based on the interval assigned to $S_1$, see Algorithm
%  ~\ref{ds-algo}.  If $I_l$ is assigned to $S_1$, then we get one, and
%   the other ICPIA is obtained by assigning $I_r$ to $S_1$.  For each
%   subsequent set, say $S^j$, the interval to be assigned is forced.
%   It is forced due to the fact that the interval assigned to $S^j$ is
%   based on the interval assigned to $S^i$, where $S^i \cap S^j \not =
%   \phi$, and $S^i \not\subseteq S^j$, and $S^j \not\subseteq S^i$.
%   Given the fact that the algorithm is an exact implementation of
%   these observations, it follows that Algorithm~\ref{ds-algo} finds an
%   ICPIA if there is one.
% \end{proof}

\chapter{ Research - [title TBD]} 
\tnote[a]{the matrix theory submitted in walcom, any other work - need to see notes.}


\chapter{Conclusion}

%%%%%%%%%%%%%%%%%%%%%%%%%%%%%%%%%%%%%%%%%%%%%%%%%%%%%%%%%%%%
% Appendices.

\appendix

\chapter{Appendix A}


%%%%%%%%%%%%%%%%%%%%%%%%%%%%%%%%%%%%%%%%%%%%%%%%%%%%%%%%%%%%
% Bibliography.

\begin{singlespace}
  \bibliography{../lib/cop-variants__thesis}
\end{singlespace}


%%%%%%%%%%%%%%%%%%%%%%%%%%%%%%%%%%%%%%%%%%%%%%%%%%%%%%%%%%%%
% List of papers

\listofpapers

\begin{enumerate}  
\item Authors....  \newblock
 Title...
  \newblock {\em Journal}, Volume,
  Page, (year).
\end{enumerate}  

\end{document}



%%%%%%%%%%%%%%%%%%%%%%%%%%%%%%%%%%%%%%%%%%%%%%%%%%
% DOCUMENTATION ON THE THESIS CLASS FILE ETC.


% Before reading any further please note that you are strongly advised
% against changing any of the formatting options used in the class
% provided in this directory, unless you are absolutely sure that it
% does not violate the IITM formatting guidelines.  \emph{Please do not
%   change the margins or the spacing.}  If you do change the formatting
% you are on your own (don't blame me if you need to reprint your entire
% thesis).  In the case that you do change the formatting despite these
% warnings, the least I ask is that you do not redistribute your style
% files to your friends (or enemies).

% It is also a good idea to take a quick look at the formatting
% guidelines.  Your office or advisor should have a copy.  If they
% don't, pester them, they really should have the formatting guidelines
% readily available somewhere.


% To generate PDF's with the links from the \verb+hyperref+ package use
% the following command:
% \begin{verbatim}
% % dvipdfm -o thesis.pdf thesis.dvi
% \end{verbatim}

% \section{Package Options}

% Use this thesis as a basic template to format your thesis.  The
% \verb+iitmdiss+ class can be used by simply using something like this:
% \begin{verbatim}
% \documentclass[PhD]{iitmdiss}  
% \end{verbatim}

% To change the title page for different degrees just change the option
% from \verb+PhD+ to one of \verb+MS+, \verb+MTech+ or \verb+BTech+.
% The dual degree pages are not supported yet but should be quite easy
% to add.  The title page formatting really depends on how large or
% small your thesis title is.  Consequently it might require some hand
% tuning.  Edit your version of \verb+iitmdiss.cls+ suitably to do this.
% I recommend that this be done once your title is final.

% To write a synopsis simply use the \verb+synopsis.tex+ file as a
% simple template.  The synopsis option turns this on and can be used as
% shown below.
% \begin{verbatim}
% \documentclass[PhD,synopsis]{iitmdiss}                                
% \end{verbatim}

% Once again the title page may require some small amount of fine
% tuning.  This is again easily done by editing the class file.

% This sample file uses the \verb+hyperref+ package that makes all
% labels and references clickable in both the generated DVI and PDF
% files.  These are very useful when reading the document online and do
% not affect the output when the files are printed.


% \section{Example Figures and tables}

% Fig.~\ref{fig:iitm} shows a simple figure for illustration along with
% a long caption.  The formatting of the caption text is automatically
% single spaced and indented.  Table~\ref{tab:sample} shows a sample
% table with the caption placed correctly.  The caption for this should
% always be placed before the table as shown in the example.


% \begin{figure}[htpb]
%   \begin{center}
%     \resizebox{50mm}{!} {\includegraphics *{iitm.eps}}
%     \resizebox{50mm}{!} {\includegraphics *{iitm.eps}}
%     \caption {Two IITM logos in a row.  This is also an
%       illustration of a very long figure caption that wraps around two
%       two lines.  Notice that the caption is single-spaced.}
%   \label{fig:iitm}
%   \end{center}
% \end{figure}

% \begin{table}[htbp]
%   \caption{A sample table with a table caption placed
%     appropriately. This caption is also very long and is
%     single-spaced.  Also notice how the text is aligned.}
%   \begin{center}
%   \begin{tabular}[c]{|c|r|} \hline
%     $x$ & $x^2$ \\ \hline
%     1  &  1   \\
%     2  &  4  \\
%     3  &  9  \\
%     4  &  16  \\
%     5  &  25  \\
%     6  &  36  \\
%     7  &  49  \\
%     8  &  64  \\ \hline
%   \end{tabular}
%   \label{tab:sample}
%   \end{center}
% \end{table}

% \section{Bibliography with BIB\TeX}

% I strongly recommend that you use BIB\TeX\ to automatically generate
% your bibliography.  It makes managing your references much easier.  It
% is an excellent way to organize your references and reuse them.  You
% can use one set of entries for your references and cite them in your
% thesis, papers and reports.  If you haven't used it anytime before
% please invest some time learning how to use it.  

% I've included a simple example BIB\TeX\ file along in this directory
% called \verb+refs.bib+.  The \verb+iitmdiss.cls+ class package which
% is used in this thesis and for the synopsis uses the \verb+natbib+
% package to format the references along with a customized bibliography
% style provided as the \verb+iitm.bst+ file in the directory containing
% \verb+thesis.tex+.  Documentation for the \verb+natbib+ package should
% be available in your distribution of \LaTeX.  Basically, to cite the
% author along with the author name and year use \verb+\cite{key}+ where
% \verb+key+ is the citation key for your bibliography entry.  You can
% also use \verb+\citet{key}+ to get the same effect.  To make the
% citation without the author name in the main text but inside the
% parenthesis use \verb+\citep{key}+.  The following paragraph shows how
% citations can be used in text effectively.

% More information on BIB\TeX\ is available in the book by
% \cite{lamport:86}.  There are many
% references~\citep{lamport:86,prabhu:xx} that explain how to use
% BIB\TeX.  Read the \verb+natbib+ package documentation for more
% details on how to cite things differently.

% Here are other references for example.  \citet{viz:mayavi} presents a
% Python based visualization system called MayaVi in a conference paper.
% \citet{pan:pr:flat-fst} illustrates a journal article with multiple
% authors.  Python~\citep{py:python} is a programming language and is
% cited here to show how to cite something that is best identified with
% a URL.

% \section{Other useful \LaTeX\ packages}

% The following packages might be useful when writing your thesis.

% \begin{itemize}  
% \item It is very useful to include line numbers in your document.
%   That way, it is very easy for people to suggest corrections to your
%   text.  I recommend the use of the \texttt{lineno} package for this
%   purpose.  This is not a standard package but can be obtained on the
%   internet.  The directory containing this file should contain a
%   lineno directory that includes the package along with documentation
%   for it.

% \item The \texttt{listings} package should be available with your
%   distribution of \LaTeX.  This package is very useful when one needs
%   to list source code or pseudo-code.

% \item For special figure captions the \texttt{ccaption} package may be
%   useful.  This is specially useful if one has a figure that spans
%   more than two pages and you need to use the same figure number.

% \item The notation page can be entered manually or automatically
%   generated using the \texttt{nomencl} package.

% \end{itemize}

% More details on how to use these specific packages are available along
% with the documentation of the respective packages.

% END OF DOCUMENTATION
%%%%%%%%%%%%%%%%%%%%%%%%%%%%%%%%%%%%%%%%%%%%%%%%%%%%%%%%%%%%

