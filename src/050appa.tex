\chapter{More proofs \tnote{remove/modify}}
\begin{lemma}
  \label{lem:fourpaths} Consider four paths in a tree $Q_1, Q_2, Q_3,
  Q_4$ such that they have nonempty pairwise intersection and $Q_1,
  Q_2$ share a leaf. Then there exists distinct $i, j, k \in
  \{1,2,3,4\}$ such that, $Q_1 \cap Q_2 \cap Q_3 \cap Q_4 = Q_i \cap
  Q_j \cap Q_k$.
\end{lemma}
\begin{proof}\thesisspacing
  {\em Case 1:} w.l.o.g, consider $Q_3 \cap Q_4$ and let us call it
  $Q$. This is clearly a path (intersection of two paths is a path).
  % Since $Q_1, Q_2$ share a leaf, the following are paths $Q_1
  % \setminus Q_2$, $Q_2 \setminus Q_1$, $Q_1 \cap Q_2$ and they are
  % mutually disjoint.
  Suppose $Q$ does not intersect with $Q_1 \setminus Q_2$, i.e. $Q
  \cap (Q_1 \setminus Q_2) = \O$. Then $Q \cap Q_1 \cap Q_2 = Q \cap
  Q_2$. Similarly, if $Q \cap (Q_2 \setminus Q_1) = \O$, $Q \cap Q_1
  \cap Q_2 = Q \cap Q_1$. Thus it is clear that if the intersection of
  any two paths does not intersect with any of the set differences of
  the remaining two paths, the claim in the lemma is true.
  % Note that $Q_1 \setminus Q_2$ and $Q_2
  % \setminus Q_1$ are paths because $Q_1, Q_2$ share a leaf.\\
  {\em Case 2:} Let us consider the compliment of the previous
  case. i.e. the intersection of any two paths intersects with both
  the set differences of the other two. First let us consider $Q \cap
  (Q_1 \setminus Q_2) \ne \O$ and $Q \cap (Q_1 \setminus Q_2) \ne \O$,
  where $Q = Q_3 \cap Q_4$. Since $Q_1$ and $Q_2$ share a leaf, there
  is exactly one vertex at which they branch off from the path $Q_1
  \cap Q_2$ into two paths $Q_1 \setminus Q_2$ and $Q_2 \setminus
  Q_1$. Let this vertex be $v$. It is clear that if path $Q_3 \cap
  Q_4$, must intersect with paths $Q_1 \setminus Q_2$ and $Q_2
  \setminus Q_1$, it must contain $v$ since these are paths from a
  tree. Moreover, $Q_3 \cap Q_4$ intersects with $Q_1 \cap Q_2$ at
  exactly $v$ and only at $v$ which means that $Q_1 \cap Q_2$ does not
  intersect with $Q_3 \setminus Q_4$ or $Q_4 \setminus Q_3$ which
  contradicts initial condition of this case. Thus this
  case cannot occur and case 1 is the only possible scenario. \\
  Thus lemma is proven %\qed
\end{proof}
