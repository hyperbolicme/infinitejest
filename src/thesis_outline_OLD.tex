\documentstyle[epsfig,algorithm,algorithmic,graphicx,graphics]{MS}

% \usepackage{fullpage} \usepackage{graphicx} \usepackage{epsfig}
% \usepackage{graphics} \usepackage{setspace} \usepackage{fancyhdr}
% \usepackage{algorithmic} \usepackage{algorithm} \date{}

\newtheorem{theorem}{Theorem}[section]
\newtheorem{lemma}[theorem]{Lemma}
\newtheorem{proposition}[theorem]{Proposition}
\newtheorem{corollary}[theorem]{Corollary} \newtheorem{fact}{Fact}
% \newenvironment{proof}[1][Proof]{\begin{trivlist}
%   \item[\hskip \labelsep {\bfseries #1}]}{\end{trivlist}}
\newenvironment{proof}{Proof:}{$\Box$}
\newtheorem{definition}{Definition}[section]
%%%%

%%%%

\begin{document}

\pagenumbering{roman}


\input{frontpage.tex} \input{certificate.tex} \input{dedication.tex}
\input{ack.tex} \input{abstract.tex}

\tableofcontents
\listoffigures

\chapter{Intro}
description, org of paper
\section{to area}
\section{to problem}
\section{to your thesis}
\section{thesis outline}


   
\chapter{Survey of COP and related problems}
\pagenumbering{arabic}

\section{characterization of COP - (tucker, mcconnell, prime submatrix)}
\section{Recognition of COP}
\subsection{poly algorithms (pq tree, pqr, icpia)}
\subsection{tuckers submatrices, Dom's algo to find them}
\subsection{mc connells certificate}
\section{algorithms to make a matrix into one with COP (min/max-cos-r/c) (dom)}
\section{brief analysis of hard problems with input having COP (chap 5, 6 in dom)}
\section{complexity - poly, logspace}

\chapter{Research}
\section{old and new cop tree results}




\end{document}


