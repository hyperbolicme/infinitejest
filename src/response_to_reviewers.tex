\documentclass{article}
\usepackage{../lib/mythesislib}
\usepackage{../lib/mylatexlib}
\begin{document}

\title{Response to M.S. Thesis Reviewers\\ 
 {\em Thesis Title ``Generalization of the Consecutive-ones Property''}}
\author{Anju Srinivasan}
\date{25 March 2013}

\maketitle


\section{Reviewers}

\begin{enumerate}
\item  {\bf Dr. C. R. Subramaniam}\\ Institute of Mathematical Sciences, Chennai
\item  {\bf Dr. Subhas Chandra Nandy}\\ Indian Statistical Institute, Kolkata
\end{enumerate}

\section{Response to Reviewer 1}

\subsection{Observations}
The following are a few observations mentioned in the review document
with my responses/changes.

\begin{enumerate}
\item Observation: (in Page 1, second last paragraph) {\bf The result on characterization of TPL in
    the thesis is a variant of the 1978 JCT(B) result due to Fournier
  who showed that two hypergraphs $H_1 = (V_1, X)$ and $H_2 = V_2, Y$
such that $|X| = |Y| = m$ are isomorphic iff there is a bijection
$\phi$ on $I = \{1, ... , m\}$ s.t. for every $L \subset I,
|\bigcap_{i \in L}X_i| = |\bigcap_{j=\phi(i),i \in L}Y_j|$} \\
Response: This is an important observation and was not cited in the thesis. Also \cite{f80}
generalizes \cite{br72} and \cite{fg65} by characterizing the
isomorphism of two hypergraphs by means of equicardinality of certain
edge intersections and the exclusion of certain pairs of
subhypergraphs. TPL characterization is a special case of this with
one of the hypergraphs having hyperedges that are paths from a tree
and the characterization only uses edge intersections of at most 3
hyperedges. This has been added in the Conclusions chapter Page 72
under {\em Graph isomorphism and logspace canonization}.

\item Observation: (in Page 2, first paragraph) {\bf A $k$-subdivided tree is
  a tree with a single central vertex with a number of paths each of
  $k$ edges emanating from it.}\\
Response: $k$-subdivided as defined in this thesis (Section
3.2.4, Page 43) is such that each
of the aforementioned paths are of length ``$k+1$'' edges since each such
path has $k$ nodes apart from the central node and the leaf node, i.e.
$k+2$ nodes. No change in the thesis.
\end{enumerate}


\subsection{Comments}
Responses to the comments in review document Page 2 items 0 to 7.

\begin{enumerate}
\item[0] Comment: {\bf In thesis Page 7, Definition 1.3.7: After $x_m
  \le a$, you need to add $x_m \ne a$.}\\ 
Response: True, earlier definition did not accomodate the fact $x_m
\le x_m$. Suggested change made in Page 7 to the definiton of Poset.

\item Comment: {\bf In thesis Page 7, Definition
    1.3.8: $E$ is a set of ``unordered'' pairs. You need to add the
    word ``unordered''}\\
Response: Suggested change made in to definition of Undirected Graph Page 7 Definition 1.3.8 (1).

\item Comment: {\bf In thesis Page 8, A tree is a {\em connected} and
    acyclic graph}\\
Response: The word ``connected'' has been added to the definition of Tree in
Page 7 Definition 1.3.8 (3).

\item Comment: {\bf In thesis Page 8, Definition 1.3.10 you need to
    add that one is looking at intervals of integers. Also define
    isomorphism between graphs for the sake of completeness}.\\
Response: Definition of Interval graph has been changed to say ``a set
of intervals (of integers)'' in Page 8, Definition 1.3.10. Graph
isomorphism definition has been added in Page 7, Definition 1.3.8(4)
as follows: A \textbf{graph isomorphism} between two graphs $G_1$ and
    $G_2$ is a bijection $\phi: V(G_1) \rightarrow V(G_2)$ such that
    $u, v \in V(G_1)$ are adjacent in $G_1$ if and only if $\phi(u),
    \phi(v)$ are adjacent in $G_2$. If such a bijection exists, it is
    said that $G_1$ is \textbf{isomorphic} to $G_2$, denoted by $G_1
    \cong G_2$.

\item Comment: {\bf In thesis Page 8, Definition 1.3.10 definition of
path graph: You need to emphasise that $\cP$ forms a set system}\\
Response: This change has been made - ``a set system of paths $\cP$''
in Page 8 Definition 1.3.10(2).

\item Comment: {\bf In thesis Page 12, third para of Section 1.5: you
    need to write it more precisely and carefully conveying the
    idea. As it is it is not clear what you are saying.}\\
  Response: The problem {\sc Compute feasible tree path labeling} and
  its result have been described formally and precisely in Chapter 3
  (New results). The aforementioned paragraph describes the same
  informally. These informal sentences have been footnoted with the
  corresponding definitions and theorems from Chapter 3. This change
  is in Page 12.

\item Comment: {\bf In thesis Page 28, Definition 7 (of binary
    operator /): it is not defined precisely. What do you mean by
    ``$b$ is a new element added..''?}.\\
Response: This was in thesis Page 26. $A/B$ is created by removing the elements of second operand set $B$ from
the first operand set $A$ and introducing a new element {\em not} in the
universe to $A$ to represent set $B$. This is a set operation 
mentioned in \cite{mm96} (they use ``$B$'' to denote the new element to represent
set $B$ which is also confusing) which is eventually used to decompose
permutation of elements in $A$ to permutation of elements in $A/B$ and
permutation of elements in $B$.
Perhaps it would be clearer if
$b$ is replaced by $x_B$. This change has been made in Page 26
Definition 2.2.4 (7) and other mentions of this operation ``/'' Page
27 item (iii) and $h$ has been replaced with $x_H$ in Corollary
2.2.8. There was also a typo in Page 27 paragraph under Corollary
2.2.8 item (ii) where ``$U \setminus H$'' was written instead of $U / H$.
This has also been corrected.

\item Comment: {\bf In thesis Page 39 Definition 2.3.1: the definition
  of $A_\cF$ is not clear. Where is the dependence on $\cF$?}\\
Response: This was in thesis Page 36. It is true, there is no
dependence in $\cF$. It is only dependent on the universe $U$ of
$\cF$. All occurences of $A_\cF$ has been changed to $A_\U$ in Page 36.
\end{enumerate}



\section{Response to Reviewer 2}


\bibliographystyle{alpha}%
\bibliography{../lib/cop-variants__thesis} %


\end{document}