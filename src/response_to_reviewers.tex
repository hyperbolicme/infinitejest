\documentclass[12pt,a4paper]{article}
\usepackage{fullpage}
\usepackage[normalem]{ulem}
\usepackage{../lib/mythesislib}
\usepackage{../lib/mylatexlib}
\usepackage{setspace}
%

\begin{document}

\title{Response to M.S. Thesis Review Comments\\ 
 {Thesis Title: {\em Generalization of the Consecutive-ones Property}}}
\author{Anju Srinivasan\\IIT Madras, Chennai - 600036.}
\date{25 March 2013}

\maketitle

\onehalfspacing

\section{Reviewers}

\begin{enumerate}
\item  {\bf Dr. C. R. Subramaniam}\\ Institute of Mathematical Sciences, Chennai
\item  {\bf Dr. Subhas Chandra Nandy}\\ Indian Statistical Institute, Kolkata
\end{enumerate}
The reviewers careful comments are acknowledged and
appreciated. Responses to the same are given in following sections.

\section{Response to Reviewer 1}


\subsection{Comments on corrections}
Responses to the comments in review document Page 2 items 0 to 7.

\begin{enumerate}
\item[0.] Comment: {\bf In thesis Page 7, Definition 1.3.7: After $x_m
    \le a$, you need to add $x_m \ne a$.}\\
  Response: True, earlier definition did not accomodate the fact $x_m
  \le x_m$. Suggested change made in Page 7 to the definiton of Poset.

\item Comment: {\bf In thesis Page 7, Definition 1.3.8: $E$ is a set
    of ``unordered'' pairs. You need to add the
    word ``unordered''}\\
  Response: Suggested change made in to definition of Undirected Graph
  Page 7 Definition 1.3.8 (1).

\item Comment: {\bf In thesis Page 8, A tree is a {\em connected} and
    acyclic graph}\\
  Response: The word ``connected'' has been added to the definition of
  Tree in Page 7 Definition 1.3.8 (3).

\item Comment: {\bf In thesis Page 8, Definition 1.3.10 you need to
    add that one is looking at intervals of integers. Also define
    isomorphism between graphs for the sake of completeness}.\\
  Response: Definition of Interval graph has been changed to say ``a
  set of intervals (of integers)'' in Page 8, Definition 1.3.10. Graph
  isomorphism definition has been added in Page 7, Definition 1.3.8(4)
  as follows:

  {\em A \textbf{graph isomorphism} between two graphs $G_1$ and $G_2$
    is a bijection $\phi: V(G_1) \rightarrow V(G_2)$ such that $u, v
    \in V(G_1)$ are adjacent in $G_1$ if and only if $\phi(u),
    \phi(v)$ are adjacent in $G_2$. If such a bijection exists, it is
    said that $G_1$ is \textbf{isomorphic} to $G_2$, denoted by $G_1
    \cong G_2$.}

\item Comment: {\bf In thesis Page 8, Definition 1.3.10 definition of
    path graph: You need to emphasise that $\cP$ forms a set system}\\
  Response: This change has been made - ``a set system of paths
  $\cP$'' in Page 8 Definition 1.3.10(2).

\item Comment: {\bf In thesis Page 12, third para of Section 1.5: you
    need to write it more precisely and carefully conveying the
    idea. As it is it is not clear what you are saying.}\\
  Response: The problem {\sc Compute feasible tree path labeling} and
  its result have been described formally and precisely in Chapter 3
  (New results). The aforementioned paragraph describes the same
  informally. These informal sentences have been footnoted with the
  corresponding definitions and theorems from Chapter 3. This change
  is in Page 12.

\item Comment: {\bf In thesis Page 28, Definition 7 (of binary
    operator /): it is not defined precisely. What do you mean by
    ``$b$ is a new element added..''?}.\\
  Response: This was in thesis Page 26. $A/B$ is created by removing
  the elements of second operand set $B$ from the first operand set
  $A$ and introducing a new element {\em not} in the universe to $A$
  to represent set $B$. This is a set operation mentioned in
  \cite{mm96} (they use ``$B$'' to denote the new element to represent
  set $B$ which is also confusing) which is eventually used to
  decompose permutation of elements in $A$ to permutation of elements
  in $A/B$ and permutation of elements in $B$.  Perhaps it would be
  clearer if $b$ is replaced by $x_B$. This change has been made in
  Page 26 Definition 2.2.4 (7) and other mentions of this operation
  ``/'' Page 27 item (iii) and $h$ has been replaced with $x_H$ in
  Corollary 2.2.8. There was also a typo in Page 27 paragraph under
  Corollary 2.2.8 item (ii) where ``$U \setminus H$'' was written
  instead of $U / H$.  This has also been corrected.

\item Comment: {\bf In thesis Page 39 Definition 2.3.1: the definition
    of $A_\cF$ is not clear. Where is the dependence on $\cF$?}\\
  Response: This was in thesis Page 36. It is true, there is no
  dependence in $\cF$. It is only dependent on the universe $U$ of
  $\cF$. All occurences of $A_\cF$ has been changed to $A_U$ in Page
  36.
\end{enumerate}


\subsection{Other comments}
The following are a few observations mentioned in the review document
with my responses/changes.

\begin{enumerate}
\item Observation: (in Page 1, second last paragraph) {\bf The result
    on characterization of TPL in the thesis is a variant of the 1978
    JCT(B) result due to Fournier who showed that two hypergraphs $H_1
    = (V_1, X)$ and $H_2 = (V_2, Y)$ such that $|X| = |Y| = m$ are
    isomorphic iff there is a bijection $\phi$ on $I = \{1, ... , m\}$
    s.t. for every $L \subset I,
    |\bigcap_{i \in L}X_i| = |\bigcap_{j=\phi(i),i \in L}Y_j|$} \\
  Response: This is an important observation and was not cited in the
  thesis. Also \cite{f80} generalizes \cite{br72} and \cite{fg65} by
  characterizing the isomorphism of two hypergraphs by means of
  equicardinality of certain edge intersections and the exclusion of
  certain pairs of subhypergraphs. TPL characterization is a special
  case of this with one of the hypergraphs having hyperedges that are
  paths from a tree and the characterization only uses edge
  intersections of at most 3 hyperedges. This has been added in the
  Conclusions chapter Page 72 under {\em Graph isomorphism and
    logspace canonization}.

\item Observation: (in Page 2, first paragraph) {\bf A $k$-subdivided
    \sout{tree} [star] is a tree with a single central vertex with a number of paths
    each of $k$ edges emanating from it.}\\
  Response: $k$-subdivided star as defined in this thesis (Section 3.2.4,
  Page 43) is such that each of the aforementioned paths are of length
  ``$k+1$'' edges since each such path has $k$ nodes apart from the
  central node and the leaf node, i.e.  $k+2$ nodes. No change in the
  thesis.
\end{enumerate}


\section{Response to Reviewer 2}

\begin{enumerate}
\item Observation: {\bf The definition of $k$-subdivided star is not
    given in the thesis. What I understood is that it has a root node $v$
    of degree $k$, the degree of all other nodes is at most 2, and
    each path from root to leaf is $k+2$.}\\
Response: The definition of $k$-subdevided star was given in Section
3.2.4, Definition 3.2.6. Perhaps this has not been sufficiently
referred to, hence references to this definition has been made in
Chapter 1 Page 13 second last paragraph where it is first referred and
in Chapter 3 Page 39 second paragraph where {\sc Compute
  $k$-subdivided Star Path Labeling} problem is defined.

The reviewer's understanding is 
correct except that the root node can have any degree, thus a
$k$-subdivided star is a collection of graphs with degree of all
non-root nodes as at most 2, each path from root to leaf is $k+2$ and 
the root having any degree. See thesis Figure 1.5 in page 13 for examples.


\item Comment: {\bf The proof of Lemma 3.3.4 is unclear.}\\
Response: The proof was correct but terse. It has been elaborated to as
follows:

  {\em 
  Lemma:   Let $(\cF, \cl)$ be an \ICPPL and $\cl(S_i)=P_i$, $S_i \in \cF$, $1
  \leq i \leq 4$.  Then, $|\cap_{i=1}^4 S_i| = |\cap_{i=1}^4 P_i|$.

  Proof: Consider the set of set intersections with $S_1$ $\cF' = \{S_2 \cap S_1, S_3
  \cap S_1, S_4 \cap S_1\}$ and let $\cl'$ be a tree path labeling of
  $\cF'$ such that $\cl'(S_i \cap S_1) = P_i \cap P_1, 2 \le i \le
  4$. Clearly, $P_2 \cap P_1$, $P_3 \cap P_1$, and $P_4 \cap P_1$ are
  subpaths of path $P_1$, thus equivalent to intervals.  Due to the
  three way intersection cardinality preservation property of the
  ICPPL $(\cF, \cl)$, this new tree path labeling $(\cF', \cl')$
  preserves pairwise intersection cardinalities.  Now by applying
  Lemma 3.3.1 %~\ref{lem:intersection-cardinality} 
  to the sets in $\cF'$ and
  their corresponding path images due to $\cl'$, it follows that
  $|\cap_{i=1}^4 S_i| =| \cap_{i=1}^4 P_i|$.}


\item Comment: {\bf While studying Feasible Tree Path Labeling Problem
  in Section 3.3, two filters are mentioned and their pseudocodes
  given. If possible, state the problem in words. ... Explicitly state
the roles of these two filters in the algorithm with proper
figures.}\\
Response: The problems are described in words but they were part of a
long narrative. It has been summarised now in Page 44 before going
into details of the algorithms as follows:

{\em \begin{enumerate}
\item [Filter 1.] \filteri Refine $\cF$ such that the resulting
  labeling will not have paths that share a leaf thus each leaf being
  unique to a path. This is done by breaking the paths into subpaths and their
  corresponding preimage sets as described in Algorithm 1.
\item [Filter 2.] \filterii Find the element in universe $U$ that maps to each
  leaf in $T$ as described in Algorithm 2. 
\end{enumerate}
Remove the leaves from $T$ and their corresponding preimages from $U$
and call the filters again. This is repeated until the resulting
truncated tree is a path. The remaining mapping can be found using
ICPIA.
}

Hopefully this makes the role of the filter algorithms clear.
I felt that describing details of the pseudocode in words in the narrative would
be tedious and redundant and the pseudocode has comments that describe what is being
done whenever it is not obvious. Regarding figures, Section 3.4 in
page 52 has an example worked out in detail with figures. This was
added with the intention of illustrating the filter algorithms.


\item Comment: {\bf It is shown that TREE PATH LABELING PROBLEM on
    arbitrary trees can be solved polynomial time if the two
    subproblems, identified in section 3.6 in the thesis are
    polynomial time solvable. The formal definition of these two
    subproblems are not given. However, in page 71 item numbers 1 and
    2 the subproblems are described with several notations, where the
    motivation is unclear.}\\
Response: The prime submatrices partial order theory and notations are
required for a formal definition.
The two subproblems are defined now as \FOCPS and \FMFT and motivation is
given in Page 61 before setting up the theory and formal defintion is
given in page 66 after setting up the theory required to define it.

{\em

\begin{problemdef}{\FOCPS}{A hypergraph $\cF$ with its in-tree
    partitions $\{\bP_1,\ldots,\bP_r\}$ and
    a tree $T$.}
  Compute a partition of $T$ into subtrees $\{T_1, \ldots, T_r\}$ such
  that there exists an \ICPPL for $mub(\bP_i)$ from subtree
  $T_i$ for all $i \in [r]$ and there exists an ICPIA for all other
  prime submatrices in $\bP_i$ as claimed in Lemma 3.6.8.
\end{problemdef}

\begin{problemdef}{\FMFT}{An in-tree partition $\bP_i$ of a hypergraph
    $\cF$ and a subtree $T_i$ of a tree $T$.}
  Compute a feasible TPL for $mub(\bP_i)$ from subtree $T_i$.
\end{problemdef}
}


\item Comment: {\bf ... it is shown that consecutive 1s property
    testing for a \sout {graph} [hypergraph] is solvable in LOGSPACE. {\em This is a very
      important contribution.}}\\
Response: Since this is a conclusion from earlier results, it has not
been mentioned as a completely ``new'' result. It is mentioned in the abstract as a
derived result.
\end{enumerate}




\bibliographystyle{alpha}%
\bibliography{../lib/cop-variants__thesis} %


\end{document}