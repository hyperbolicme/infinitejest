\begin{abstract}
  \noindent {\em {\bf Keywords}:
    \hspace*{0.5em} \parbox[t]{4.4in}{consecutive ones property,
      algorithmic graph theory, hypergraph isomorphism, interval
      labeling}} \vspace*{24pt}

  \def \tem {}
  \noindent
  Consecutive-ones property is a non-trivial property of binary
  matrices that has been studied widely in the literature for over
  past 50 years. Detection of COP in a matrix is possible efficiently
  and there are several algorithms that achieve the same. This thesis
  documents the work done on an extension of COP extended from the
  equivalent interval assignment problem in \cite{nsnrs09}. These new
  results rigorously prove a natural extension (to trees) of their
  characterization as well as makes connections to graph isomorphism,
  namely path graph isomorphism.

  \tnote{EXPAND.  Abstract must be a brief about what results we have
    and how it fits
    in the body of research.\\
    -- Area: Broad to Specialized. i.e. Combinatorial algorithms ->
    Matrix reorganization -> general data reorganization
    (interval assignment) -> path assignment\\
    -- Class of problems: say, data reorganization.\\
    -- Nature of results: Is a generalization. We have a
    Polynomial algorithm for a subset of the generalization.\\
    % maybe add - we explore a natural generalization of results on
    % binary matrices with the |\tem consecutive-ones property|.  We
    % consider the following constraint satisfaction problem. Given
    % (i) a set system $\F \subseteq$ $(2^{U} \setminus \emptyset)$ of
    % a finite set $U$ of cardinality $n$, (ii) a tree $T$ of size $n$
    % and (iii) a bijection called |\tem tree path labeling|, $\cl$
    % mapping the sets in $\cF$ to paths in $T$, does there exist at
    % least one bijection $\phi:U \rightarrow V(T)$ such that for each
    % $S \in \cF$, $\{\phi(x) \mid x \in S\} = \cl(S)$?  A tree path
    % labeling of a set system is called |\tem feasible| if there
    % exists such a bijection $\phi$.  We present an algorithmic
    % characterization of feasible tree path labeling. COP is a
    % special instance of tree path labeling problem when $T$ is a
    % path.  We conclude with a polynomial time algorithm to find a
    % feasible tree path labeling of a given set system when $T$ is a
    % |\tem $k$-subdivided star|, set system has a single containment
    % tree of overlap components and set size is limited to at most
    % $k+2$.
  }
\end{abstract}
