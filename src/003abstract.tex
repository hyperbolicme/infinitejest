\xclearpage

\begin{abstract}
  \noindent {\em {\bf Keywords}:
    \hspace*{0.5em} \parbox[t]{4.4in}{consecutive ones property,
      algorithmic graph theory, hypergraph isomorphism, interval
      labeling}} \vspace*{24pt}

  \def \tem {}
  \noindent
  Consecutive-ones property is a non-trivial combinatorial property of
  binary matrices that has been studied widely in the literature for
  over past 50 years. Detection of COP in a matrix is possible
  efficiently and there are several algorithms that achieve the
  same. This thesis documents the work done on an extension of COP.
  We extend \COP from the equivalent interval assignment problem to what we
  call {\em Tree Path Labeling (TPL)} problem. Inspired by the idea
  that \COP is akin to assigning intervals to sets in a set system, we
  ask the question what if the assignments to sets are made not of
  intervals but of paths from any tree?  While we show that TPL is in
  NP, we obtain a more efficient algorithm for a special case tree
  class called {\em $k$-subdivided stars} as well as for
  characterizing a feasible TPL. For extension of \COP to
  arbitrary trees, we analyse the structure of a hypergraph (set
  system / columns of matrix) using the idea of prime submatrices and
  yield an algorithm for a feasible path labeling if one indeed
  exists.  These new results rigorously prove the natural extension
  (to trees) of the ICPIA characterization as well as makes
  connections to graph isomorphism, namely path graph isomorphism.

  \noindent
  Moreover, this thesis discusses the significance of \COP and the
  evolution of existing data structures and algorithms used in
  detection of \COP over the years -- from \PQtrees to Intersection
  Cardinality Preserving Interval Assignment (ICPIA) data
  structure. We draw conclusions and connections of which some of them
  (for instance, \PQRtree and generalized \PQtree) have not be
  observed before, to the best of our knowledge. This includes the
  fact that \COP can be tested in logspace.

  
\end{abstract}
