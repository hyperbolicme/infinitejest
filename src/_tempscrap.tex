\def\copintervalpaths{%
  \begin{tikzpicture}[every node/.style={circle,inner sep=1pt,fill=none}]%
    % \draw[help lines] (-2,0) grid (3,5);%
    \draw%
      (0,0) node[fill,circle=2pt](i1){} --%
      (1,0) node[fill,circle=2pt](i2){} --%
      (2,0) node[fill,circle=2pt](i3){} --%
      (3,0) node[fill,circle=2pt](i5){} --%
      (4,0) node[fill,circle=2pt](i6){} --%
      (5,0) node[fill,circle=2pt](i8){} --%
      (6,0) node[fill,circle=2pt](i9){} --%
      (7,0) node[fill,circle=2pt](i7){} --%
      (8,0) node[fill,circle=2pt](i4){}%
      ;%
    \draw%
      (i1) node[below](i1){$i_1$} %
      (i2) node[below](i2){$i_2$} %
      (i3) node[below](i3){$i_3$} %
      (i5) node[below](i5){$i_5$} %
      (i6) node[below](i6){$i_6$} %
      (i8) node[below](i8){$i_8$} %
      (i9) node[below](i9){$i_9$} %
      (i7) node[below](i7){$i_7$} %
      (i4) node[below](i4){$i_4$}%
      ;%

    \begin{pgfonlayer}{background}
    \node[rectangle, rounded corners=5pt, inner sep=1pt,%
      fill=red!30, opacity=0.3, draw, fit = (i1) (i2)]{}; %%
    \node[rectangle, rounded corners=5pt, inner sep=1pt,%
      fill=red!30, opacity=0.3, draw, fit = (i4) (i7)]{}; %%
    \node[rectangle, rounded corners=5pt, inner sep=7pt,%
      fill=red!30, opacity=0.3, draw, fit = (i7) (i8) (i9)]{}; %%
    \node[rectangle, rounded corners=5pt, inner sep=3pt,%
      fill=red!30, opacity=0.3, draw, fit = (i2) (i3) (i5) (i6) (i8) (i9)]{}; %%
    \node[rectangle, rounded corners=5pt, inner sep=7pt,%
      fill=red!30, opacity=0.3, draw, fit = (i2) (i3) (i5) ]{}; %%
    \end{pgfonlayer}
  \end{tikzpicture}
}%




\begin{figure}[htb]
  \centering

  \begin{tabular}[h]{l}
     \tplmatrix %        
     \\     
     \tpltreepaths %       
  \end{tabular}

  \caption[\figtabsize Partial order on prime submatrices]{\figtabsize
    Partial order on prime submatrices. The table on top illustrates
    the structural properties of prime matrices and partial order
    $\primeless$. A hypergraph $\cF$ with FTPL has sets $S_1, \ldots,
    S_{12}$ which are represented by the binary matrix $M$. The
    feasibility bijection is explicitly shown by labeling the rows
    (which represents elements in hypergraph's universe) with the
    vertices of the target tree $T$, $v_1, \ldots, v_{12}$.  $M$
    decomposed into prime submatrices $M_1, \ldots, M_6$ which is
    collectively represented by $\cP$. It can be verified that they
    correspond to the overlap components in $\bO(\cF)$. The
    containment partitions of $\cP$ are $\bP_1, \bP_2$ as shown. Note
    that the empty cells of the matrix represent 0s -- they are left
    empty to make the containment partitions more obvious. The graph
    below is the target tree with each set $S_i, 1 \le i \le 12$ shown
    as tree paths and intervals. Note that the sets that do not belong
    to the $mub$s $M_1$ and $M_4$ are shown as intervals.}
  \label{fig:hypergraphs}
\end{figure}
