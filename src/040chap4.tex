\xclearpage

\chapter{Conclusion}
\label{ch:conclusion}

\tnote[pressing]{TBD survey
  see blue notes (in notebook) under Graph Isomorphism. namely
  citations in cite:aas93 (3) a brief on heirarchy of chordal
  graphs, path graphs, interval graphs, peo, clique tree etc. and
  results. (4) interval graphs are incomparability graphs. see
  Golumbic. [tag:classification] (5) finding min length hole in
  bipartite graph is polynomial time [Sec 3.3.2 in cite:d08phd] (6)
  theorem 2.2 in cite:d08phd - $G$ is union of v.d. caterpillars iff
  $M$ has COP in rows. $M$ is edge vertex incidence matrix $(*,2)$
  of $G$
}%



\begin{center} {\tt MOVED FROM INTRO SUBSECTION IN ch:myresearch
    CHAPTER begin}
\end{center}

% \annote{
It is an interesting fact that for a matrix with the COP, the
intersection graph of the corresponding set system is an interval
graph. % }{the following text could be in Further Study/Conclusion}
A similar connection to a subclass of chordal graphs and a superclass
of interval graphs exists for the generalization of COP.  In this
case, the intersection graph of the corresponding set system must be a
{\em path graph}. Chordal graphs are of great significance, are
extensively studied, and have several applications.  One of the well
known and interesting properties of a chordal graphs is its connection
with intersection graphs \cite{mcg04}. For every chordal graph, there
exists a tree and a family of subtrees of this tree such that the
intersection graph of this family is isomorphic to the chordal graph
\cite{plr70,gav78,bp93}.  These trees when in a certain format, are
called {\em clique trees} \cite{apy92} of the chordal graph. This is a
compact representation of the chordal graph. There has also been work
done on the generalization of clique trees to clique hypergraphs
\cite{km02}.  If the chordal graph can be represented as the
intersection graph of paths in a tree, then the graph is called path
graph \cite{mcg04}.  Therefore, it is clear that if there is a
bijection from $U$ to $V(T)$ such that for every set, the elements in
it map to vertices of a unique path in $T$, then the intersection
graph of the set system is indeed a path graph.  However, this is only
a necessary condition and can be checked efficiently because path
graph recognition is polynomial time
solvable\cite{gav78,aas93}. Indeed, it is possible to construct a set
system and tree, such that the intersection graph is a path graph, but
there is no bijection between $U$ and $V(T)$ such that the sets map to
paths. Path graph isomorphism is known be isomorphism-complete, see
for example \cite{kklv10}. An interesting area of research would be to
see what this result tells us about the complexity of the tree path
labeling problem (not covered in this paper).

\begin{center} {\tt MOVED FROM INTRO SUBSECTION IN ch:myresearch
    CHAPTER\\ end}
\end{center}

We give a characterization for feasible tree path labeling of path
hypergraphs. The proof for this is constructive and computes a
feasibility bijection mapping vertices of the hypergraph to vertices
of the given target tree. This thesis also discovered an exponential
algorithm that computes a feasible TPL to a given hypergraph if it is
a path hypergraph.

Our results have close connections to recognition of path graphs and
to path graph isomorphism.  Graphs which can be represented as the
intersection graph of paths in a tree are called {\em path
  graphs}\cite{mcg04}. Thus, a hypergraph $\cF$ can be interpreted as
paths in a tree, if and only if the intersection graph of $\cF$ is a
{\em path graph}. Path graphs are a subclass of chordal graphs since
chordal graphs are characterized as the intersection graphs of
subtrees of a tree\cite{mcg04}.  Path graphs are well studied in the
literature \cite{plr70}--\cite{mcg04}\tnote{Check if cite:bp93 is
  indeed for path graphs}.  Path graph recognition can be done in
polynomial time\cite{gav78,aas93}.  Clearly, this is a necessary
condition in terms of the intersection graph of the input hypergraph
$\cF$ in \FTPL. However, one can easily obtain a counterexample to
show the insufficiency of this condition\tnote{Give a figure
  showing insuff.}.  Path graph isomorphism is known be
isomorphism-complete\cite{kklv10}. Therefore, \annote{it is
  unlikely that we can solve the problem of finding feasible path
  labeling $\cl$ for a given $\cF$ and }{WHY?} tree $T$.  It is
definitely interesting to classify the kinds of trees and hypergraphs
for which feasible path labelings can be\tnote{ELABORATE.}  found
efficiently.  \annote{These results would form a natural
  generalization of COP testing and interval graph isomorphism,
  culminating in Graph Isomorphism }{REWORD} itself.  To this effect
we consider TPL on {\kstar} and give a polynomial time solution for
the same.


While we address the computation of TPL for a given
hypergraph% \footnote{The term {\em hypergraph} is used
% instead {\em set systems} are used interchangably in this
% section. Hypergraphs that have TPLs on some target tree are called
% {\em path hypergraphs}.}
from a given target tree in this research, however optimization
problems in TPL, akin to problems in Section~\ref{sec:optcop} for COP,
is outside the scope of this thesis. Whether TPL on general trees is
solvable in $P$ remains open. So do optimization opportunities in TPL
(possible extensions to optimization for COP in the
Section~\ref{sec:optcop}.\tnote{link to section}).


% \newpage %DUMMY
\tnote{ ||Complexity challenges:|| Canonization is an important
  tool in graph isomorphism. Invention of a
  % deterministic method of canonization for any class of graphs
  % naturally results in an algorithm for isomorphism. All that is
  % required is to check if the canons of two graphs are the
  % same. Thus complexity of graph isomorphism can be studied by
  % studying canonization methods. While general graph ismorphism
  % remains elusive in terms of hardness, canonization has been
  % studied for smaller classes of graphs thus giving
  % complexity/hardness results for them.

  % In 1992, \cite{sl92} made an important discovery that tree
  % isomorphism is in logspace. It was done by inventing a method of
  % canonization of trees using a logspace depth first traversal
  % algorithm. \tnote{confirm the statement of claim}.

  % \cite{kklv10} proved that interval graph canonization is also in
  % logspace thus drawing logspace conslusions about COT.
  % % This result is
  % % built on the fact that
  % Interval graphs are FP+C (fixed point with counting) definable
  % \tnote{elucidate} and \cite{l10} showed that this implies that it
  % captures PTIME. \tnote{elucidate} This result along with that of
  % undirected graph reachability being in logspace \cite{rei08},
  % \cite{kklv10} proved their logspace result.
}%
\tnote{ (1) the interesting survey in conclusion section of kklv10
  (2) [[[COP helping some problems with hardness]]] having COP in the
  input structure makes some problems less hard than in a general
  input. -- min set cover problem is known to be NP hard. so is min
  hitting
  set. these two problems are equivalent cite-adp80\\
  -- -- complexity results: log factor polynomial time approximable cite-f98, W[2]-complete with solution size as parameter cite-df99\\
  -- a couple of generalizations:\\
  -- -- min degree hypergraph (MDH) problem\\
  -- -- red blue covering (RBSC) problem\\
  ----- cite-d08phd Chapter 5.\\
  ----- what are cite-d08phd results\\
  ----- Rectangle stabbing \\
  ----- see cite-d08phd Chapter 6?  }

