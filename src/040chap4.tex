\xclearpage

\chapter[Conclusion]{Conclusion and Further Research}
\label{ch:conclusion}

In this thesis, we surveyed \cop testing algorithms and as new results
we gave a characterization for a generalization of \COP, namely,
feasible tree path labeling of path hypergraphs. The proof for this is
constructive and computes a feasibility bijection mapping vertices of
the hypergraph to vertices of the given target tree. We also described
an exponential algorithm that computes a feasible TPL to a given
hypergraph if it is a path hypergraph.  In this chapter we will see a
few ideas where tree path labeling, in our experience, shows promising
scope for further research.

\parindent 0pt

\textbf{Graph isomorphism and logspace canonization.}  Canonization is
an important tool in graph isomorphism. Invention of a deterministic
method of canonization for any class of graphs naturally results in an
algorithm for isomorphism. All that is required is to check if the
canons of two graphs are the same. Thus one way to study the
complexity of graph isomorphism is by studying canonization
methods. While general graph isomorphism remains elusive in terms of
hardness, canonization has been studied for smaller classes of graphs
thus giving complexity or hardness results for them -- some of the
canonization results are \cite{sl92,dlntw09,adkk09,kklv10}.

Ten years ago, \cite{sl92} made an important discovery that tree
isomorphism is in \logspace. It was done by inventing a method of
canonization of trees using a logspace depth first traversal
algorithm.  Less than a couple of years ago, \cite{kklv10} proved that
interval graph canonization is also in \logspace thus drawing logspace
conclusions about \COT. Interval graphs are \fpplusc (fixed point
logic with counting) definable and \cite{l10} showed that this implies
that it captures \ptime. This result along with that of undirected
graph reachability being in logspace \cite{rei08} helped \cite{kklv10}
conclude their logspace result.

How the work in this thesis connects to graph isomorphism and
canonization is as follows. For a matrix with the \COP, the
intersection graph of the input hypergraph is an interval graph.  A
similar connection exists for \TPL to path graphs which is a subclass
of chordal graphs and a superclass of interval graphs. Graphs which
can be represented as the intersection graph of paths in a tree are
path graphs (as seen in Section~\ref{sec:basicprelim}).  % In TPL, the
% intersection graph of the input hypergraph must be a path graph.  % Our
% results have close connections to recognition of path graphs and to
% path graph isomorphism.  
Thus, a hypergraph $\cF$ can be labeled with paths in a tree (path
labeling as defined in Section~\ref{ch:prelims}), \iff the
intersection graph $\bI(\cF)$ is a path graph. Path graphs are a
subclass of chordal graphs since chordal graphs are characterized as
the intersection graphs of subtrees of a tree \cite[Ch.~4,
Sec.~5]{mcg04}. Chordal graphs are of great significance, are
extensively studied, and have several applications. For every chordal
graph, there exists a tree and a family of subtrees of this tree such
that the intersection graph of this family is isomorphic to the
chordal graph \cite{jrw72phd,gav74b,bun74}.  These trees when in a
certain format, are called {\em clique trees} \cite{ppy92} of the
chordal graph. This is a compact representation of the chordal
graph. There has also been work done on the generalization of clique
trees to clique hypergraphs \cite{km02}.  Connections between
hypergraph isomorphism and properties of set systems is also explored
in \cite{br72} by extending Whitney's theorem of graph isomorphism to
hypergraphs.  Moreover, \cite{f80} generalizes \cite{br72} and
\cite{fg65} by characterizing the isomorphism of two hypergraphs by
means of equicardinality of certain edge intersections and the
exclusion of certain pairs of subhypergraphs.  Thus ICPPL is also a
variant of the characterization in \cite{f80} with one of the
hypergraphs having hyperedges that are paths from a tree (path
hypergraph) and ICPPL uses only equicardinality of edge intersections
of at most three hyperedges.

Path graphs are well studied
in the literature (characterization for \uvgraphs was perhaps first
mentioned by \cite{plr70}).
% Clearly, this is a necessary condition in terms of the intersection
% graph of the input hypergraph $\cF$ in \FTPL. However, one can easily
% obtain a counterexample to show the insufficiency of this
% condition\tnote{Give a figure showing insuff.}.
Indeed, path graph recognition is a necessary condition for \CFTPL but
not sufficient.  We know that path graph recognition can be done in
polynomial time \cite[first algorithm.
$O\left(pn^3\right)$]{gav78},\cite[$O\left(p(m+n)\right)$]{aas93}
\cite{sb94}, \cite[linear algorithm]{db95}, \cite[PR-trees]{cha11}
\footnote{$n$ is number of vertices, $m$ is number of edges, $p$ is
  number of maximal cliques in the input graph.}.  Path graph
isomorphism is also known be \gicomplete \cite{bpt96}.  It would be
interesting to pursue research to answer the following questions in
relation to TPL and graph isomorphism.
\begin{enumerate}
\item What does path graph isomorphism hardness tell us about the
  complexity of \CFTPL?
\item Can TPL help in computing a canon for path graphs, especially a
  logspace canon?
\end{enumerate}



%\noindent
\textbf{Complexity of TPL in other tree classes.}  We know that on
intervals, \CFTPL is easy (Section~\ref{sec:icpplicpia}).  We showed
that \CFTPL on {\kstar} is also polynomial time solvable in
Section~\ref{sec:ksubdivstar}. However, due to its close connections
with path graph isomorphism which is \gicomplete \cite{kklv10}, we
conjecture that \CFTPL on general trees cannot do better than
\gicomplete. Our algorithm in Section~\ref{sec:norestraint} is unable to
solve it in \p. It would therefore, be certainly of interest to
classify the kinds of trees and hypergraphs for which feasible path
labelings can be found efficiently, or better still, to solve \CFTPL
for general trees itself in \p-time or prove that it is not in \p.


%\noindent
\textbf{Optimization problems in TPL.}  There are optimization
problems for matrices that do not have \COP which we only mention in
Section~\ref{sec:optcop} but did not discuss further since it is not
in the scope of the main work of this thesis. We mention them here
again to encourage further analogous research in optimization problems
of \TPL.  The central questions in this area are (i) how close is the
matrix to having \COP, mainly in terms of Tucker's forbidden
submatrices, and (ii) how optimally can one alter the matrix to attain
\COP.  With regard to the latter question, recent literature indicate
that there has been a lot of interest in matrix modification problems
to make a matrix have \COP \cite{hg02, tz07}. While we address the
computation of TPL for a given hypergraph from a given target tree in
this research, however optimization opportunities in TPL are yet to be
explored and their complexity remains open. We suggest a few
optimization ideas here and there could be many more such questions
that could be asked by a motivated researcher. 
\begin{enumerate}
\item If \CFTPL is not possible with the input target tree $T$, 
  can $T$ be changed to get a feasible TPL?
\item What is the characterization of a hypergraph that will fail the
  \CFTPL problem for any target tree? \ie what is the characterization
  of a hypergraph that is not a path hypergraph? Can a forbidden
  substructure be found?
\item If a given hypergraph is not a path hypergraph, can at most $k$
  vertices be deleted from the hypergraph to make it a path
  hypergraph? A dual problem would be to select at least $k$ vertices
  to induce a path hypergraph? A different flavour of the problem
  would be deleting (or selecting) hyperedges in the input to get a
  path hypergraph (analogous to {\sc Min COS-R, Min COS-C, Max
    COS-R, Max COS-C} problems in \COP).
\item Can at most $k$ vertex memberships in hyperedges be
  reversed to get a path hypergraph? (analogous to {\sc COP
    Augmentation} problem).
\end{enumerate}
It is unlikely that these problems will have efficient exact
algorithms since their \COP counterparts are all NP hard
problems. Thus the scope of research would be to find approximation
algorithms or fixed parameter algorithms or hardness results.



% %\noindent

% \temptext{ 
%   (2) [[[COP helping some problems with hardness]]] having COP in the
%   input structure makes some problems less hard than in a general
%   input. -- min set cover problem is known to be NP hard. so is min
%   hitting
%   set. these two problems are equivalent cite-adp80\\
%   -- -- complexity results: log factor polynomial time approximable
%   cite-f98, W[2]-complete with solution size as parameter cite-df99\\ 
%   -- a couple of generalizations:\\
%   -- -- min degree hypergraph (MDH) problem\\
%   -- -- red blue covering (RBSC) problem\\
%   ----- cite-d08phd Chapter 5.\\
%   ----- what are cite-d08phd results\\
%   ----- Rectangle stabbing \\
%   ----- see cite-d08phd Chapter 6?  }
 
%\noindent
\textbf{x-dimensional generalization of \COP.} As seen in the chapters
of this thesis \TPL is a graph theoretic generalization of
\COP. Intervals are target trees with maximum degree 2 and general
trees have no bound on maximum degree. Another way to look at
intervals is using geometry. Intervals can be imagined in one
dimension as line segments along a single line in the Euclidean space
and the elements of hypergraph universe are coordinate points on this
line segment. The generalization of this visualization is obvious --
how does the interval labeling property generalize when each hyperedge
must be mapped to coordinate points bounded by a rectilinear 2-D shape
on a single plane, say rectangles (analogous to coordinate points on
line segments in interval labeling problem). One hardness result is
already known -- the intersection graphs of axis-parallel boxes in
$\bR^d$ is GI-complete \cite{ueh08}.